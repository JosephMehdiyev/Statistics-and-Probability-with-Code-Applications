\chapter{Bootstrap}
Bootstrap is a technique about generating more sample data from existing data. It is used for computing confidence intervals and estimating standard errors. The idea is simple. We have sample data $X_1, \ldots X_n$ from unknown distribution $F$. First we estimate $F$ with $\widehat{F}_n$, then we draw random resamples from $\widehat{F}_n$ multiple times and calculate our wanted statistical functions.
\section{Basic Introduction}
\begin{definition}\textbf{Bootstrapping}\\
    Let $X = \{X_1, \ldots X_n \}$ be our sample data from unknown distribution $F$. That is,
    \[X_1,X_2, \ldots X_n \sim F \]
    Let' s assume we are interested in $T(X)$. In practical statistical problems, we need to know about the distribution of $T(X)$. We can estimate our unknown distribution $F$ with $\widehat{F}_n$, and draw random samples from that known distribution, that is,
    \[X^*_1, X^*_2, \ldots, X^*_n  \sim \widehat{F}_n\]
    and compute $T^*_n$ from these samples.
\end{definition}
\section{References}
\begin{enumerate}
    \item \url{https://stats.stackexchange.com/questions/26088/explaining-to-laypeople-why-bootstrapping-works}
    \item \url{https://ocw.mit.edu/courses/14-384-time-series-analysis-fall-2013/2fdf997bca65d6ed82ba7a94f6cdc970_MIT14_384F13_lec9.pdf}
\end{enumerate}

