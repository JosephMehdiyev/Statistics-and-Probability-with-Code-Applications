\chapter{Random Variables}
``A random variable can be compared with the holy roman empire: The Holy Roman Empire was not holy, it was not roman, and it was not an empire.'' - Some random guy in internet \\ 

From the first chapter, we have been using events and sample spaces to develop the idea of probability and calculating it. But, in practical world,  we have to link the events and sample spaces to \textbf{data}. This concept is called ``Random Variable'' or r.v shortly.
A Random Variable, in informal terms, places the $\Omega$ in real line so we can work with it more easily. There are still events, but in terms on Random Variables now.
\\From now on, we may write Random Variables as r.v or r.v. or r.vs or r.v or r.vs shortly for convenience.


%----------------------------------------------------------

\section{Introduction to Random Variables}

A Random Variable describes the data or the outcome $\omega$ as a real number. There is a reason this concept exists, since it opens new concepts for practical applications.
\newline
Let's begin with the formal definition of \textit{ Random Variable}.

\begin{definition}
    A \textbf{Random Variable X} is a function,
    $$ X \ : \ \Omega \rightarrow \mathbb{R}$$
    That assigns a real number $X(\omega)$ to each outcome $\omega$. \\
    In \textbf{layman terms}, a random variable is a way to assign a numerical code to each possible outcome. A r.v is \textbf{neither random, nor a variable.} It is just a function. 
\end{definition}

This concept is heavily used instead of sample spaces . From now on, sample space will be mentioned rarely. Think this way, when we work on functions in algebra or sometimes in calculus, we don't think about about the domain of the function, but the properties of function itself.
Here are some examples to understand the concept better.
\begin{example} Flip a coin. We know that $\Omega =\{H,T\}$. A r.v $X$ might assign $X(H) = 1$ and $X(T) = 0$. That is, heads is ``coded'' as 1 and tails is ``coded" as 0.
    
\end{example}
\begin{example}
    Flip a fair coin $n$ times. Let $X$ represent the number of heads we get. Then, $X$ is a random variable that takes values $\{0,1,2,...,n\}$.
\end{example}

\begin{example}
    Toss a fair six sided dice 2 times. Let $X$ be the sum of the two rolls we get. Then, $X$ is a random variable that takes values $\{2,3,4,...,12\}$.
\end{example}

\begin{example}
    A students wants to write a real number in intervals $[0,1]$. Let $X$ be the number student writes. Then, $X$ is also a random variable that takes any real numbers in that interval.
\end{example}

As you may guess, this extremely looks similar to events. Random Variables also have \textit{Independence, Conditional Random Variable, a probability function} and so on. Additionally , Random Variables can be either \textbf{Discrete} or \textbf{Continous}.
\par 
Discrete Random Variable's range is finite or countably infinite. The first two examples we gave are Discrete. Continuous Random Variables's range is uncountably infinite like the third example \newline

I want to emphasize  that Random Variables are neither random or a variable, they are functions. It is a bit hard to grasp the idea of this concept, so I highly recommend lurking in mathematical forums and try to understand it ( that is what I did). But in short, we use Random Variables instead of outcomes, since Random variables are \textbf{numbers}. Numbers are easier to work with, we can process the numbers, do algebraic operations to them, also they have a structure that outcomes do not.
Turn \textbf{ Example 2.1.1} to in sample space and events language, which is easier to work with? Bunch of $H,T$ or just a number?


%--------------------------------------------------------



\section{Distribution Functions c.d.f, p.m.f, p.d.f, p.p.f}
\subsection*{c.d.f and p.p.f}
We define  \textbf{Cumulative Distribution Function} as,

\begin{definition}
    The \textbf{Cumulative Distribution Function} or shortly \textbf{c.d.f} is a function $F_X \ : \ \mathbb{R} \rightarrow [0,1]$ such that
    \[ F_X(x) = P( X \le x) \]
    \textbf{Remark :} Every r.v (discrete and continuous) have c.d.f. For this reason, we can use c.d.f for unified treatment of r.v properties (that is, generalized concepts for all r.v). Moreover, c.d.f contains all the information about r.v, both continous and discrete ones. That is why c.d.f is very useful, even in practical world.
\end{definition}
\par
In informal terms, c.d.f is the probability that $X$ will take a value less than or equal to $x$. This property holds both for continuous and discrete r.v.


\begin{example}
    We toss a fair coin two times. Let $X$ represent the number of heads we get. Then c.d.f of $X$ is,
    \[F_X(x) = 
        \begin{cases} 
          0 \qquad &x < 0\\
          1/4 \qquad &0 \le x <1 \\
          3/4 \qquad &1 \leq x < 2 \\
          1 \qquad &x \ge 2
        \end{cases} 
\]
\end{example}

The variable $x$ can get $\textbf{any real numbers}$, such as $2, 4.14$ and $\pi$. It is just that in discrete case the probability equals to $0$ (Which in other words converts discrete to continous function?). It a bit tricky, they simply take the values from corresponding inequalities.
\\
Now, let's look at some properties of c.d.f,
\begin{theorem}
    Let $X$ have \textit{c.d.f} $F$ and $Y$  have \textit{c.d.f} $G$. If $F(x)=G(x)$ for all $x$, then,
    \[ P(X \in A) = P(Y \in A) \quad \text{for all $A$} \] 
\end{theorem}

\begin{theorem}
    the function $F \ : \mathbb{R} \rightarrow [0,1]$ is a c.d.f for some r.v if and only if $F$ satisfies three conditions:
    \begin{enumerate}
        \item $F$ is non-decreasing
        \item $F$ is normalized i.e \[ \lim_{x \rightarrow -\infty} F(x) = 0\ \ \land \ \lim_{x \rightarrow \infty} F(x) =1 \]
        \item $F$ is right continuous.
    \end{enumerate}
    \begin{proof}[Remarks]
        The first and second properties are simple and intuitive, therefore we will ignore them ( I know it is not the mathematical way, but whatever   ). \\
        
        Third property, however, is worth having discussion about. This property directly follows from the inequality $\le$. We could even define c.d.f with strict inequality $F= P(X<x)$, and it would still work. It is matter of convention.
    \end{proof}
\end{theorem}
\begin{definition}
    \textbf{Quantile Percent point function}, or shortly p.p.f, is defined as inverse of c.d.f i.e,
    \[Q(x) = F^{-1}(x)\]
\end{definition}
%------------------------------
\subsection*{c.d.f and p.d.f}
Similar to probabilities of Events, we can calculate probability of $X$ , depending on discrete or Continuous with functions called \textbf{ Probability Mass Function} and $\textbf{Probability Density Function}$, shortly $\textbf{p.m.f}$ and $\textbf{PDF}$ respectively,

\begin{definition}
    If $X$ is discrete, and it takes \textit{countably} values $ \{ x_1,x_2,..,
    x_n \}$ we define \textbf{Probability Mass Function} of X as follows:
    $$f_X(x)= P(X = x)$$
    Note that $P(X = x)$ is a function, not a number. We have to specify $x$ first to get a number.
    \textbf{Remark}: $ \{ X=x \} $ are disjoint events that form partition of $\Omega$.
\end{definition}

   With the properties of probability, we have $f_X \ge 0$ for all $x \in \mathbb{R}$ and $\sum_{i} f_X (x_i) =1 $. Let's revisit our Example 2.2.1

   \begin{example}
    We toss a fair coin two times. Let $X$ represent the number of heads we get. Then c.d.f of $X$ is,
    \[f_X(x) = 
        \begin{cases} 
          1/4 \qquad &x=0\\
          1/2 \qquad &x=1 \\
          1/4 \qquad &x=2\\
          0 \qquad &\text{otherwise}
        \end{cases} 
\]
\end{example}



Moreover, for any set of real numbers, $S$, we have
\[ P (X \in S) = \sum_{x \in S} f_X(x)\]
Since all $\{X = x \}$  are disjoint.\\
\par
We can apply similar rules to continuous r.vs,
\begin{definition}
    If $X$ is continuous, we can represent the probability distribution of $X$ with,
    \[ P(a < X < b) = \int_{a}^{b} f_X(x) dx \]
    Function  $f_X$ is called \textbf{Probability Density Function} or PDF as shortly.
\end{definition}
Nothing new here really, we just change the properties of p.m.f that we can use it on continuous r.vs. Now, let's look at some examples,\\
\par
You may noticed that c.d.f is similar to p.m.f and PDF. Indeed, the are related, c.d.f is just sum of these functions we defined over some interval $x$.
\begin{definition}
    c.d.f is related to p.m.f and PDF. For discrete r.vs,
    \[F_X(x) = P(X \le x) = \sum_{x_i \le x} f_X(x_i)\]
    And for continuous r.vs,
    \[F_X(x)= \int_{-\infty}^x f_X(x)dx \]
    And $f_X(x) = F_X^{'}(x)$ for for all differentiable points $x$.\\
  
    Note that this definition is heavily used instead of direct definitions above, since we can work with c.d.f only, and derive it to get needed functions.
\end{definition}
\par



%-------------------------------------------------------------------------------------


\section{Important Random Variables and their distribution}
\begin{definition}
    If $X$ has distribution $A$, we write
    \[ X \sim A\]
    Usually $A$ depends on some fixed numbers to define properly, we call them $\textbf{parameters}$. For example, the distribution \textbf{Bernoulli}$(p)$ has parameter $p$. We show parameters in c.d.f and p.m.f as,
    \[ f(x; parameters) \quad \text{and} \quad F(x; parameters) \]
    
\end{definition}
There are some specific examples of r.v. that are very useful in practical applications. We will show most important ones, and briefly discuss them. In later chapters, we will learn more about them. Note that we will write the notation with the name of the distribution.

\subsubsection{Degenerate distribution or Point mass distribution: $X \sim \delta_a$}
Consider tossing coin or dice where all the sides show the same value. The p.m.f is ,
\[f_X(x; \delta_a) = 1 \quad \text{for} \ x = a\]
You might guess why it is called degenerate sometimes. It is not random, but the distribution satisfies the definitions!
\subsubsection{Discrete Uniform distribution}
This distribution is the one of the most known ones. When there are finitely many values and each of them have the same probability, then $p = \frac{1}{n}$. Simple coin tossing, dice rolling are prime example of these. The p.m.f is,
\[f_X(x) = \frac{1}{n}\]
Where $x \in \{1,2,...,n \}$. Nothing new here.
for other cases, $f_X(x) = 0$. 
\subsubsection*{Bernoulli distribution: $X \sim \op{Bernoulli}(p)$ }
This distribution describes ``Yes or No'' type of experiments such as coin flipping. Therefore, $P(X = 1) = p$, $P(X = 0) = 1-p$. We can also calculate p.m.f,
\[f_X(x; p) = p^x(1-p)^{1-x} \quad \text{for} \ x \in \{0,1\}\]

\subsubsection*{Binomial distribution:  $ X \sim \op{Binomial}(n,p)$}
This distribution is generalized form of \textbf{Bernoulli distribution}. Similar to Bernoulli, this distribution describes ``Yes or no'' type of experiments, but for $n$ times of tries e.g tossing a coin $n$ times. Assuming tries are independent of each other, we can show p.m.f as,
\[f_X(x; n,p) = \binom{n}{x}p^x(1-p)^{n-x} \quad \text{for} \ x \in \{0,1,...,n\}\]
Notice that Binomial$(1,p)$ = Bernoulli$(p)$.
\subsubsection*{Geometric distribution: $X \sim \op{Geom}(p)$}
This distribution is also specified with Bernoulli. The geometric distribution describes the probability of the first occurrence of success requires after $x$ independent trials e.g getting the first head after $x$ tosses. The p.m.f is,
\[ f_{X}(x; p) = (1-p)^{x-1}p \quad \text{for} \ x \ge 1 \]
\subsubsection*{Poisson distribution: $X \sim \op{Poisson}(\lambda)$}
This distribution is mainly used for counts of events like photons hitting a detector in a time interval, number of car accidents, students achieving a low and high mark on exam, or number of pieces of chewing gum on a tile of a sidewalk. its p.m.f is,
\[f_X(x; \lambda) = e^{-\lambda} \frac{\lambda^{x}}{x!} \quad \text{for} \ x \ge 0 \]
Usually, $\lambda = rt$ where $r$ is average rate the events occur and $t$ is the time interval. The r.v $X$ represents the number of events.
\subsubsection*{Unfiorm distribution: $X \sim \op{Uniform}(a,b)$}
The p.d.f of $X$ is defined as,
\[f_X(x; a,b)= \frac{1}{b-a} \quad \text{for} \ x \in [a,b]\]
\subsubsection*{Normal (Gaussian) distribution: $X \sim N(\mu, \sigma^2)$ or $X \sim \mathcal{N}(\mu, \sigma^2)$}
This distribution is one of the most popular ones even for non-mathematician, layman people. The famous IQ graph, badly made ``memes'' are example of this. This distribution plays important role in statistics and probability. Moreover, we can observe this distribution in nature. p.d.f is defined as,
\[f_X(x; \mu, \sigma^2) = \frac{1}{\sigma \sqrt{2 \pi}}  e^{- \frac{(x- \mu)^2}{2\sigma^2}} \quad \text{for} \ x \in \R \]
We will learn about $\mu$ and $\sigma$ in later chapters. For now, just assume they are some random parameters. \\
We define  \textbf{standart Normal distribution} as $N(0,1)$, r.v as $Z$. This specific distribution is very important, so much that we show its p.d.f and c.d.f with new notation, namely $\phi(z)$ and $\Phi(z)$. There is no closed form expression for $\Phi(z)$. In modern days, the programming libaries calculate them. \\
It can be shown that we can show \textbf{any normal  probabilities} we want with $\Phi(z)$.
\subsubsection*{Exponential distribution: $X \sim \op{Exp}(\lambda)$}
This distribution is continous analogue of the geometric distribution. This distribution (and the ones after this) has complex properties so we will be brief with this.
We define its c.d.f as ,
\[f_X(x; \lambda) = \lambda e^{-x \lambda}\]
\subsubsection*{Gamma distribution: $X \sim \op{Gamma} (\alpha, \beta)$}
First, we start with a definition,
\begin{definition}
    For $\alpha > 0$, we define \textbf{Gamma function} as,
    \[\Gamma(\alpha) = \int_{0}^{\infty} t^{\alpha-1} e^{-t}dt\]
    It is generalized form of more simple and specific version, 
    \[ \Gamma(n) = (n-1)! \quad \op{for} \ n \in \mathbb{N} \]
\end{definition}
    We define p.d.f of $x$ as,
    \[f_X(x; \alpha, \beta) =  \frac{1}{\beta^{\alpha} \Gamma(\alpha)} x^{\alpha-1} e^{-x/ \beta} \quad x > 0\]
    Notice that $ \operatorname{Gamma}(1, \beta) =\operatorname{Exp}( \beta)$. That is, exponentional distribution is a specific case of gamma distribution.
    Gamma distribution itself is very useful and heavily used in this field. Moreover, we can derive more advanced ( which I have difficulties understanding) distributions from the gamma function. For now, we will end our discussion here.
%--------------------------

\section{Multivariate Distribution}
In practical word, we often work with multiple r.v in the same experiment. This can be a medical research with multiple tests, where tests are related with each other with the same sample space $\Omega$ and the same probability.\\
First, we define a special vector,
\begin{definition}
    Let $X_1,X_2,..,X_n$ be r.vs. We call $X = \{ X_1, X_2, ... ,X_n\}$ \textbf{a random vector}.
\end{definition}
\begin{definition}
    If r.vs $X_1,X_2,...,X_n$ are \textbf{independent} an have the same \textbf{marginal distribution} with c.d.f $F$, we define these r.vs as \textbf{independent and identically distributed}, shortly $\op{IID}$, with notation,
    \[X_1,...,X_n \sim F\]
    similarly, we show the p.d.f the same way. IID property is very important in statistical field.
\end{definition}
We can apply multivariate c.d.f as
\begin{definition}
    For $n$  r.v $\{ X_1,X_2,..,X_n \}$, the multivariate c.d.f $F_{X_1,X_2,...,X_n}$ is given by,
    \[F_{X_1,X_2,...,X_n}(x_1,x_2,...,x_n) = P(X_1 \le x_1,...,X_n \le x_n) \]
\end{definition}
There is nothing fancy here, actually. We simply redefine c.d.f in general sense for $n$ r.vs.
\par
Similarly, we can define multivariate p.m.f as,
\begin{definition} For random vector $X$,, the multivariate p.m.f  $f_{X_1,X_2,...,X_n}$ is given by,
    \[ f_{X_1,X_2,...,X_n}(x_1,x_2,...,x_n) = P(X_1=x_1 ,...,  \ X_n = x_n) \]
    This is generalized form of p.m.f

\end{definition}
Similarly, we define,
\begin{definition}
    We know that c.d.f and p.d.f are related by derivative. Then,
    For random vector $X$, the multivariate p.d.f $f_{X_1,..,x_N}$ is given by,
    \[f_{X_1,X_2,...,X_n}(x_1,x_2,...,x_n) = \frac{{\partial}^nF_{X_1,..,X_n}(x_1,...,x_n)}{\partial x_1\partial x_2...\partial x_n}\]
\end{definition}

The Properties and theorems are similar, but are generalized for $n$ r.vs.

\section{Marginal Distribution}
If more than one variable is defined in an experiment, it is important to distinguish between the multivariate probability of $(X_1,X_2,..,X_n)$ and individual probability distributions of $X_1,X_2,..,X_n$\\

Formally, \textbf{Marginal distribution is the probability of a single event (or r.v) occuring, independent of other events}. Therefore implementing marginal distributions are rather easy. In multivariate distributions, we redefine the needed variable as a ``constant'' and work with other variables only.
\begin{definition}
    If $X$ is a random vector with p.m.f $f_{X_1,X_2,..,X_n}$, then we define marginal distribution as,
    \[f_{X_1}= P(X_1 = x_1)= \sum_{x_1 \ constant} P(X_1=x_1,..,X_n=x_n)= \sum_{x_1\ constant} f_{X_1,X_2,..,X_n}(x_1,x_2,...,x_n)\]
\end{definition}
Similarly,
\begin{definition}
    We define marginal p.d.f as ,
    \[f_{X_i}(x_i) =  \int \int \int ... \int f(x_1,x_2,..,x_n)dx_1..dx_{i-1}dx_{i+1}...dx_n \]
\end{definition}
Similarly, c.d.f follows the same rule. $F_X(x)= F(x,a,b,c,...)$. \\
\textbf{Remark:} Marginality and conditionality are not the same thing. They look similiar, but their definitions are subtly different.
\par
%------------------------
\section{Independence}
Similar to events, r.vs also can be independent,
\begin{definition}
    Two r.vs $X$ and $Y$ are \textbf{independent} if, for every $A$ and $B$,
    \[P(X \in A, Y \in B)= P(X \in A)P(Y \in B)\]
    The definition persists for multivariate distributions.
\end{definition}
To check Independence, we need to check the above question for every subsets $A,B$. Additionally, we have the theorem,
\begin{theorem}
    Let $X$ and $Y$ have p.m.f $f_{X_y}$. THen $X$ and $Y$ are independent only and only if ,
    \[f_{X,Y}(x,y)=f_X(x)f_Y(y) \]
    The definiton persists for multivariate distributions.
\end{theorem}


%--------------------------
\section{Conditioning}
Similar to events, r.v $X$ can also have conditional distributions given that we have $Y=y$. We show the conditionality with,
\begin{definition}
    We can show conditional distribution of $X$ respect to $Y$ with,
    \[P(X=x| Y=y) = \frac{ P(X=x,Y=y)}{P(Y=y)} \]
\end{definition}
Moreover we can also define \textbf{conditional p.m.f} as,
\begin{definition}
    p.m.f of $X$ conditional respect to $Y$ can be written as 
    \[ f_{X|Y}(x|y)= \frac{f_{X,Y}(x,y)}{f_Y(y)}\]
\end{definition}
 

%---------------------
\section{Transformations of a Random Variable}
In some applications, we really are interested in distributions of some function of $X$. We call this concept \textbf{Transformation of X}.
\begin{definition}
    Let $X$ be r.v with PDF/p.m.f $f_X$ and c.d.f $F_X$. Let $Y=r(X)$ i.e $Y=X^2$ or $Y = \ln X$. We call $Y=r(X)$ \textbf{transformation of x}. \\
    \newline
    If $Y$ is discrete, p.m.f is given by,
    \[f_Y(y)= P(Y=y) =  P( r(X) = y) = P( \{x \ : \ r(x) = y\})= P( X \in r^{-1}(y)) \]
    \newline
    If $Y$ is continuous, we first calculate c.d.f and find derivative of it. 
    \begin{align*}
        F_Y(y) &= P( Y \le y) = P(r(X) < y) \\
               &= P(\{x \ : \ r(x) \le y \}) = P( A_y)    \\
               &= \int_{A_y} f_X(x)dx
    \end{align*}
    And the last step, $f_Y(y) = F^{'}(y)$.
\end{definition}
We can also generalize this concepts for Multivariate distributions, which we just increase dimensions we work with (too lazy, add this later).

\section{Exercises}
    1. Let $X \sim N(0,1)$ and $Y= e^X$.
    \begin{enumerate}
        \item plot p.d.f of $X$ and $Y$.
        \item (The practical experiment) let $x$ be a vector with 10000 numbers randomly selected from $N(0,1)$, i .e $x = (x_1,..,x_{10000})$. Let $y$ be a similiar vector where $y_i = e^{x_i}$, and plot the result of the histogram. Compare with the theorical $p.d.f$
    \end{enumerate}
    \textbf{Solution1}    \embed{src/chapter2/1.py}{Source}  \\
    First, let's derive $f_Y(y)$.
    \begin{align*}
        F_Y(y) = P(Y \le y) & = P (e^X \le y) = P( X \le \ln y) \\
                            &= F_X(\ln y ) = \Phi( \ln y) \Longrightarrow \\
                            f_Y(y) = \odv{\Phi(\ln y)}{y} &= \odv{\Phi(\ln y)}{\ln y} \odv{\ln y}{y} = \phi(\ln y) / y
    \end{align*}

    \inputminted{python}{src/chapter2/1.py} \\
    \begin{center} %% Creator: Matplotlib, PGF backend
%%
%% To include the figure in your LaTeX document, write
%%   \input{<filename>.pgf}
%%
%% Make sure the required packages are loaded in your preamble
%%   \usepackage{pgf}
%%
%% Also ensure that all the required font packages are loaded; for instance,
%% the lmodern package is sometimes necessary when using math font.
%%   \usepackage{lmodern}
%%
%% Figures using additional raster images can only be included by \input if
%% they are in the same directory as the main LaTeX file. For loading figures
%% from other directories you can use the `import` package
%%   \usepackage{import}
%%
%% and then include the figures with
%%   \import{<path to file>}{<filename>.pgf}
%%
%% Matplotlib used the following preamble
%%   \def\mathdefault#1{#1}
%%   \everymath=\expandafter{\the\everymath\displaystyle}
%%   
%%   \usepackage{fontspec}
%%   \setmainfont{DejaVuSerif.ttf}[Path=\detokenize{/usr/lib/python3.12/site-packages/matplotlib/mpl-data/fonts/ttf/}]
%%   \setsansfont{DejaVuSans.ttf}[Path=\detokenize{/usr/lib/python3.12/site-packages/matplotlib/mpl-data/fonts/ttf/}]
%%   \setmonofont{DejaVuSansMono.ttf}[Path=\detokenize{/usr/lib/python3.12/site-packages/matplotlib/mpl-data/fonts/ttf/}]
%%   \makeatletter\@ifpackageloaded{underscore}{}{\usepackage[strings]{underscore}}\makeatother
%%
\begingroup%
\makeatletter%
\begin{pgfpicture}%
\pgfpathrectangle{\pgfpointorigin}{\pgfqpoint{6.000000in}{3.500000in}}%
\pgfusepath{use as bounding box, clip}%
\begin{pgfscope}%
\pgfsetbuttcap%
\pgfsetmiterjoin%
\definecolor{currentfill}{rgb}{1.000000,1.000000,1.000000}%
\pgfsetfillcolor{currentfill}%
\pgfsetlinewidth{0.000000pt}%
\definecolor{currentstroke}{rgb}{1.000000,1.000000,1.000000}%
\pgfsetstrokecolor{currentstroke}%
\pgfsetdash{}{0pt}%
\pgfpathmoveto{\pgfqpoint{0.000000in}{0.000000in}}%
\pgfpathlineto{\pgfqpoint{6.000000in}{0.000000in}}%
\pgfpathlineto{\pgfqpoint{6.000000in}{3.500000in}}%
\pgfpathlineto{\pgfqpoint{0.000000in}{3.500000in}}%
\pgfpathlineto{\pgfqpoint{0.000000in}{0.000000in}}%
\pgfpathclose%
\pgfusepath{fill}%
\end{pgfscope}%
\begin{pgfscope}%
\pgfsetbuttcap%
\pgfsetmiterjoin%
\definecolor{currentfill}{rgb}{1.000000,1.000000,1.000000}%
\pgfsetfillcolor{currentfill}%
\pgfsetlinewidth{0.000000pt}%
\definecolor{currentstroke}{rgb}{0.000000,0.000000,0.000000}%
\pgfsetstrokecolor{currentstroke}%
\pgfsetstrokeopacity{0.000000}%
\pgfsetdash{}{0pt}%
\pgfpathmoveto{\pgfqpoint{0.750000in}{0.385000in}}%
\pgfpathlineto{\pgfqpoint{5.400000in}{0.385000in}}%
\pgfpathlineto{\pgfqpoint{5.400000in}{3.080000in}}%
\pgfpathlineto{\pgfqpoint{0.750000in}{3.080000in}}%
\pgfpathlineto{\pgfqpoint{0.750000in}{0.385000in}}%
\pgfpathclose%
\pgfusepath{fill}%
\end{pgfscope}%
\begin{pgfscope}%
\pgfpathrectangle{\pgfqpoint{0.750000in}{0.385000in}}{\pgfqpoint{4.650000in}{2.695000in}}%
\pgfusepath{clip}%
\pgfsetbuttcap%
\pgfsetmiterjoin%
\definecolor{currentfill}{rgb}{0.000000,0.500000,0.000000}%
\pgfsetfillcolor{currentfill}%
\pgfsetfillopacity{0.600000}%
\pgfsetlinewidth{0.000000pt}%
\definecolor{currentstroke}{rgb}{0.000000,0.000000,0.000000}%
\pgfsetstrokecolor{currentstroke}%
\pgfsetstrokeopacity{0.600000}%
\pgfsetdash{}{0pt}%
\pgfpathmoveto{\pgfqpoint{2.882851in}{0.385000in}}%
\pgfpathlineto{\pgfqpoint{2.894380in}{0.385000in}}%
\pgfpathlineto{\pgfqpoint{2.894380in}{0.396938in}}%
\pgfpathlineto{\pgfqpoint{2.882851in}{0.396938in}}%
\pgfpathlineto{\pgfqpoint{2.882851in}{0.385000in}}%
\pgfpathclose%
\pgfusepath{fill}%
\end{pgfscope}%
\begin{pgfscope}%
\pgfpathrectangle{\pgfqpoint{0.750000in}{0.385000in}}{\pgfqpoint{4.650000in}{2.695000in}}%
\pgfusepath{clip}%
\pgfsetbuttcap%
\pgfsetmiterjoin%
\definecolor{currentfill}{rgb}{0.000000,0.500000,0.000000}%
\pgfsetfillcolor{currentfill}%
\pgfsetfillopacity{0.600000}%
\pgfsetlinewidth{0.000000pt}%
\definecolor{currentstroke}{rgb}{0.000000,0.000000,0.000000}%
\pgfsetstrokecolor{currentstroke}%
\pgfsetstrokeopacity{0.600000}%
\pgfsetdash{}{0pt}%
\pgfpathmoveto{\pgfqpoint{2.894380in}{0.385000in}}%
\pgfpathlineto{\pgfqpoint{2.905909in}{0.385000in}}%
\pgfpathlineto{\pgfqpoint{2.905909in}{0.540194in}}%
\pgfpathlineto{\pgfqpoint{2.894380in}{0.540194in}}%
\pgfpathlineto{\pgfqpoint{2.894380in}{0.385000in}}%
\pgfpathclose%
\pgfusepath{fill}%
\end{pgfscope}%
\begin{pgfscope}%
\pgfpathrectangle{\pgfqpoint{0.750000in}{0.385000in}}{\pgfqpoint{4.650000in}{2.695000in}}%
\pgfusepath{clip}%
\pgfsetbuttcap%
\pgfsetmiterjoin%
\definecolor{currentfill}{rgb}{0.000000,0.500000,0.000000}%
\pgfsetfillcolor{currentfill}%
\pgfsetfillopacity{0.600000}%
\pgfsetlinewidth{0.000000pt}%
\definecolor{currentstroke}{rgb}{0.000000,0.000000,0.000000}%
\pgfsetstrokecolor{currentstroke}%
\pgfsetstrokeopacity{0.600000}%
\pgfsetdash{}{0pt}%
\pgfpathmoveto{\pgfqpoint{2.905909in}{0.385000in}}%
\pgfpathlineto{\pgfqpoint{2.917438in}{0.385000in}}%
\pgfpathlineto{\pgfqpoint{2.917438in}{1.125155in}}%
\pgfpathlineto{\pgfqpoint{2.905909in}{1.125155in}}%
\pgfpathlineto{\pgfqpoint{2.905909in}{0.385000in}}%
\pgfpathclose%
\pgfusepath{fill}%
\end{pgfscope}%
\begin{pgfscope}%
\pgfpathrectangle{\pgfqpoint{0.750000in}{0.385000in}}{\pgfqpoint{4.650000in}{2.695000in}}%
\pgfusepath{clip}%
\pgfsetbuttcap%
\pgfsetmiterjoin%
\definecolor{currentfill}{rgb}{0.000000,0.500000,0.000000}%
\pgfsetfillcolor{currentfill}%
\pgfsetfillopacity{0.600000}%
\pgfsetlinewidth{0.000000pt}%
\definecolor{currentstroke}{rgb}{0.000000,0.000000,0.000000}%
\pgfsetstrokecolor{currentstroke}%
\pgfsetstrokeopacity{0.600000}%
\pgfsetdash{}{0pt}%
\pgfpathmoveto{\pgfqpoint{2.917438in}{0.385000in}}%
\pgfpathlineto{\pgfqpoint{2.928967in}{0.385000in}}%
\pgfpathlineto{\pgfqpoint{2.928967in}{1.566860in}}%
\pgfpathlineto{\pgfqpoint{2.917438in}{1.566860in}}%
\pgfpathlineto{\pgfqpoint{2.917438in}{0.385000in}}%
\pgfpathclose%
\pgfusepath{fill}%
\end{pgfscope}%
\begin{pgfscope}%
\pgfpathrectangle{\pgfqpoint{0.750000in}{0.385000in}}{\pgfqpoint{4.650000in}{2.695000in}}%
\pgfusepath{clip}%
\pgfsetbuttcap%
\pgfsetmiterjoin%
\definecolor{currentfill}{rgb}{0.000000,0.500000,0.000000}%
\pgfsetfillcolor{currentfill}%
\pgfsetfillopacity{0.600000}%
\pgfsetlinewidth{0.000000pt}%
\definecolor{currentstroke}{rgb}{0.000000,0.000000,0.000000}%
\pgfsetstrokecolor{currentstroke}%
\pgfsetstrokeopacity{0.600000}%
\pgfsetdash{}{0pt}%
\pgfpathmoveto{\pgfqpoint{2.928967in}{0.385000in}}%
\pgfpathlineto{\pgfqpoint{2.940496in}{0.385000in}}%
\pgfpathlineto{\pgfqpoint{2.940496in}{1.686240in}}%
\pgfpathlineto{\pgfqpoint{2.928967in}{1.686240in}}%
\pgfpathlineto{\pgfqpoint{2.928967in}{0.385000in}}%
\pgfpathclose%
\pgfusepath{fill}%
\end{pgfscope}%
\begin{pgfscope}%
\pgfpathrectangle{\pgfqpoint{0.750000in}{0.385000in}}{\pgfqpoint{4.650000in}{2.695000in}}%
\pgfusepath{clip}%
\pgfsetbuttcap%
\pgfsetmiterjoin%
\definecolor{currentfill}{rgb}{0.000000,0.500000,0.000000}%
\pgfsetfillcolor{currentfill}%
\pgfsetfillopacity{0.600000}%
\pgfsetlinewidth{0.000000pt}%
\definecolor{currentstroke}{rgb}{0.000000,0.000000,0.000000}%
\pgfsetstrokecolor{currentstroke}%
\pgfsetstrokeopacity{0.600000}%
\pgfsetdash{}{0pt}%
\pgfpathmoveto{\pgfqpoint{2.940496in}{0.385000in}}%
\pgfpathlineto{\pgfqpoint{2.952025in}{0.385000in}}%
\pgfpathlineto{\pgfqpoint{2.952025in}{2.116008in}}%
\pgfpathlineto{\pgfqpoint{2.940496in}{2.116008in}}%
\pgfpathlineto{\pgfqpoint{2.940496in}{0.385000in}}%
\pgfpathclose%
\pgfusepath{fill}%
\end{pgfscope}%
\begin{pgfscope}%
\pgfpathrectangle{\pgfqpoint{0.750000in}{0.385000in}}{\pgfqpoint{4.650000in}{2.695000in}}%
\pgfusepath{clip}%
\pgfsetbuttcap%
\pgfsetmiterjoin%
\definecolor{currentfill}{rgb}{0.000000,0.500000,0.000000}%
\pgfsetfillcolor{currentfill}%
\pgfsetfillopacity{0.600000}%
\pgfsetlinewidth{0.000000pt}%
\definecolor{currentstroke}{rgb}{0.000000,0.000000,0.000000}%
\pgfsetstrokecolor{currentstroke}%
\pgfsetstrokeopacity{0.600000}%
\pgfsetdash{}{0pt}%
\pgfpathmoveto{\pgfqpoint{2.952025in}{0.385000in}}%
\pgfpathlineto{\pgfqpoint{2.963554in}{0.385000in}}%
\pgfpathlineto{\pgfqpoint{2.963554in}{2.533837in}}%
\pgfpathlineto{\pgfqpoint{2.952025in}{2.533837in}}%
\pgfpathlineto{\pgfqpoint{2.952025in}{0.385000in}}%
\pgfpathclose%
\pgfusepath{fill}%
\end{pgfscope}%
\begin{pgfscope}%
\pgfpathrectangle{\pgfqpoint{0.750000in}{0.385000in}}{\pgfqpoint{4.650000in}{2.695000in}}%
\pgfusepath{clip}%
\pgfsetbuttcap%
\pgfsetmiterjoin%
\definecolor{currentfill}{rgb}{0.000000,0.500000,0.000000}%
\pgfsetfillcolor{currentfill}%
\pgfsetfillopacity{0.600000}%
\pgfsetlinewidth{0.000000pt}%
\definecolor{currentstroke}{rgb}{0.000000,0.000000,0.000000}%
\pgfsetstrokecolor{currentstroke}%
\pgfsetstrokeopacity{0.600000}%
\pgfsetdash{}{0pt}%
\pgfpathmoveto{\pgfqpoint{2.963554in}{0.385000in}}%
\pgfpathlineto{\pgfqpoint{2.975083in}{0.385000in}}%
\pgfpathlineto{\pgfqpoint{2.975083in}{2.414457in}}%
\pgfpathlineto{\pgfqpoint{2.963554in}{2.414457in}}%
\pgfpathlineto{\pgfqpoint{2.963554in}{0.385000in}}%
\pgfpathclose%
\pgfusepath{fill}%
\end{pgfscope}%
\begin{pgfscope}%
\pgfpathrectangle{\pgfqpoint{0.750000in}{0.385000in}}{\pgfqpoint{4.650000in}{2.695000in}}%
\pgfusepath{clip}%
\pgfsetbuttcap%
\pgfsetmiterjoin%
\definecolor{currentfill}{rgb}{0.000000,0.500000,0.000000}%
\pgfsetfillcolor{currentfill}%
\pgfsetfillopacity{0.600000}%
\pgfsetlinewidth{0.000000pt}%
\definecolor{currentstroke}{rgb}{0.000000,0.000000,0.000000}%
\pgfsetstrokecolor{currentstroke}%
\pgfsetstrokeopacity{0.600000}%
\pgfsetdash{}{0pt}%
\pgfpathmoveto{\pgfqpoint{2.975083in}{0.385000in}}%
\pgfpathlineto{\pgfqpoint{2.986612in}{0.385000in}}%
\pgfpathlineto{\pgfqpoint{2.986612in}{2.414457in}}%
\pgfpathlineto{\pgfqpoint{2.975083in}{2.414457in}}%
\pgfpathlineto{\pgfqpoint{2.975083in}{0.385000in}}%
\pgfpathclose%
\pgfusepath{fill}%
\end{pgfscope}%
\begin{pgfscope}%
\pgfpathrectangle{\pgfqpoint{0.750000in}{0.385000in}}{\pgfqpoint{4.650000in}{2.695000in}}%
\pgfusepath{clip}%
\pgfsetbuttcap%
\pgfsetmiterjoin%
\definecolor{currentfill}{rgb}{0.000000,0.500000,0.000000}%
\pgfsetfillcolor{currentfill}%
\pgfsetfillopacity{0.600000}%
\pgfsetlinewidth{0.000000pt}%
\definecolor{currentstroke}{rgb}{0.000000,0.000000,0.000000}%
\pgfsetstrokecolor{currentstroke}%
\pgfsetstrokeopacity{0.600000}%
\pgfsetdash{}{0pt}%
\pgfpathmoveto{\pgfqpoint{2.986612in}{0.385000in}}%
\pgfpathlineto{\pgfqpoint{2.998140in}{0.385000in}}%
\pgfpathlineto{\pgfqpoint{2.998140in}{2.736783in}}%
\pgfpathlineto{\pgfqpoint{2.986612in}{2.736783in}}%
\pgfpathlineto{\pgfqpoint{2.986612in}{0.385000in}}%
\pgfpathclose%
\pgfusepath{fill}%
\end{pgfscope}%
\begin{pgfscope}%
\pgfpathrectangle{\pgfqpoint{0.750000in}{0.385000in}}{\pgfqpoint{4.650000in}{2.695000in}}%
\pgfusepath{clip}%
\pgfsetbuttcap%
\pgfsetmiterjoin%
\definecolor{currentfill}{rgb}{0.000000,0.500000,0.000000}%
\pgfsetfillcolor{currentfill}%
\pgfsetfillopacity{0.600000}%
\pgfsetlinewidth{0.000000pt}%
\definecolor{currentstroke}{rgb}{0.000000,0.000000,0.000000}%
\pgfsetstrokecolor{currentstroke}%
\pgfsetstrokeopacity{0.600000}%
\pgfsetdash{}{0pt}%
\pgfpathmoveto{\pgfqpoint{2.998140in}{0.385000in}}%
\pgfpathlineto{\pgfqpoint{3.009669in}{0.385000in}}%
\pgfpathlineto{\pgfqpoint{3.009669in}{2.951667in}}%
\pgfpathlineto{\pgfqpoint{2.998140in}{2.951667in}}%
\pgfpathlineto{\pgfqpoint{2.998140in}{0.385000in}}%
\pgfpathclose%
\pgfusepath{fill}%
\end{pgfscope}%
\begin{pgfscope}%
\pgfpathrectangle{\pgfqpoint{0.750000in}{0.385000in}}{\pgfqpoint{4.650000in}{2.695000in}}%
\pgfusepath{clip}%
\pgfsetbuttcap%
\pgfsetmiterjoin%
\definecolor{currentfill}{rgb}{0.000000,0.500000,0.000000}%
\pgfsetfillcolor{currentfill}%
\pgfsetfillopacity{0.600000}%
\pgfsetlinewidth{0.000000pt}%
\definecolor{currentstroke}{rgb}{0.000000,0.000000,0.000000}%
\pgfsetstrokecolor{currentstroke}%
\pgfsetstrokeopacity{0.600000}%
\pgfsetdash{}{0pt}%
\pgfpathmoveto{\pgfqpoint{3.009669in}{0.385000in}}%
\pgfpathlineto{\pgfqpoint{3.021198in}{0.385000in}}%
\pgfpathlineto{\pgfqpoint{3.021198in}{2.641279in}}%
\pgfpathlineto{\pgfqpoint{3.009669in}{2.641279in}}%
\pgfpathlineto{\pgfqpoint{3.009669in}{0.385000in}}%
\pgfpathclose%
\pgfusepath{fill}%
\end{pgfscope}%
\begin{pgfscope}%
\pgfpathrectangle{\pgfqpoint{0.750000in}{0.385000in}}{\pgfqpoint{4.650000in}{2.695000in}}%
\pgfusepath{clip}%
\pgfsetbuttcap%
\pgfsetmiterjoin%
\definecolor{currentfill}{rgb}{0.000000,0.500000,0.000000}%
\pgfsetfillcolor{currentfill}%
\pgfsetfillopacity{0.600000}%
\pgfsetlinewidth{0.000000pt}%
\definecolor{currentstroke}{rgb}{0.000000,0.000000,0.000000}%
\pgfsetstrokecolor{currentstroke}%
\pgfsetstrokeopacity{0.600000}%
\pgfsetdash{}{0pt}%
\pgfpathmoveto{\pgfqpoint{3.021198in}{0.385000in}}%
\pgfpathlineto{\pgfqpoint{3.032727in}{0.385000in}}%
\pgfpathlineto{\pgfqpoint{3.032727in}{2.772597in}}%
\pgfpathlineto{\pgfqpoint{3.021198in}{2.772597in}}%
\pgfpathlineto{\pgfqpoint{3.021198in}{0.385000in}}%
\pgfpathclose%
\pgfusepath{fill}%
\end{pgfscope}%
\begin{pgfscope}%
\pgfpathrectangle{\pgfqpoint{0.750000in}{0.385000in}}{\pgfqpoint{4.650000in}{2.695000in}}%
\pgfusepath{clip}%
\pgfsetbuttcap%
\pgfsetmiterjoin%
\definecolor{currentfill}{rgb}{0.000000,0.500000,0.000000}%
\pgfsetfillcolor{currentfill}%
\pgfsetfillopacity{0.600000}%
\pgfsetlinewidth{0.000000pt}%
\definecolor{currentstroke}{rgb}{0.000000,0.000000,0.000000}%
\pgfsetstrokecolor{currentstroke}%
\pgfsetstrokeopacity{0.600000}%
\pgfsetdash{}{0pt}%
\pgfpathmoveto{\pgfqpoint{3.032727in}{0.385000in}}%
\pgfpathlineto{\pgfqpoint{3.044256in}{0.385000in}}%
\pgfpathlineto{\pgfqpoint{3.044256in}{2.784535in}}%
\pgfpathlineto{\pgfqpoint{3.032727in}{2.784535in}}%
\pgfpathlineto{\pgfqpoint{3.032727in}{0.385000in}}%
\pgfpathclose%
\pgfusepath{fill}%
\end{pgfscope}%
\begin{pgfscope}%
\pgfpathrectangle{\pgfqpoint{0.750000in}{0.385000in}}{\pgfqpoint{4.650000in}{2.695000in}}%
\pgfusepath{clip}%
\pgfsetbuttcap%
\pgfsetmiterjoin%
\definecolor{currentfill}{rgb}{0.000000,0.500000,0.000000}%
\pgfsetfillcolor{currentfill}%
\pgfsetfillopacity{0.600000}%
\pgfsetlinewidth{0.000000pt}%
\definecolor{currentstroke}{rgb}{0.000000,0.000000,0.000000}%
\pgfsetstrokecolor{currentstroke}%
\pgfsetstrokeopacity{0.600000}%
\pgfsetdash{}{0pt}%
\pgfpathmoveto{\pgfqpoint{3.044256in}{0.385000in}}%
\pgfpathlineto{\pgfqpoint{3.055785in}{0.385000in}}%
\pgfpathlineto{\pgfqpoint{3.055785in}{2.509961in}}%
\pgfpathlineto{\pgfqpoint{3.044256in}{2.509961in}}%
\pgfpathlineto{\pgfqpoint{3.044256in}{0.385000in}}%
\pgfpathclose%
\pgfusepath{fill}%
\end{pgfscope}%
\begin{pgfscope}%
\pgfpathrectangle{\pgfqpoint{0.750000in}{0.385000in}}{\pgfqpoint{4.650000in}{2.695000in}}%
\pgfusepath{clip}%
\pgfsetbuttcap%
\pgfsetmiterjoin%
\definecolor{currentfill}{rgb}{0.000000,0.500000,0.000000}%
\pgfsetfillcolor{currentfill}%
\pgfsetfillopacity{0.600000}%
\pgfsetlinewidth{0.000000pt}%
\definecolor{currentstroke}{rgb}{0.000000,0.000000,0.000000}%
\pgfsetstrokecolor{currentstroke}%
\pgfsetstrokeopacity{0.600000}%
\pgfsetdash{}{0pt}%
\pgfpathmoveto{\pgfqpoint{3.055785in}{0.385000in}}%
\pgfpathlineto{\pgfqpoint{3.067314in}{0.385000in}}%
\pgfpathlineto{\pgfqpoint{3.067314in}{2.617403in}}%
\pgfpathlineto{\pgfqpoint{3.055785in}{2.617403in}}%
\pgfpathlineto{\pgfqpoint{3.055785in}{0.385000in}}%
\pgfpathclose%
\pgfusepath{fill}%
\end{pgfscope}%
\begin{pgfscope}%
\pgfpathrectangle{\pgfqpoint{0.750000in}{0.385000in}}{\pgfqpoint{4.650000in}{2.695000in}}%
\pgfusepath{clip}%
\pgfsetbuttcap%
\pgfsetmiterjoin%
\definecolor{currentfill}{rgb}{0.000000,0.500000,0.000000}%
\pgfsetfillcolor{currentfill}%
\pgfsetfillopacity{0.600000}%
\pgfsetlinewidth{0.000000pt}%
\definecolor{currentstroke}{rgb}{0.000000,0.000000,0.000000}%
\pgfsetstrokecolor{currentstroke}%
\pgfsetstrokeopacity{0.600000}%
\pgfsetdash{}{0pt}%
\pgfpathmoveto{\pgfqpoint{3.067314in}{0.385000in}}%
\pgfpathlineto{\pgfqpoint{3.078843in}{0.385000in}}%
\pgfpathlineto{\pgfqpoint{3.078843in}{2.760659in}}%
\pgfpathlineto{\pgfqpoint{3.067314in}{2.760659in}}%
\pgfpathlineto{\pgfqpoint{3.067314in}{0.385000in}}%
\pgfpathclose%
\pgfusepath{fill}%
\end{pgfscope}%
\begin{pgfscope}%
\pgfpathrectangle{\pgfqpoint{0.750000in}{0.385000in}}{\pgfqpoint{4.650000in}{2.695000in}}%
\pgfusepath{clip}%
\pgfsetbuttcap%
\pgfsetmiterjoin%
\definecolor{currentfill}{rgb}{0.000000,0.500000,0.000000}%
\pgfsetfillcolor{currentfill}%
\pgfsetfillopacity{0.600000}%
\pgfsetlinewidth{0.000000pt}%
\definecolor{currentstroke}{rgb}{0.000000,0.000000,0.000000}%
\pgfsetstrokecolor{currentstroke}%
\pgfsetstrokeopacity{0.600000}%
\pgfsetdash{}{0pt}%
\pgfpathmoveto{\pgfqpoint{3.078843in}{0.385000in}}%
\pgfpathlineto{\pgfqpoint{3.090372in}{0.385000in}}%
\pgfpathlineto{\pgfqpoint{3.090372in}{2.665155in}}%
\pgfpathlineto{\pgfqpoint{3.078843in}{2.665155in}}%
\pgfpathlineto{\pgfqpoint{3.078843in}{0.385000in}}%
\pgfpathclose%
\pgfusepath{fill}%
\end{pgfscope}%
\begin{pgfscope}%
\pgfpathrectangle{\pgfqpoint{0.750000in}{0.385000in}}{\pgfqpoint{4.650000in}{2.695000in}}%
\pgfusepath{clip}%
\pgfsetbuttcap%
\pgfsetmiterjoin%
\definecolor{currentfill}{rgb}{0.000000,0.500000,0.000000}%
\pgfsetfillcolor{currentfill}%
\pgfsetfillopacity{0.600000}%
\pgfsetlinewidth{0.000000pt}%
\definecolor{currentstroke}{rgb}{0.000000,0.000000,0.000000}%
\pgfsetstrokecolor{currentstroke}%
\pgfsetstrokeopacity{0.600000}%
\pgfsetdash{}{0pt}%
\pgfpathmoveto{\pgfqpoint{3.090372in}{0.385000in}}%
\pgfpathlineto{\pgfqpoint{3.101901in}{0.385000in}}%
\pgfpathlineto{\pgfqpoint{3.101901in}{2.689031in}}%
\pgfpathlineto{\pgfqpoint{3.090372in}{2.689031in}}%
\pgfpathlineto{\pgfqpoint{3.090372in}{0.385000in}}%
\pgfpathclose%
\pgfusepath{fill}%
\end{pgfscope}%
\begin{pgfscope}%
\pgfpathrectangle{\pgfqpoint{0.750000in}{0.385000in}}{\pgfqpoint{4.650000in}{2.695000in}}%
\pgfusepath{clip}%
\pgfsetbuttcap%
\pgfsetmiterjoin%
\definecolor{currentfill}{rgb}{0.000000,0.500000,0.000000}%
\pgfsetfillcolor{currentfill}%
\pgfsetfillopacity{0.600000}%
\pgfsetlinewidth{0.000000pt}%
\definecolor{currentstroke}{rgb}{0.000000,0.000000,0.000000}%
\pgfsetstrokecolor{currentstroke}%
\pgfsetstrokeopacity{0.600000}%
\pgfsetdash{}{0pt}%
\pgfpathmoveto{\pgfqpoint{3.101901in}{0.385000in}}%
\pgfpathlineto{\pgfqpoint{3.113430in}{0.385000in}}%
\pgfpathlineto{\pgfqpoint{3.113430in}{2.402519in}}%
\pgfpathlineto{\pgfqpoint{3.101901in}{2.402519in}}%
\pgfpathlineto{\pgfqpoint{3.101901in}{0.385000in}}%
\pgfpathclose%
\pgfusepath{fill}%
\end{pgfscope}%
\begin{pgfscope}%
\pgfpathrectangle{\pgfqpoint{0.750000in}{0.385000in}}{\pgfqpoint{4.650000in}{2.695000in}}%
\pgfusepath{clip}%
\pgfsetbuttcap%
\pgfsetmiterjoin%
\definecolor{currentfill}{rgb}{0.000000,0.500000,0.000000}%
\pgfsetfillcolor{currentfill}%
\pgfsetfillopacity{0.600000}%
\pgfsetlinewidth{0.000000pt}%
\definecolor{currentstroke}{rgb}{0.000000,0.000000,0.000000}%
\pgfsetstrokecolor{currentstroke}%
\pgfsetstrokeopacity{0.600000}%
\pgfsetdash{}{0pt}%
\pgfpathmoveto{\pgfqpoint{3.113430in}{0.385000in}}%
\pgfpathlineto{\pgfqpoint{3.124959in}{0.385000in}}%
\pgfpathlineto{\pgfqpoint{3.124959in}{2.366705in}}%
\pgfpathlineto{\pgfqpoint{3.113430in}{2.366705in}}%
\pgfpathlineto{\pgfqpoint{3.113430in}{0.385000in}}%
\pgfpathclose%
\pgfusepath{fill}%
\end{pgfscope}%
\begin{pgfscope}%
\pgfpathrectangle{\pgfqpoint{0.750000in}{0.385000in}}{\pgfqpoint{4.650000in}{2.695000in}}%
\pgfusepath{clip}%
\pgfsetbuttcap%
\pgfsetmiterjoin%
\definecolor{currentfill}{rgb}{0.000000,0.500000,0.000000}%
\pgfsetfillcolor{currentfill}%
\pgfsetfillopacity{0.600000}%
\pgfsetlinewidth{0.000000pt}%
\definecolor{currentstroke}{rgb}{0.000000,0.000000,0.000000}%
\pgfsetstrokecolor{currentstroke}%
\pgfsetstrokeopacity{0.600000}%
\pgfsetdash{}{0pt}%
\pgfpathmoveto{\pgfqpoint{3.124959in}{0.385000in}}%
\pgfpathlineto{\pgfqpoint{3.136488in}{0.385000in}}%
\pgfpathlineto{\pgfqpoint{3.136488in}{2.426395in}}%
\pgfpathlineto{\pgfqpoint{3.124959in}{2.426395in}}%
\pgfpathlineto{\pgfqpoint{3.124959in}{0.385000in}}%
\pgfpathclose%
\pgfusepath{fill}%
\end{pgfscope}%
\begin{pgfscope}%
\pgfpathrectangle{\pgfqpoint{0.750000in}{0.385000in}}{\pgfqpoint{4.650000in}{2.695000in}}%
\pgfusepath{clip}%
\pgfsetbuttcap%
\pgfsetmiterjoin%
\definecolor{currentfill}{rgb}{0.000000,0.500000,0.000000}%
\pgfsetfillcolor{currentfill}%
\pgfsetfillopacity{0.600000}%
\pgfsetlinewidth{0.000000pt}%
\definecolor{currentstroke}{rgb}{0.000000,0.000000,0.000000}%
\pgfsetstrokecolor{currentstroke}%
\pgfsetstrokeopacity{0.600000}%
\pgfsetdash{}{0pt}%
\pgfpathmoveto{\pgfqpoint{3.136488in}{0.385000in}}%
\pgfpathlineto{\pgfqpoint{3.148017in}{0.385000in}}%
\pgfpathlineto{\pgfqpoint{3.148017in}{2.092132in}}%
\pgfpathlineto{\pgfqpoint{3.136488in}{2.092132in}}%
\pgfpathlineto{\pgfqpoint{3.136488in}{0.385000in}}%
\pgfpathclose%
\pgfusepath{fill}%
\end{pgfscope}%
\begin{pgfscope}%
\pgfpathrectangle{\pgfqpoint{0.750000in}{0.385000in}}{\pgfqpoint{4.650000in}{2.695000in}}%
\pgfusepath{clip}%
\pgfsetbuttcap%
\pgfsetmiterjoin%
\definecolor{currentfill}{rgb}{0.000000,0.500000,0.000000}%
\pgfsetfillcolor{currentfill}%
\pgfsetfillopacity{0.600000}%
\pgfsetlinewidth{0.000000pt}%
\definecolor{currentstroke}{rgb}{0.000000,0.000000,0.000000}%
\pgfsetstrokecolor{currentstroke}%
\pgfsetstrokeopacity{0.600000}%
\pgfsetdash{}{0pt}%
\pgfpathmoveto{\pgfqpoint{3.148017in}{0.385000in}}%
\pgfpathlineto{\pgfqpoint{3.159545in}{0.385000in}}%
\pgfpathlineto{\pgfqpoint{3.159545in}{2.259264in}}%
\pgfpathlineto{\pgfqpoint{3.148017in}{2.259264in}}%
\pgfpathlineto{\pgfqpoint{3.148017in}{0.385000in}}%
\pgfpathclose%
\pgfusepath{fill}%
\end{pgfscope}%
\begin{pgfscope}%
\pgfpathrectangle{\pgfqpoint{0.750000in}{0.385000in}}{\pgfqpoint{4.650000in}{2.695000in}}%
\pgfusepath{clip}%
\pgfsetbuttcap%
\pgfsetmiterjoin%
\definecolor{currentfill}{rgb}{0.000000,0.500000,0.000000}%
\pgfsetfillcolor{currentfill}%
\pgfsetfillopacity{0.600000}%
\pgfsetlinewidth{0.000000pt}%
\definecolor{currentstroke}{rgb}{0.000000,0.000000,0.000000}%
\pgfsetstrokecolor{currentstroke}%
\pgfsetstrokeopacity{0.600000}%
\pgfsetdash{}{0pt}%
\pgfpathmoveto{\pgfqpoint{3.159545in}{0.385000in}}%
\pgfpathlineto{\pgfqpoint{3.171074in}{0.385000in}}%
\pgfpathlineto{\pgfqpoint{3.171074in}{2.545775in}}%
\pgfpathlineto{\pgfqpoint{3.159545in}{2.545775in}}%
\pgfpathlineto{\pgfqpoint{3.159545in}{0.385000in}}%
\pgfpathclose%
\pgfusepath{fill}%
\end{pgfscope}%
\begin{pgfscope}%
\pgfpathrectangle{\pgfqpoint{0.750000in}{0.385000in}}{\pgfqpoint{4.650000in}{2.695000in}}%
\pgfusepath{clip}%
\pgfsetbuttcap%
\pgfsetmiterjoin%
\definecolor{currentfill}{rgb}{0.000000,0.500000,0.000000}%
\pgfsetfillcolor{currentfill}%
\pgfsetfillopacity{0.600000}%
\pgfsetlinewidth{0.000000pt}%
\definecolor{currentstroke}{rgb}{0.000000,0.000000,0.000000}%
\pgfsetstrokecolor{currentstroke}%
\pgfsetstrokeopacity{0.600000}%
\pgfsetdash{}{0pt}%
\pgfpathmoveto{\pgfqpoint{3.171074in}{0.385000in}}%
\pgfpathlineto{\pgfqpoint{3.182603in}{0.385000in}}%
\pgfpathlineto{\pgfqpoint{3.182603in}{2.175698in}}%
\pgfpathlineto{\pgfqpoint{3.171074in}{2.175698in}}%
\pgfpathlineto{\pgfqpoint{3.171074in}{0.385000in}}%
\pgfpathclose%
\pgfusepath{fill}%
\end{pgfscope}%
\begin{pgfscope}%
\pgfpathrectangle{\pgfqpoint{0.750000in}{0.385000in}}{\pgfqpoint{4.650000in}{2.695000in}}%
\pgfusepath{clip}%
\pgfsetbuttcap%
\pgfsetmiterjoin%
\definecolor{currentfill}{rgb}{0.000000,0.500000,0.000000}%
\pgfsetfillcolor{currentfill}%
\pgfsetfillopacity{0.600000}%
\pgfsetlinewidth{0.000000pt}%
\definecolor{currentstroke}{rgb}{0.000000,0.000000,0.000000}%
\pgfsetstrokecolor{currentstroke}%
\pgfsetstrokeopacity{0.600000}%
\pgfsetdash{}{0pt}%
\pgfpathmoveto{\pgfqpoint{3.182603in}{0.385000in}}%
\pgfpathlineto{\pgfqpoint{3.194132in}{0.385000in}}%
\pgfpathlineto{\pgfqpoint{3.194132in}{2.092132in}}%
\pgfpathlineto{\pgfqpoint{3.182603in}{2.092132in}}%
\pgfpathlineto{\pgfqpoint{3.182603in}{0.385000in}}%
\pgfpathclose%
\pgfusepath{fill}%
\end{pgfscope}%
\begin{pgfscope}%
\pgfpathrectangle{\pgfqpoint{0.750000in}{0.385000in}}{\pgfqpoint{4.650000in}{2.695000in}}%
\pgfusepath{clip}%
\pgfsetbuttcap%
\pgfsetmiterjoin%
\definecolor{currentfill}{rgb}{0.000000,0.500000,0.000000}%
\pgfsetfillcolor{currentfill}%
\pgfsetfillopacity{0.600000}%
\pgfsetlinewidth{0.000000pt}%
\definecolor{currentstroke}{rgb}{0.000000,0.000000,0.000000}%
\pgfsetstrokecolor{currentstroke}%
\pgfsetstrokeopacity{0.600000}%
\pgfsetdash{}{0pt}%
\pgfpathmoveto{\pgfqpoint{3.194132in}{0.385000in}}%
\pgfpathlineto{\pgfqpoint{3.205661in}{0.385000in}}%
\pgfpathlineto{\pgfqpoint{3.205661in}{1.829496in}}%
\pgfpathlineto{\pgfqpoint{3.194132in}{1.829496in}}%
\pgfpathlineto{\pgfqpoint{3.194132in}{0.385000in}}%
\pgfpathclose%
\pgfusepath{fill}%
\end{pgfscope}%
\begin{pgfscope}%
\pgfpathrectangle{\pgfqpoint{0.750000in}{0.385000in}}{\pgfqpoint{4.650000in}{2.695000in}}%
\pgfusepath{clip}%
\pgfsetbuttcap%
\pgfsetmiterjoin%
\definecolor{currentfill}{rgb}{0.000000,0.500000,0.000000}%
\pgfsetfillcolor{currentfill}%
\pgfsetfillopacity{0.600000}%
\pgfsetlinewidth{0.000000pt}%
\definecolor{currentstroke}{rgb}{0.000000,0.000000,0.000000}%
\pgfsetstrokecolor{currentstroke}%
\pgfsetstrokeopacity{0.600000}%
\pgfsetdash{}{0pt}%
\pgfpathmoveto{\pgfqpoint{3.205661in}{0.385000in}}%
\pgfpathlineto{\pgfqpoint{3.217190in}{0.385000in}}%
\pgfpathlineto{\pgfqpoint{3.217190in}{2.068256in}}%
\pgfpathlineto{\pgfqpoint{3.205661in}{2.068256in}}%
\pgfpathlineto{\pgfqpoint{3.205661in}{0.385000in}}%
\pgfpathclose%
\pgfusepath{fill}%
\end{pgfscope}%
\begin{pgfscope}%
\pgfpathrectangle{\pgfqpoint{0.750000in}{0.385000in}}{\pgfqpoint{4.650000in}{2.695000in}}%
\pgfusepath{clip}%
\pgfsetbuttcap%
\pgfsetmiterjoin%
\definecolor{currentfill}{rgb}{0.000000,0.500000,0.000000}%
\pgfsetfillcolor{currentfill}%
\pgfsetfillopacity{0.600000}%
\pgfsetlinewidth{0.000000pt}%
\definecolor{currentstroke}{rgb}{0.000000,0.000000,0.000000}%
\pgfsetstrokecolor{currentstroke}%
\pgfsetstrokeopacity{0.600000}%
\pgfsetdash{}{0pt}%
\pgfpathmoveto{\pgfqpoint{3.217190in}{0.385000in}}%
\pgfpathlineto{\pgfqpoint{3.228719in}{0.385000in}}%
\pgfpathlineto{\pgfqpoint{3.228719in}{2.116008in}}%
\pgfpathlineto{\pgfqpoint{3.217190in}{2.116008in}}%
\pgfpathlineto{\pgfqpoint{3.217190in}{0.385000in}}%
\pgfpathclose%
\pgfusepath{fill}%
\end{pgfscope}%
\begin{pgfscope}%
\pgfpathrectangle{\pgfqpoint{0.750000in}{0.385000in}}{\pgfqpoint{4.650000in}{2.695000in}}%
\pgfusepath{clip}%
\pgfsetbuttcap%
\pgfsetmiterjoin%
\definecolor{currentfill}{rgb}{0.000000,0.500000,0.000000}%
\pgfsetfillcolor{currentfill}%
\pgfsetfillopacity{0.600000}%
\pgfsetlinewidth{0.000000pt}%
\definecolor{currentstroke}{rgb}{0.000000,0.000000,0.000000}%
\pgfsetstrokecolor{currentstroke}%
\pgfsetstrokeopacity{0.600000}%
\pgfsetdash{}{0pt}%
\pgfpathmoveto{\pgfqpoint{3.228719in}{0.385000in}}%
\pgfpathlineto{\pgfqpoint{3.240248in}{0.385000in}}%
\pgfpathlineto{\pgfqpoint{3.240248in}{1.841434in}}%
\pgfpathlineto{\pgfqpoint{3.228719in}{1.841434in}}%
\pgfpathlineto{\pgfqpoint{3.228719in}{0.385000in}}%
\pgfpathclose%
\pgfusepath{fill}%
\end{pgfscope}%
\begin{pgfscope}%
\pgfpathrectangle{\pgfqpoint{0.750000in}{0.385000in}}{\pgfqpoint{4.650000in}{2.695000in}}%
\pgfusepath{clip}%
\pgfsetbuttcap%
\pgfsetmiterjoin%
\definecolor{currentfill}{rgb}{0.000000,0.500000,0.000000}%
\pgfsetfillcolor{currentfill}%
\pgfsetfillopacity{0.600000}%
\pgfsetlinewidth{0.000000pt}%
\definecolor{currentstroke}{rgb}{0.000000,0.000000,0.000000}%
\pgfsetstrokecolor{currentstroke}%
\pgfsetstrokeopacity{0.600000}%
\pgfsetdash{}{0pt}%
\pgfpathmoveto{\pgfqpoint{3.240248in}{0.385000in}}%
\pgfpathlineto{\pgfqpoint{3.251777in}{0.385000in}}%
\pgfpathlineto{\pgfqpoint{3.251777in}{1.960814in}}%
\pgfpathlineto{\pgfqpoint{3.240248in}{1.960814in}}%
\pgfpathlineto{\pgfqpoint{3.240248in}{0.385000in}}%
\pgfpathclose%
\pgfusepath{fill}%
\end{pgfscope}%
\begin{pgfscope}%
\pgfpathrectangle{\pgfqpoint{0.750000in}{0.385000in}}{\pgfqpoint{4.650000in}{2.695000in}}%
\pgfusepath{clip}%
\pgfsetbuttcap%
\pgfsetmiterjoin%
\definecolor{currentfill}{rgb}{0.000000,0.500000,0.000000}%
\pgfsetfillcolor{currentfill}%
\pgfsetfillopacity{0.600000}%
\pgfsetlinewidth{0.000000pt}%
\definecolor{currentstroke}{rgb}{0.000000,0.000000,0.000000}%
\pgfsetstrokecolor{currentstroke}%
\pgfsetstrokeopacity{0.600000}%
\pgfsetdash{}{0pt}%
\pgfpathmoveto{\pgfqpoint{3.251777in}{0.385000in}}%
\pgfpathlineto{\pgfqpoint{3.263306in}{0.385000in}}%
\pgfpathlineto{\pgfqpoint{3.263306in}{1.853372in}}%
\pgfpathlineto{\pgfqpoint{3.251777in}{1.853372in}}%
\pgfpathlineto{\pgfqpoint{3.251777in}{0.385000in}}%
\pgfpathclose%
\pgfusepath{fill}%
\end{pgfscope}%
\begin{pgfscope}%
\pgfpathrectangle{\pgfqpoint{0.750000in}{0.385000in}}{\pgfqpoint{4.650000in}{2.695000in}}%
\pgfusepath{clip}%
\pgfsetbuttcap%
\pgfsetmiterjoin%
\definecolor{currentfill}{rgb}{0.000000,0.500000,0.000000}%
\pgfsetfillcolor{currentfill}%
\pgfsetfillopacity{0.600000}%
\pgfsetlinewidth{0.000000pt}%
\definecolor{currentstroke}{rgb}{0.000000,0.000000,0.000000}%
\pgfsetstrokecolor{currentstroke}%
\pgfsetstrokeopacity{0.600000}%
\pgfsetdash{}{0pt}%
\pgfpathmoveto{\pgfqpoint{3.263306in}{0.385000in}}%
\pgfpathlineto{\pgfqpoint{3.274835in}{0.385000in}}%
\pgfpathlineto{\pgfqpoint{3.274835in}{1.853372in}}%
\pgfpathlineto{\pgfqpoint{3.263306in}{1.853372in}}%
\pgfpathlineto{\pgfqpoint{3.263306in}{0.385000in}}%
\pgfpathclose%
\pgfusepath{fill}%
\end{pgfscope}%
\begin{pgfscope}%
\pgfpathrectangle{\pgfqpoint{0.750000in}{0.385000in}}{\pgfqpoint{4.650000in}{2.695000in}}%
\pgfusepath{clip}%
\pgfsetbuttcap%
\pgfsetmiterjoin%
\definecolor{currentfill}{rgb}{0.000000,0.500000,0.000000}%
\pgfsetfillcolor{currentfill}%
\pgfsetfillopacity{0.600000}%
\pgfsetlinewidth{0.000000pt}%
\definecolor{currentstroke}{rgb}{0.000000,0.000000,0.000000}%
\pgfsetstrokecolor{currentstroke}%
\pgfsetstrokeopacity{0.600000}%
\pgfsetdash{}{0pt}%
\pgfpathmoveto{\pgfqpoint{3.274835in}{0.385000in}}%
\pgfpathlineto{\pgfqpoint{3.286364in}{0.385000in}}%
\pgfpathlineto{\pgfqpoint{3.286364in}{1.841434in}}%
\pgfpathlineto{\pgfqpoint{3.274835in}{1.841434in}}%
\pgfpathlineto{\pgfqpoint{3.274835in}{0.385000in}}%
\pgfpathclose%
\pgfusepath{fill}%
\end{pgfscope}%
\begin{pgfscope}%
\pgfpathrectangle{\pgfqpoint{0.750000in}{0.385000in}}{\pgfqpoint{4.650000in}{2.695000in}}%
\pgfusepath{clip}%
\pgfsetbuttcap%
\pgfsetmiterjoin%
\definecolor{currentfill}{rgb}{0.000000,0.500000,0.000000}%
\pgfsetfillcolor{currentfill}%
\pgfsetfillopacity{0.600000}%
\pgfsetlinewidth{0.000000pt}%
\definecolor{currentstroke}{rgb}{0.000000,0.000000,0.000000}%
\pgfsetstrokecolor{currentstroke}%
\pgfsetstrokeopacity{0.600000}%
\pgfsetdash{}{0pt}%
\pgfpathmoveto{\pgfqpoint{3.286364in}{0.385000in}}%
\pgfpathlineto{\pgfqpoint{3.297893in}{0.385000in}}%
\pgfpathlineto{\pgfqpoint{3.297893in}{1.781744in}}%
\pgfpathlineto{\pgfqpoint{3.286364in}{1.781744in}}%
\pgfpathlineto{\pgfqpoint{3.286364in}{0.385000in}}%
\pgfpathclose%
\pgfusepath{fill}%
\end{pgfscope}%
\begin{pgfscope}%
\pgfpathrectangle{\pgfqpoint{0.750000in}{0.385000in}}{\pgfqpoint{4.650000in}{2.695000in}}%
\pgfusepath{clip}%
\pgfsetbuttcap%
\pgfsetmiterjoin%
\definecolor{currentfill}{rgb}{0.000000,0.500000,0.000000}%
\pgfsetfillcolor{currentfill}%
\pgfsetfillopacity{0.600000}%
\pgfsetlinewidth{0.000000pt}%
\definecolor{currentstroke}{rgb}{0.000000,0.000000,0.000000}%
\pgfsetstrokecolor{currentstroke}%
\pgfsetstrokeopacity{0.600000}%
\pgfsetdash{}{0pt}%
\pgfpathmoveto{\pgfqpoint{3.297893in}{0.385000in}}%
\pgfpathlineto{\pgfqpoint{3.309421in}{0.385000in}}%
\pgfpathlineto{\pgfqpoint{3.309421in}{1.745930in}}%
\pgfpathlineto{\pgfqpoint{3.297893in}{1.745930in}}%
\pgfpathlineto{\pgfqpoint{3.297893in}{0.385000in}}%
\pgfpathclose%
\pgfusepath{fill}%
\end{pgfscope}%
\begin{pgfscope}%
\pgfpathrectangle{\pgfqpoint{0.750000in}{0.385000in}}{\pgfqpoint{4.650000in}{2.695000in}}%
\pgfusepath{clip}%
\pgfsetbuttcap%
\pgfsetmiterjoin%
\definecolor{currentfill}{rgb}{0.000000,0.500000,0.000000}%
\pgfsetfillcolor{currentfill}%
\pgfsetfillopacity{0.600000}%
\pgfsetlinewidth{0.000000pt}%
\definecolor{currentstroke}{rgb}{0.000000,0.000000,0.000000}%
\pgfsetstrokecolor{currentstroke}%
\pgfsetstrokeopacity{0.600000}%
\pgfsetdash{}{0pt}%
\pgfpathmoveto{\pgfqpoint{3.309421in}{0.385000in}}%
\pgfpathlineto{\pgfqpoint{3.320950in}{0.385000in}}%
\pgfpathlineto{\pgfqpoint{3.320950in}{1.483295in}}%
\pgfpathlineto{\pgfqpoint{3.309421in}{1.483295in}}%
\pgfpathlineto{\pgfqpoint{3.309421in}{0.385000in}}%
\pgfpathclose%
\pgfusepath{fill}%
\end{pgfscope}%
\begin{pgfscope}%
\pgfpathrectangle{\pgfqpoint{0.750000in}{0.385000in}}{\pgfqpoint{4.650000in}{2.695000in}}%
\pgfusepath{clip}%
\pgfsetbuttcap%
\pgfsetmiterjoin%
\definecolor{currentfill}{rgb}{0.000000,0.500000,0.000000}%
\pgfsetfillcolor{currentfill}%
\pgfsetfillopacity{0.600000}%
\pgfsetlinewidth{0.000000pt}%
\definecolor{currentstroke}{rgb}{0.000000,0.000000,0.000000}%
\pgfsetstrokecolor{currentstroke}%
\pgfsetstrokeopacity{0.600000}%
\pgfsetdash{}{0pt}%
\pgfpathmoveto{\pgfqpoint{3.320950in}{0.385000in}}%
\pgfpathlineto{\pgfqpoint{3.332479in}{0.385000in}}%
\pgfpathlineto{\pgfqpoint{3.332479in}{1.662364in}}%
\pgfpathlineto{\pgfqpoint{3.320950in}{1.662364in}}%
\pgfpathlineto{\pgfqpoint{3.320950in}{0.385000in}}%
\pgfpathclose%
\pgfusepath{fill}%
\end{pgfscope}%
\begin{pgfscope}%
\pgfpathrectangle{\pgfqpoint{0.750000in}{0.385000in}}{\pgfqpoint{4.650000in}{2.695000in}}%
\pgfusepath{clip}%
\pgfsetbuttcap%
\pgfsetmiterjoin%
\definecolor{currentfill}{rgb}{0.000000,0.500000,0.000000}%
\pgfsetfillcolor{currentfill}%
\pgfsetfillopacity{0.600000}%
\pgfsetlinewidth{0.000000pt}%
\definecolor{currentstroke}{rgb}{0.000000,0.000000,0.000000}%
\pgfsetstrokecolor{currentstroke}%
\pgfsetstrokeopacity{0.600000}%
\pgfsetdash{}{0pt}%
\pgfpathmoveto{\pgfqpoint{3.332479in}{0.385000in}}%
\pgfpathlineto{\pgfqpoint{3.344008in}{0.385000in}}%
\pgfpathlineto{\pgfqpoint{3.344008in}{1.411667in}}%
\pgfpathlineto{\pgfqpoint{3.332479in}{1.411667in}}%
\pgfpathlineto{\pgfqpoint{3.332479in}{0.385000in}}%
\pgfpathclose%
\pgfusepath{fill}%
\end{pgfscope}%
\begin{pgfscope}%
\pgfpathrectangle{\pgfqpoint{0.750000in}{0.385000in}}{\pgfqpoint{4.650000in}{2.695000in}}%
\pgfusepath{clip}%
\pgfsetbuttcap%
\pgfsetmiterjoin%
\definecolor{currentfill}{rgb}{0.000000,0.500000,0.000000}%
\pgfsetfillcolor{currentfill}%
\pgfsetfillopacity{0.600000}%
\pgfsetlinewidth{0.000000pt}%
\definecolor{currentstroke}{rgb}{0.000000,0.000000,0.000000}%
\pgfsetstrokecolor{currentstroke}%
\pgfsetstrokeopacity{0.600000}%
\pgfsetdash{}{0pt}%
\pgfpathmoveto{\pgfqpoint{3.344008in}{0.385000in}}%
\pgfpathlineto{\pgfqpoint{3.355537in}{0.385000in}}%
\pgfpathlineto{\pgfqpoint{3.355537in}{1.554922in}}%
\pgfpathlineto{\pgfqpoint{3.344008in}{1.554922in}}%
\pgfpathlineto{\pgfqpoint{3.344008in}{0.385000in}}%
\pgfpathclose%
\pgfusepath{fill}%
\end{pgfscope}%
\begin{pgfscope}%
\pgfpathrectangle{\pgfqpoint{0.750000in}{0.385000in}}{\pgfqpoint{4.650000in}{2.695000in}}%
\pgfusepath{clip}%
\pgfsetbuttcap%
\pgfsetmiterjoin%
\definecolor{currentfill}{rgb}{0.000000,0.500000,0.000000}%
\pgfsetfillcolor{currentfill}%
\pgfsetfillopacity{0.600000}%
\pgfsetlinewidth{0.000000pt}%
\definecolor{currentstroke}{rgb}{0.000000,0.000000,0.000000}%
\pgfsetstrokecolor{currentstroke}%
\pgfsetstrokeopacity{0.600000}%
\pgfsetdash{}{0pt}%
\pgfpathmoveto{\pgfqpoint{3.355537in}{0.385000in}}%
\pgfpathlineto{\pgfqpoint{3.367066in}{0.385000in}}%
\pgfpathlineto{\pgfqpoint{3.367066in}{1.602674in}}%
\pgfpathlineto{\pgfqpoint{3.355537in}{1.602674in}}%
\pgfpathlineto{\pgfqpoint{3.355537in}{0.385000in}}%
\pgfpathclose%
\pgfusepath{fill}%
\end{pgfscope}%
\begin{pgfscope}%
\pgfpathrectangle{\pgfqpoint{0.750000in}{0.385000in}}{\pgfqpoint{4.650000in}{2.695000in}}%
\pgfusepath{clip}%
\pgfsetbuttcap%
\pgfsetmiterjoin%
\definecolor{currentfill}{rgb}{0.000000,0.500000,0.000000}%
\pgfsetfillcolor{currentfill}%
\pgfsetfillopacity{0.600000}%
\pgfsetlinewidth{0.000000pt}%
\definecolor{currentstroke}{rgb}{0.000000,0.000000,0.000000}%
\pgfsetstrokecolor{currentstroke}%
\pgfsetstrokeopacity{0.600000}%
\pgfsetdash{}{0pt}%
\pgfpathmoveto{\pgfqpoint{3.367066in}{0.385000in}}%
\pgfpathlineto{\pgfqpoint{3.378595in}{0.385000in}}%
\pgfpathlineto{\pgfqpoint{3.378595in}{1.602674in}}%
\pgfpathlineto{\pgfqpoint{3.367066in}{1.602674in}}%
\pgfpathlineto{\pgfqpoint{3.367066in}{0.385000in}}%
\pgfpathclose%
\pgfusepath{fill}%
\end{pgfscope}%
\begin{pgfscope}%
\pgfpathrectangle{\pgfqpoint{0.750000in}{0.385000in}}{\pgfqpoint{4.650000in}{2.695000in}}%
\pgfusepath{clip}%
\pgfsetbuttcap%
\pgfsetmiterjoin%
\definecolor{currentfill}{rgb}{0.000000,0.500000,0.000000}%
\pgfsetfillcolor{currentfill}%
\pgfsetfillopacity{0.600000}%
\pgfsetlinewidth{0.000000pt}%
\definecolor{currentstroke}{rgb}{0.000000,0.000000,0.000000}%
\pgfsetstrokecolor{currentstroke}%
\pgfsetstrokeopacity{0.600000}%
\pgfsetdash{}{0pt}%
\pgfpathmoveto{\pgfqpoint{3.378595in}{0.385000in}}%
\pgfpathlineto{\pgfqpoint{3.390124in}{0.385000in}}%
\pgfpathlineto{\pgfqpoint{3.390124in}{1.375853in}}%
\pgfpathlineto{\pgfqpoint{3.378595in}{1.375853in}}%
\pgfpathlineto{\pgfqpoint{3.378595in}{0.385000in}}%
\pgfpathclose%
\pgfusepath{fill}%
\end{pgfscope}%
\begin{pgfscope}%
\pgfpathrectangle{\pgfqpoint{0.750000in}{0.385000in}}{\pgfqpoint{4.650000in}{2.695000in}}%
\pgfusepath{clip}%
\pgfsetbuttcap%
\pgfsetmiterjoin%
\definecolor{currentfill}{rgb}{0.000000,0.500000,0.000000}%
\pgfsetfillcolor{currentfill}%
\pgfsetfillopacity{0.600000}%
\pgfsetlinewidth{0.000000pt}%
\definecolor{currentstroke}{rgb}{0.000000,0.000000,0.000000}%
\pgfsetstrokecolor{currentstroke}%
\pgfsetstrokeopacity{0.600000}%
\pgfsetdash{}{0pt}%
\pgfpathmoveto{\pgfqpoint{3.390124in}{0.385000in}}%
\pgfpathlineto{\pgfqpoint{3.401653in}{0.385000in}}%
\pgfpathlineto{\pgfqpoint{3.401653in}{1.423605in}}%
\pgfpathlineto{\pgfqpoint{3.390124in}{1.423605in}}%
\pgfpathlineto{\pgfqpoint{3.390124in}{0.385000in}}%
\pgfpathclose%
\pgfusepath{fill}%
\end{pgfscope}%
\begin{pgfscope}%
\pgfpathrectangle{\pgfqpoint{0.750000in}{0.385000in}}{\pgfqpoint{4.650000in}{2.695000in}}%
\pgfusepath{clip}%
\pgfsetbuttcap%
\pgfsetmiterjoin%
\definecolor{currentfill}{rgb}{0.000000,0.500000,0.000000}%
\pgfsetfillcolor{currentfill}%
\pgfsetfillopacity{0.600000}%
\pgfsetlinewidth{0.000000pt}%
\definecolor{currentstroke}{rgb}{0.000000,0.000000,0.000000}%
\pgfsetstrokecolor{currentstroke}%
\pgfsetstrokeopacity{0.600000}%
\pgfsetdash{}{0pt}%
\pgfpathmoveto{\pgfqpoint{3.401653in}{0.385000in}}%
\pgfpathlineto{\pgfqpoint{3.413182in}{0.385000in}}%
\pgfpathlineto{\pgfqpoint{3.413182in}{1.268411in}}%
\pgfpathlineto{\pgfqpoint{3.401653in}{1.268411in}}%
\pgfpathlineto{\pgfqpoint{3.401653in}{0.385000in}}%
\pgfpathclose%
\pgfusepath{fill}%
\end{pgfscope}%
\begin{pgfscope}%
\pgfpathrectangle{\pgfqpoint{0.750000in}{0.385000in}}{\pgfqpoint{4.650000in}{2.695000in}}%
\pgfusepath{clip}%
\pgfsetbuttcap%
\pgfsetmiterjoin%
\definecolor{currentfill}{rgb}{0.000000,0.500000,0.000000}%
\pgfsetfillcolor{currentfill}%
\pgfsetfillopacity{0.600000}%
\pgfsetlinewidth{0.000000pt}%
\definecolor{currentstroke}{rgb}{0.000000,0.000000,0.000000}%
\pgfsetstrokecolor{currentstroke}%
\pgfsetstrokeopacity{0.600000}%
\pgfsetdash{}{0pt}%
\pgfpathmoveto{\pgfqpoint{3.413182in}{0.385000in}}%
\pgfpathlineto{\pgfqpoint{3.424711in}{0.385000in}}%
\pgfpathlineto{\pgfqpoint{3.424711in}{1.423605in}}%
\pgfpathlineto{\pgfqpoint{3.413182in}{1.423605in}}%
\pgfpathlineto{\pgfqpoint{3.413182in}{0.385000in}}%
\pgfpathclose%
\pgfusepath{fill}%
\end{pgfscope}%
\begin{pgfscope}%
\pgfpathrectangle{\pgfqpoint{0.750000in}{0.385000in}}{\pgfqpoint{4.650000in}{2.695000in}}%
\pgfusepath{clip}%
\pgfsetbuttcap%
\pgfsetmiterjoin%
\definecolor{currentfill}{rgb}{0.000000,0.500000,0.000000}%
\pgfsetfillcolor{currentfill}%
\pgfsetfillopacity{0.600000}%
\pgfsetlinewidth{0.000000pt}%
\definecolor{currentstroke}{rgb}{0.000000,0.000000,0.000000}%
\pgfsetstrokecolor{currentstroke}%
\pgfsetstrokeopacity{0.600000}%
\pgfsetdash{}{0pt}%
\pgfpathmoveto{\pgfqpoint{3.424711in}{0.385000in}}%
\pgfpathlineto{\pgfqpoint{3.436240in}{0.385000in}}%
\pgfpathlineto{\pgfqpoint{3.436240in}{1.363915in}}%
\pgfpathlineto{\pgfqpoint{3.424711in}{1.363915in}}%
\pgfpathlineto{\pgfqpoint{3.424711in}{0.385000in}}%
\pgfpathclose%
\pgfusepath{fill}%
\end{pgfscope}%
\begin{pgfscope}%
\pgfpathrectangle{\pgfqpoint{0.750000in}{0.385000in}}{\pgfqpoint{4.650000in}{2.695000in}}%
\pgfusepath{clip}%
\pgfsetbuttcap%
\pgfsetmiterjoin%
\definecolor{currentfill}{rgb}{0.000000,0.500000,0.000000}%
\pgfsetfillcolor{currentfill}%
\pgfsetfillopacity{0.600000}%
\pgfsetlinewidth{0.000000pt}%
\definecolor{currentstroke}{rgb}{0.000000,0.000000,0.000000}%
\pgfsetstrokecolor{currentstroke}%
\pgfsetstrokeopacity{0.600000}%
\pgfsetdash{}{0pt}%
\pgfpathmoveto{\pgfqpoint{3.436240in}{0.385000in}}%
\pgfpathlineto{\pgfqpoint{3.447769in}{0.385000in}}%
\pgfpathlineto{\pgfqpoint{3.447769in}{1.328101in}}%
\pgfpathlineto{\pgfqpoint{3.436240in}{1.328101in}}%
\pgfpathlineto{\pgfqpoint{3.436240in}{0.385000in}}%
\pgfpathclose%
\pgfusepath{fill}%
\end{pgfscope}%
\begin{pgfscope}%
\pgfpathrectangle{\pgfqpoint{0.750000in}{0.385000in}}{\pgfqpoint{4.650000in}{2.695000in}}%
\pgfusepath{clip}%
\pgfsetbuttcap%
\pgfsetmiterjoin%
\definecolor{currentfill}{rgb}{0.000000,0.500000,0.000000}%
\pgfsetfillcolor{currentfill}%
\pgfsetfillopacity{0.600000}%
\pgfsetlinewidth{0.000000pt}%
\definecolor{currentstroke}{rgb}{0.000000,0.000000,0.000000}%
\pgfsetstrokecolor{currentstroke}%
\pgfsetstrokeopacity{0.600000}%
\pgfsetdash{}{0pt}%
\pgfpathmoveto{\pgfqpoint{3.447769in}{0.385000in}}%
\pgfpathlineto{\pgfqpoint{3.459298in}{0.385000in}}%
\pgfpathlineto{\pgfqpoint{3.459298in}{1.196783in}}%
\pgfpathlineto{\pgfqpoint{3.447769in}{1.196783in}}%
\pgfpathlineto{\pgfqpoint{3.447769in}{0.385000in}}%
\pgfpathclose%
\pgfusepath{fill}%
\end{pgfscope}%
\begin{pgfscope}%
\pgfpathrectangle{\pgfqpoint{0.750000in}{0.385000in}}{\pgfqpoint{4.650000in}{2.695000in}}%
\pgfusepath{clip}%
\pgfsetbuttcap%
\pgfsetmiterjoin%
\definecolor{currentfill}{rgb}{0.000000,0.500000,0.000000}%
\pgfsetfillcolor{currentfill}%
\pgfsetfillopacity{0.600000}%
\pgfsetlinewidth{0.000000pt}%
\definecolor{currentstroke}{rgb}{0.000000,0.000000,0.000000}%
\pgfsetstrokecolor{currentstroke}%
\pgfsetstrokeopacity{0.600000}%
\pgfsetdash{}{0pt}%
\pgfpathmoveto{\pgfqpoint{3.459298in}{0.385000in}}%
\pgfpathlineto{\pgfqpoint{3.470826in}{0.385000in}}%
\pgfpathlineto{\pgfqpoint{3.470826in}{1.280349in}}%
\pgfpathlineto{\pgfqpoint{3.459298in}{1.280349in}}%
\pgfpathlineto{\pgfqpoint{3.459298in}{0.385000in}}%
\pgfpathclose%
\pgfusepath{fill}%
\end{pgfscope}%
\begin{pgfscope}%
\pgfpathrectangle{\pgfqpoint{0.750000in}{0.385000in}}{\pgfqpoint{4.650000in}{2.695000in}}%
\pgfusepath{clip}%
\pgfsetbuttcap%
\pgfsetmiterjoin%
\definecolor{currentfill}{rgb}{0.000000,0.500000,0.000000}%
\pgfsetfillcolor{currentfill}%
\pgfsetfillopacity{0.600000}%
\pgfsetlinewidth{0.000000pt}%
\definecolor{currentstroke}{rgb}{0.000000,0.000000,0.000000}%
\pgfsetstrokecolor{currentstroke}%
\pgfsetstrokeopacity{0.600000}%
\pgfsetdash{}{0pt}%
\pgfpathmoveto{\pgfqpoint{3.470826in}{0.385000in}}%
\pgfpathlineto{\pgfqpoint{3.482355in}{0.385000in}}%
\pgfpathlineto{\pgfqpoint{3.482355in}{1.065465in}}%
\pgfpathlineto{\pgfqpoint{3.470826in}{1.065465in}}%
\pgfpathlineto{\pgfqpoint{3.470826in}{0.385000in}}%
\pgfpathclose%
\pgfusepath{fill}%
\end{pgfscope}%
\begin{pgfscope}%
\pgfpathrectangle{\pgfqpoint{0.750000in}{0.385000in}}{\pgfqpoint{4.650000in}{2.695000in}}%
\pgfusepath{clip}%
\pgfsetbuttcap%
\pgfsetmiterjoin%
\definecolor{currentfill}{rgb}{0.000000,0.500000,0.000000}%
\pgfsetfillcolor{currentfill}%
\pgfsetfillopacity{0.600000}%
\pgfsetlinewidth{0.000000pt}%
\definecolor{currentstroke}{rgb}{0.000000,0.000000,0.000000}%
\pgfsetstrokecolor{currentstroke}%
\pgfsetstrokeopacity{0.600000}%
\pgfsetdash{}{0pt}%
\pgfpathmoveto{\pgfqpoint{3.482355in}{0.385000in}}%
\pgfpathlineto{\pgfqpoint{3.493884in}{0.385000in}}%
\pgfpathlineto{\pgfqpoint{3.493884in}{1.220659in}}%
\pgfpathlineto{\pgfqpoint{3.482355in}{1.220659in}}%
\pgfpathlineto{\pgfqpoint{3.482355in}{0.385000in}}%
\pgfpathclose%
\pgfusepath{fill}%
\end{pgfscope}%
\begin{pgfscope}%
\pgfpathrectangle{\pgfqpoint{0.750000in}{0.385000in}}{\pgfqpoint{4.650000in}{2.695000in}}%
\pgfusepath{clip}%
\pgfsetbuttcap%
\pgfsetmiterjoin%
\definecolor{currentfill}{rgb}{0.000000,0.500000,0.000000}%
\pgfsetfillcolor{currentfill}%
\pgfsetfillopacity{0.600000}%
\pgfsetlinewidth{0.000000pt}%
\definecolor{currentstroke}{rgb}{0.000000,0.000000,0.000000}%
\pgfsetstrokecolor{currentstroke}%
\pgfsetstrokeopacity{0.600000}%
\pgfsetdash{}{0pt}%
\pgfpathmoveto{\pgfqpoint{3.493884in}{0.385000in}}%
\pgfpathlineto{\pgfqpoint{3.505413in}{0.385000in}}%
\pgfpathlineto{\pgfqpoint{3.505413in}{1.065465in}}%
\pgfpathlineto{\pgfqpoint{3.493884in}{1.065465in}}%
\pgfpathlineto{\pgfqpoint{3.493884in}{0.385000in}}%
\pgfpathclose%
\pgfusepath{fill}%
\end{pgfscope}%
\begin{pgfscope}%
\pgfpathrectangle{\pgfqpoint{0.750000in}{0.385000in}}{\pgfqpoint{4.650000in}{2.695000in}}%
\pgfusepath{clip}%
\pgfsetbuttcap%
\pgfsetmiterjoin%
\definecolor{currentfill}{rgb}{0.000000,0.500000,0.000000}%
\pgfsetfillcolor{currentfill}%
\pgfsetfillopacity{0.600000}%
\pgfsetlinewidth{0.000000pt}%
\definecolor{currentstroke}{rgb}{0.000000,0.000000,0.000000}%
\pgfsetstrokecolor{currentstroke}%
\pgfsetstrokeopacity{0.600000}%
\pgfsetdash{}{0pt}%
\pgfpathmoveto{\pgfqpoint{3.505413in}{0.385000in}}%
\pgfpathlineto{\pgfqpoint{3.516942in}{0.385000in}}%
\pgfpathlineto{\pgfqpoint{3.516942in}{1.328101in}}%
\pgfpathlineto{\pgfqpoint{3.505413in}{1.328101in}}%
\pgfpathlineto{\pgfqpoint{3.505413in}{0.385000in}}%
\pgfpathclose%
\pgfusepath{fill}%
\end{pgfscope}%
\begin{pgfscope}%
\pgfpathrectangle{\pgfqpoint{0.750000in}{0.385000in}}{\pgfqpoint{4.650000in}{2.695000in}}%
\pgfusepath{clip}%
\pgfsetbuttcap%
\pgfsetmiterjoin%
\definecolor{currentfill}{rgb}{0.000000,0.500000,0.000000}%
\pgfsetfillcolor{currentfill}%
\pgfsetfillopacity{0.600000}%
\pgfsetlinewidth{0.000000pt}%
\definecolor{currentstroke}{rgb}{0.000000,0.000000,0.000000}%
\pgfsetstrokecolor{currentstroke}%
\pgfsetstrokeopacity{0.600000}%
\pgfsetdash{}{0pt}%
\pgfpathmoveto{\pgfqpoint{3.516942in}{0.385000in}}%
\pgfpathlineto{\pgfqpoint{3.528471in}{0.385000in}}%
\pgfpathlineto{\pgfqpoint{3.528471in}{0.969961in}}%
\pgfpathlineto{\pgfqpoint{3.516942in}{0.969961in}}%
\pgfpathlineto{\pgfqpoint{3.516942in}{0.385000in}}%
\pgfpathclose%
\pgfusepath{fill}%
\end{pgfscope}%
\begin{pgfscope}%
\pgfpathrectangle{\pgfqpoint{0.750000in}{0.385000in}}{\pgfqpoint{4.650000in}{2.695000in}}%
\pgfusepath{clip}%
\pgfsetbuttcap%
\pgfsetmiterjoin%
\definecolor{currentfill}{rgb}{0.000000,0.500000,0.000000}%
\pgfsetfillcolor{currentfill}%
\pgfsetfillopacity{0.600000}%
\pgfsetlinewidth{0.000000pt}%
\definecolor{currentstroke}{rgb}{0.000000,0.000000,0.000000}%
\pgfsetstrokecolor{currentstroke}%
\pgfsetstrokeopacity{0.600000}%
\pgfsetdash{}{0pt}%
\pgfpathmoveto{\pgfqpoint{3.528471in}{0.385000in}}%
\pgfpathlineto{\pgfqpoint{3.540000in}{0.385000in}}%
\pgfpathlineto{\pgfqpoint{3.540000in}{0.993837in}}%
\pgfpathlineto{\pgfqpoint{3.528471in}{0.993837in}}%
\pgfpathlineto{\pgfqpoint{3.528471in}{0.385000in}}%
\pgfpathclose%
\pgfusepath{fill}%
\end{pgfscope}%
\begin{pgfscope}%
\pgfpathrectangle{\pgfqpoint{0.750000in}{0.385000in}}{\pgfqpoint{4.650000in}{2.695000in}}%
\pgfusepath{clip}%
\pgfsetbuttcap%
\pgfsetmiterjoin%
\definecolor{currentfill}{rgb}{0.000000,0.500000,0.000000}%
\pgfsetfillcolor{currentfill}%
\pgfsetfillopacity{0.600000}%
\pgfsetlinewidth{0.000000pt}%
\definecolor{currentstroke}{rgb}{0.000000,0.000000,0.000000}%
\pgfsetstrokecolor{currentstroke}%
\pgfsetstrokeopacity{0.600000}%
\pgfsetdash{}{0pt}%
\pgfpathmoveto{\pgfqpoint{3.540000in}{0.385000in}}%
\pgfpathlineto{\pgfqpoint{3.551529in}{0.385000in}}%
\pgfpathlineto{\pgfqpoint{3.551529in}{0.946085in}}%
\pgfpathlineto{\pgfqpoint{3.540000in}{0.946085in}}%
\pgfpathlineto{\pgfqpoint{3.540000in}{0.385000in}}%
\pgfpathclose%
\pgfusepath{fill}%
\end{pgfscope}%
\begin{pgfscope}%
\pgfpathrectangle{\pgfqpoint{0.750000in}{0.385000in}}{\pgfqpoint{4.650000in}{2.695000in}}%
\pgfusepath{clip}%
\pgfsetbuttcap%
\pgfsetmiterjoin%
\definecolor{currentfill}{rgb}{0.000000,0.500000,0.000000}%
\pgfsetfillcolor{currentfill}%
\pgfsetfillopacity{0.600000}%
\pgfsetlinewidth{0.000000pt}%
\definecolor{currentstroke}{rgb}{0.000000,0.000000,0.000000}%
\pgfsetstrokecolor{currentstroke}%
\pgfsetstrokeopacity{0.600000}%
\pgfsetdash{}{0pt}%
\pgfpathmoveto{\pgfqpoint{3.551529in}{0.385000in}}%
\pgfpathlineto{\pgfqpoint{3.563058in}{0.385000in}}%
\pgfpathlineto{\pgfqpoint{3.563058in}{1.101279in}}%
\pgfpathlineto{\pgfqpoint{3.551529in}{1.101279in}}%
\pgfpathlineto{\pgfqpoint{3.551529in}{0.385000in}}%
\pgfpathclose%
\pgfusepath{fill}%
\end{pgfscope}%
\begin{pgfscope}%
\pgfpathrectangle{\pgfqpoint{0.750000in}{0.385000in}}{\pgfqpoint{4.650000in}{2.695000in}}%
\pgfusepath{clip}%
\pgfsetbuttcap%
\pgfsetmiterjoin%
\definecolor{currentfill}{rgb}{0.000000,0.500000,0.000000}%
\pgfsetfillcolor{currentfill}%
\pgfsetfillopacity{0.600000}%
\pgfsetlinewidth{0.000000pt}%
\definecolor{currentstroke}{rgb}{0.000000,0.000000,0.000000}%
\pgfsetstrokecolor{currentstroke}%
\pgfsetstrokeopacity{0.600000}%
\pgfsetdash{}{0pt}%
\pgfpathmoveto{\pgfqpoint{3.563058in}{0.385000in}}%
\pgfpathlineto{\pgfqpoint{3.574587in}{0.385000in}}%
\pgfpathlineto{\pgfqpoint{3.574587in}{0.981899in}}%
\pgfpathlineto{\pgfqpoint{3.563058in}{0.981899in}}%
\pgfpathlineto{\pgfqpoint{3.563058in}{0.385000in}}%
\pgfpathclose%
\pgfusepath{fill}%
\end{pgfscope}%
\begin{pgfscope}%
\pgfpathrectangle{\pgfqpoint{0.750000in}{0.385000in}}{\pgfqpoint{4.650000in}{2.695000in}}%
\pgfusepath{clip}%
\pgfsetbuttcap%
\pgfsetmiterjoin%
\definecolor{currentfill}{rgb}{0.000000,0.500000,0.000000}%
\pgfsetfillcolor{currentfill}%
\pgfsetfillopacity{0.600000}%
\pgfsetlinewidth{0.000000pt}%
\definecolor{currentstroke}{rgb}{0.000000,0.000000,0.000000}%
\pgfsetstrokecolor{currentstroke}%
\pgfsetstrokeopacity{0.600000}%
\pgfsetdash{}{0pt}%
\pgfpathmoveto{\pgfqpoint{3.574587in}{0.385000in}}%
\pgfpathlineto{\pgfqpoint{3.586116in}{0.385000in}}%
\pgfpathlineto{\pgfqpoint{3.586116in}{0.993837in}}%
\pgfpathlineto{\pgfqpoint{3.574587in}{0.993837in}}%
\pgfpathlineto{\pgfqpoint{3.574587in}{0.385000in}}%
\pgfpathclose%
\pgfusepath{fill}%
\end{pgfscope}%
\begin{pgfscope}%
\pgfpathrectangle{\pgfqpoint{0.750000in}{0.385000in}}{\pgfqpoint{4.650000in}{2.695000in}}%
\pgfusepath{clip}%
\pgfsetbuttcap%
\pgfsetmiterjoin%
\definecolor{currentfill}{rgb}{0.000000,0.500000,0.000000}%
\pgfsetfillcolor{currentfill}%
\pgfsetfillopacity{0.600000}%
\pgfsetlinewidth{0.000000pt}%
\definecolor{currentstroke}{rgb}{0.000000,0.000000,0.000000}%
\pgfsetstrokecolor{currentstroke}%
\pgfsetstrokeopacity{0.600000}%
\pgfsetdash{}{0pt}%
\pgfpathmoveto{\pgfqpoint{3.586116in}{0.385000in}}%
\pgfpathlineto{\pgfqpoint{3.597645in}{0.385000in}}%
\pgfpathlineto{\pgfqpoint{3.597645in}{1.041589in}}%
\pgfpathlineto{\pgfqpoint{3.586116in}{1.041589in}}%
\pgfpathlineto{\pgfqpoint{3.586116in}{0.385000in}}%
\pgfpathclose%
\pgfusepath{fill}%
\end{pgfscope}%
\begin{pgfscope}%
\pgfpathrectangle{\pgfqpoint{0.750000in}{0.385000in}}{\pgfqpoint{4.650000in}{2.695000in}}%
\pgfusepath{clip}%
\pgfsetbuttcap%
\pgfsetmiterjoin%
\definecolor{currentfill}{rgb}{0.000000,0.500000,0.000000}%
\pgfsetfillcolor{currentfill}%
\pgfsetfillopacity{0.600000}%
\pgfsetlinewidth{0.000000pt}%
\definecolor{currentstroke}{rgb}{0.000000,0.000000,0.000000}%
\pgfsetstrokecolor{currentstroke}%
\pgfsetstrokeopacity{0.600000}%
\pgfsetdash{}{0pt}%
\pgfpathmoveto{\pgfqpoint{3.597645in}{0.385000in}}%
\pgfpathlineto{\pgfqpoint{3.609174in}{0.385000in}}%
\pgfpathlineto{\pgfqpoint{3.609174in}{0.993837in}}%
\pgfpathlineto{\pgfqpoint{3.597645in}{0.993837in}}%
\pgfpathlineto{\pgfqpoint{3.597645in}{0.385000in}}%
\pgfpathclose%
\pgfusepath{fill}%
\end{pgfscope}%
\begin{pgfscope}%
\pgfpathrectangle{\pgfqpoint{0.750000in}{0.385000in}}{\pgfqpoint{4.650000in}{2.695000in}}%
\pgfusepath{clip}%
\pgfsetbuttcap%
\pgfsetmiterjoin%
\definecolor{currentfill}{rgb}{0.000000,0.500000,0.000000}%
\pgfsetfillcolor{currentfill}%
\pgfsetfillopacity{0.600000}%
\pgfsetlinewidth{0.000000pt}%
\definecolor{currentstroke}{rgb}{0.000000,0.000000,0.000000}%
\pgfsetstrokecolor{currentstroke}%
\pgfsetstrokeopacity{0.600000}%
\pgfsetdash{}{0pt}%
\pgfpathmoveto{\pgfqpoint{3.609174in}{0.385000in}}%
\pgfpathlineto{\pgfqpoint{3.620702in}{0.385000in}}%
\pgfpathlineto{\pgfqpoint{3.620702in}{0.910271in}}%
\pgfpathlineto{\pgfqpoint{3.609174in}{0.910271in}}%
\pgfpathlineto{\pgfqpoint{3.609174in}{0.385000in}}%
\pgfpathclose%
\pgfusepath{fill}%
\end{pgfscope}%
\begin{pgfscope}%
\pgfpathrectangle{\pgfqpoint{0.750000in}{0.385000in}}{\pgfqpoint{4.650000in}{2.695000in}}%
\pgfusepath{clip}%
\pgfsetbuttcap%
\pgfsetmiterjoin%
\definecolor{currentfill}{rgb}{0.000000,0.500000,0.000000}%
\pgfsetfillcolor{currentfill}%
\pgfsetfillopacity{0.600000}%
\pgfsetlinewidth{0.000000pt}%
\definecolor{currentstroke}{rgb}{0.000000,0.000000,0.000000}%
\pgfsetstrokecolor{currentstroke}%
\pgfsetstrokeopacity{0.600000}%
\pgfsetdash{}{0pt}%
\pgfpathmoveto{\pgfqpoint{3.620702in}{0.385000in}}%
\pgfpathlineto{\pgfqpoint{3.632231in}{0.385000in}}%
\pgfpathlineto{\pgfqpoint{3.632231in}{1.160969in}}%
\pgfpathlineto{\pgfqpoint{3.620702in}{1.160969in}}%
\pgfpathlineto{\pgfqpoint{3.620702in}{0.385000in}}%
\pgfpathclose%
\pgfusepath{fill}%
\end{pgfscope}%
\begin{pgfscope}%
\pgfpathrectangle{\pgfqpoint{0.750000in}{0.385000in}}{\pgfqpoint{4.650000in}{2.695000in}}%
\pgfusepath{clip}%
\pgfsetbuttcap%
\pgfsetmiterjoin%
\definecolor{currentfill}{rgb}{0.000000,0.500000,0.000000}%
\pgfsetfillcolor{currentfill}%
\pgfsetfillopacity{0.600000}%
\pgfsetlinewidth{0.000000pt}%
\definecolor{currentstroke}{rgb}{0.000000,0.000000,0.000000}%
\pgfsetstrokecolor{currentstroke}%
\pgfsetstrokeopacity{0.600000}%
\pgfsetdash{}{0pt}%
\pgfpathmoveto{\pgfqpoint{3.632231in}{0.385000in}}%
\pgfpathlineto{\pgfqpoint{3.643760in}{0.385000in}}%
\pgfpathlineto{\pgfqpoint{3.643760in}{1.196783in}}%
\pgfpathlineto{\pgfqpoint{3.632231in}{1.196783in}}%
\pgfpathlineto{\pgfqpoint{3.632231in}{0.385000in}}%
\pgfpathclose%
\pgfusepath{fill}%
\end{pgfscope}%
\begin{pgfscope}%
\pgfpathrectangle{\pgfqpoint{0.750000in}{0.385000in}}{\pgfqpoint{4.650000in}{2.695000in}}%
\pgfusepath{clip}%
\pgfsetbuttcap%
\pgfsetmiterjoin%
\definecolor{currentfill}{rgb}{0.000000,0.500000,0.000000}%
\pgfsetfillcolor{currentfill}%
\pgfsetfillopacity{0.600000}%
\pgfsetlinewidth{0.000000pt}%
\definecolor{currentstroke}{rgb}{0.000000,0.000000,0.000000}%
\pgfsetstrokecolor{currentstroke}%
\pgfsetstrokeopacity{0.600000}%
\pgfsetdash{}{0pt}%
\pgfpathmoveto{\pgfqpoint{3.643760in}{0.385000in}}%
\pgfpathlineto{\pgfqpoint{3.655289in}{0.385000in}}%
\pgfpathlineto{\pgfqpoint{3.655289in}{0.922209in}}%
\pgfpathlineto{\pgfqpoint{3.643760in}{0.922209in}}%
\pgfpathlineto{\pgfqpoint{3.643760in}{0.385000in}}%
\pgfpathclose%
\pgfusepath{fill}%
\end{pgfscope}%
\begin{pgfscope}%
\pgfpathrectangle{\pgfqpoint{0.750000in}{0.385000in}}{\pgfqpoint{4.650000in}{2.695000in}}%
\pgfusepath{clip}%
\pgfsetbuttcap%
\pgfsetmiterjoin%
\definecolor{currentfill}{rgb}{0.000000,0.500000,0.000000}%
\pgfsetfillcolor{currentfill}%
\pgfsetfillopacity{0.600000}%
\pgfsetlinewidth{0.000000pt}%
\definecolor{currentstroke}{rgb}{0.000000,0.000000,0.000000}%
\pgfsetstrokecolor{currentstroke}%
\pgfsetstrokeopacity{0.600000}%
\pgfsetdash{}{0pt}%
\pgfpathmoveto{\pgfqpoint{3.655289in}{0.385000in}}%
\pgfpathlineto{\pgfqpoint{3.666818in}{0.385000in}}%
\pgfpathlineto{\pgfqpoint{3.666818in}{0.862519in}}%
\pgfpathlineto{\pgfqpoint{3.655289in}{0.862519in}}%
\pgfpathlineto{\pgfqpoint{3.655289in}{0.385000in}}%
\pgfpathclose%
\pgfusepath{fill}%
\end{pgfscope}%
\begin{pgfscope}%
\pgfpathrectangle{\pgfqpoint{0.750000in}{0.385000in}}{\pgfqpoint{4.650000in}{2.695000in}}%
\pgfusepath{clip}%
\pgfsetbuttcap%
\pgfsetmiterjoin%
\definecolor{currentfill}{rgb}{0.000000,0.500000,0.000000}%
\pgfsetfillcolor{currentfill}%
\pgfsetfillopacity{0.600000}%
\pgfsetlinewidth{0.000000pt}%
\definecolor{currentstroke}{rgb}{0.000000,0.000000,0.000000}%
\pgfsetstrokecolor{currentstroke}%
\pgfsetstrokeopacity{0.600000}%
\pgfsetdash{}{0pt}%
\pgfpathmoveto{\pgfqpoint{3.666818in}{0.385000in}}%
\pgfpathlineto{\pgfqpoint{3.678347in}{0.385000in}}%
\pgfpathlineto{\pgfqpoint{3.678347in}{1.005775in}}%
\pgfpathlineto{\pgfqpoint{3.666818in}{1.005775in}}%
\pgfpathlineto{\pgfqpoint{3.666818in}{0.385000in}}%
\pgfpathclose%
\pgfusepath{fill}%
\end{pgfscope}%
\begin{pgfscope}%
\pgfpathrectangle{\pgfqpoint{0.750000in}{0.385000in}}{\pgfqpoint{4.650000in}{2.695000in}}%
\pgfusepath{clip}%
\pgfsetbuttcap%
\pgfsetmiterjoin%
\definecolor{currentfill}{rgb}{0.000000,0.500000,0.000000}%
\pgfsetfillcolor{currentfill}%
\pgfsetfillopacity{0.600000}%
\pgfsetlinewidth{0.000000pt}%
\definecolor{currentstroke}{rgb}{0.000000,0.000000,0.000000}%
\pgfsetstrokecolor{currentstroke}%
\pgfsetstrokeopacity{0.600000}%
\pgfsetdash{}{0pt}%
\pgfpathmoveto{\pgfqpoint{3.678347in}{0.385000in}}%
\pgfpathlineto{\pgfqpoint{3.689876in}{0.385000in}}%
\pgfpathlineto{\pgfqpoint{3.689876in}{0.826705in}}%
\pgfpathlineto{\pgfqpoint{3.678347in}{0.826705in}}%
\pgfpathlineto{\pgfqpoint{3.678347in}{0.385000in}}%
\pgfpathclose%
\pgfusepath{fill}%
\end{pgfscope}%
\begin{pgfscope}%
\pgfpathrectangle{\pgfqpoint{0.750000in}{0.385000in}}{\pgfqpoint{4.650000in}{2.695000in}}%
\pgfusepath{clip}%
\pgfsetbuttcap%
\pgfsetmiterjoin%
\definecolor{currentfill}{rgb}{0.000000,0.500000,0.000000}%
\pgfsetfillcolor{currentfill}%
\pgfsetfillopacity{0.600000}%
\pgfsetlinewidth{0.000000pt}%
\definecolor{currentstroke}{rgb}{0.000000,0.000000,0.000000}%
\pgfsetstrokecolor{currentstroke}%
\pgfsetstrokeopacity{0.600000}%
\pgfsetdash{}{0pt}%
\pgfpathmoveto{\pgfqpoint{3.689876in}{0.385000in}}%
\pgfpathlineto{\pgfqpoint{3.701405in}{0.385000in}}%
\pgfpathlineto{\pgfqpoint{3.701405in}{0.731202in}}%
\pgfpathlineto{\pgfqpoint{3.689876in}{0.731202in}}%
\pgfpathlineto{\pgfqpoint{3.689876in}{0.385000in}}%
\pgfpathclose%
\pgfusepath{fill}%
\end{pgfscope}%
\begin{pgfscope}%
\pgfpathrectangle{\pgfqpoint{0.750000in}{0.385000in}}{\pgfqpoint{4.650000in}{2.695000in}}%
\pgfusepath{clip}%
\pgfsetbuttcap%
\pgfsetmiterjoin%
\definecolor{currentfill}{rgb}{0.000000,0.500000,0.000000}%
\pgfsetfillcolor{currentfill}%
\pgfsetfillopacity{0.600000}%
\pgfsetlinewidth{0.000000pt}%
\definecolor{currentstroke}{rgb}{0.000000,0.000000,0.000000}%
\pgfsetstrokecolor{currentstroke}%
\pgfsetstrokeopacity{0.600000}%
\pgfsetdash{}{0pt}%
\pgfpathmoveto{\pgfqpoint{3.701405in}{0.385000in}}%
\pgfpathlineto{\pgfqpoint{3.712934in}{0.385000in}}%
\pgfpathlineto{\pgfqpoint{3.712934in}{0.910271in}}%
\pgfpathlineto{\pgfqpoint{3.701405in}{0.910271in}}%
\pgfpathlineto{\pgfqpoint{3.701405in}{0.385000in}}%
\pgfpathclose%
\pgfusepath{fill}%
\end{pgfscope}%
\begin{pgfscope}%
\pgfpathrectangle{\pgfqpoint{0.750000in}{0.385000in}}{\pgfqpoint{4.650000in}{2.695000in}}%
\pgfusepath{clip}%
\pgfsetbuttcap%
\pgfsetmiterjoin%
\definecolor{currentfill}{rgb}{0.000000,0.500000,0.000000}%
\pgfsetfillcolor{currentfill}%
\pgfsetfillopacity{0.600000}%
\pgfsetlinewidth{0.000000pt}%
\definecolor{currentstroke}{rgb}{0.000000,0.000000,0.000000}%
\pgfsetstrokecolor{currentstroke}%
\pgfsetstrokeopacity{0.600000}%
\pgfsetdash{}{0pt}%
\pgfpathmoveto{\pgfqpoint{3.712934in}{0.385000in}}%
\pgfpathlineto{\pgfqpoint{3.724463in}{0.385000in}}%
\pgfpathlineto{\pgfqpoint{3.724463in}{0.778953in}}%
\pgfpathlineto{\pgfqpoint{3.712934in}{0.778953in}}%
\pgfpathlineto{\pgfqpoint{3.712934in}{0.385000in}}%
\pgfpathclose%
\pgfusepath{fill}%
\end{pgfscope}%
\begin{pgfscope}%
\pgfpathrectangle{\pgfqpoint{0.750000in}{0.385000in}}{\pgfqpoint{4.650000in}{2.695000in}}%
\pgfusepath{clip}%
\pgfsetbuttcap%
\pgfsetmiterjoin%
\definecolor{currentfill}{rgb}{0.000000,0.500000,0.000000}%
\pgfsetfillcolor{currentfill}%
\pgfsetfillopacity{0.600000}%
\pgfsetlinewidth{0.000000pt}%
\definecolor{currentstroke}{rgb}{0.000000,0.000000,0.000000}%
\pgfsetstrokecolor{currentstroke}%
\pgfsetstrokeopacity{0.600000}%
\pgfsetdash{}{0pt}%
\pgfpathmoveto{\pgfqpoint{3.724463in}{0.385000in}}%
\pgfpathlineto{\pgfqpoint{3.735992in}{0.385000in}}%
\pgfpathlineto{\pgfqpoint{3.735992in}{0.934147in}}%
\pgfpathlineto{\pgfqpoint{3.724463in}{0.934147in}}%
\pgfpathlineto{\pgfqpoint{3.724463in}{0.385000in}}%
\pgfpathclose%
\pgfusepath{fill}%
\end{pgfscope}%
\begin{pgfscope}%
\pgfpathrectangle{\pgfqpoint{0.750000in}{0.385000in}}{\pgfqpoint{4.650000in}{2.695000in}}%
\pgfusepath{clip}%
\pgfsetbuttcap%
\pgfsetmiterjoin%
\definecolor{currentfill}{rgb}{0.000000,0.500000,0.000000}%
\pgfsetfillcolor{currentfill}%
\pgfsetfillopacity{0.600000}%
\pgfsetlinewidth{0.000000pt}%
\definecolor{currentstroke}{rgb}{0.000000,0.000000,0.000000}%
\pgfsetstrokecolor{currentstroke}%
\pgfsetstrokeopacity{0.600000}%
\pgfsetdash{}{0pt}%
\pgfpathmoveto{\pgfqpoint{3.735992in}{0.385000in}}%
\pgfpathlineto{\pgfqpoint{3.747521in}{0.385000in}}%
\pgfpathlineto{\pgfqpoint{3.747521in}{0.862519in}}%
\pgfpathlineto{\pgfqpoint{3.735992in}{0.862519in}}%
\pgfpathlineto{\pgfqpoint{3.735992in}{0.385000in}}%
\pgfpathclose%
\pgfusepath{fill}%
\end{pgfscope}%
\begin{pgfscope}%
\pgfpathrectangle{\pgfqpoint{0.750000in}{0.385000in}}{\pgfqpoint{4.650000in}{2.695000in}}%
\pgfusepath{clip}%
\pgfsetbuttcap%
\pgfsetmiterjoin%
\definecolor{currentfill}{rgb}{0.000000,0.500000,0.000000}%
\pgfsetfillcolor{currentfill}%
\pgfsetfillopacity{0.600000}%
\pgfsetlinewidth{0.000000pt}%
\definecolor{currentstroke}{rgb}{0.000000,0.000000,0.000000}%
\pgfsetstrokecolor{currentstroke}%
\pgfsetstrokeopacity{0.600000}%
\pgfsetdash{}{0pt}%
\pgfpathmoveto{\pgfqpoint{3.747521in}{0.385000in}}%
\pgfpathlineto{\pgfqpoint{3.759050in}{0.385000in}}%
\pgfpathlineto{\pgfqpoint{3.759050in}{0.767016in}}%
\pgfpathlineto{\pgfqpoint{3.747521in}{0.767016in}}%
\pgfpathlineto{\pgfqpoint{3.747521in}{0.385000in}}%
\pgfpathclose%
\pgfusepath{fill}%
\end{pgfscope}%
\begin{pgfscope}%
\pgfpathrectangle{\pgfqpoint{0.750000in}{0.385000in}}{\pgfqpoint{4.650000in}{2.695000in}}%
\pgfusepath{clip}%
\pgfsetbuttcap%
\pgfsetmiterjoin%
\definecolor{currentfill}{rgb}{0.000000,0.500000,0.000000}%
\pgfsetfillcolor{currentfill}%
\pgfsetfillopacity{0.600000}%
\pgfsetlinewidth{0.000000pt}%
\definecolor{currentstroke}{rgb}{0.000000,0.000000,0.000000}%
\pgfsetstrokecolor{currentstroke}%
\pgfsetstrokeopacity{0.600000}%
\pgfsetdash{}{0pt}%
\pgfpathmoveto{\pgfqpoint{3.759050in}{0.385000in}}%
\pgfpathlineto{\pgfqpoint{3.770579in}{0.385000in}}%
\pgfpathlineto{\pgfqpoint{3.770579in}{0.695388in}}%
\pgfpathlineto{\pgfqpoint{3.759050in}{0.695388in}}%
\pgfpathlineto{\pgfqpoint{3.759050in}{0.385000in}}%
\pgfpathclose%
\pgfusepath{fill}%
\end{pgfscope}%
\begin{pgfscope}%
\pgfpathrectangle{\pgfqpoint{0.750000in}{0.385000in}}{\pgfqpoint{4.650000in}{2.695000in}}%
\pgfusepath{clip}%
\pgfsetbuttcap%
\pgfsetmiterjoin%
\definecolor{currentfill}{rgb}{0.000000,0.500000,0.000000}%
\pgfsetfillcolor{currentfill}%
\pgfsetfillopacity{0.600000}%
\pgfsetlinewidth{0.000000pt}%
\definecolor{currentstroke}{rgb}{0.000000,0.000000,0.000000}%
\pgfsetstrokecolor{currentstroke}%
\pgfsetstrokeopacity{0.600000}%
\pgfsetdash{}{0pt}%
\pgfpathmoveto{\pgfqpoint{3.770579in}{0.385000in}}%
\pgfpathlineto{\pgfqpoint{3.782107in}{0.385000in}}%
\pgfpathlineto{\pgfqpoint{3.782107in}{0.743140in}}%
\pgfpathlineto{\pgfqpoint{3.770579in}{0.743140in}}%
\pgfpathlineto{\pgfqpoint{3.770579in}{0.385000in}}%
\pgfpathclose%
\pgfusepath{fill}%
\end{pgfscope}%
\begin{pgfscope}%
\pgfpathrectangle{\pgfqpoint{0.750000in}{0.385000in}}{\pgfqpoint{4.650000in}{2.695000in}}%
\pgfusepath{clip}%
\pgfsetbuttcap%
\pgfsetmiterjoin%
\definecolor{currentfill}{rgb}{0.000000,0.500000,0.000000}%
\pgfsetfillcolor{currentfill}%
\pgfsetfillopacity{0.600000}%
\pgfsetlinewidth{0.000000pt}%
\definecolor{currentstroke}{rgb}{0.000000,0.000000,0.000000}%
\pgfsetstrokecolor{currentstroke}%
\pgfsetstrokeopacity{0.600000}%
\pgfsetdash{}{0pt}%
\pgfpathmoveto{\pgfqpoint{3.782107in}{0.385000in}}%
\pgfpathlineto{\pgfqpoint{3.793636in}{0.385000in}}%
\pgfpathlineto{\pgfqpoint{3.793636in}{0.898333in}}%
\pgfpathlineto{\pgfqpoint{3.782107in}{0.898333in}}%
\pgfpathlineto{\pgfqpoint{3.782107in}{0.385000in}}%
\pgfpathclose%
\pgfusepath{fill}%
\end{pgfscope}%
\begin{pgfscope}%
\pgfpathrectangle{\pgfqpoint{0.750000in}{0.385000in}}{\pgfqpoint{4.650000in}{2.695000in}}%
\pgfusepath{clip}%
\pgfsetbuttcap%
\pgfsetmiterjoin%
\definecolor{currentfill}{rgb}{0.000000,0.500000,0.000000}%
\pgfsetfillcolor{currentfill}%
\pgfsetfillopacity{0.600000}%
\pgfsetlinewidth{0.000000pt}%
\definecolor{currentstroke}{rgb}{0.000000,0.000000,0.000000}%
\pgfsetstrokecolor{currentstroke}%
\pgfsetstrokeopacity{0.600000}%
\pgfsetdash{}{0pt}%
\pgfpathmoveto{\pgfqpoint{3.793636in}{0.385000in}}%
\pgfpathlineto{\pgfqpoint{3.805165in}{0.385000in}}%
\pgfpathlineto{\pgfqpoint{3.805165in}{0.886395in}}%
\pgfpathlineto{\pgfqpoint{3.793636in}{0.886395in}}%
\pgfpathlineto{\pgfqpoint{3.793636in}{0.385000in}}%
\pgfpathclose%
\pgfusepath{fill}%
\end{pgfscope}%
\begin{pgfscope}%
\pgfpathrectangle{\pgfqpoint{0.750000in}{0.385000in}}{\pgfqpoint{4.650000in}{2.695000in}}%
\pgfusepath{clip}%
\pgfsetbuttcap%
\pgfsetmiterjoin%
\definecolor{currentfill}{rgb}{0.000000,0.500000,0.000000}%
\pgfsetfillcolor{currentfill}%
\pgfsetfillopacity{0.600000}%
\pgfsetlinewidth{0.000000pt}%
\definecolor{currentstroke}{rgb}{0.000000,0.000000,0.000000}%
\pgfsetstrokecolor{currentstroke}%
\pgfsetstrokeopacity{0.600000}%
\pgfsetdash{}{0pt}%
\pgfpathmoveto{\pgfqpoint{3.805165in}{0.385000in}}%
\pgfpathlineto{\pgfqpoint{3.816694in}{0.385000in}}%
\pgfpathlineto{\pgfqpoint{3.816694in}{0.695388in}}%
\pgfpathlineto{\pgfqpoint{3.805165in}{0.695388in}}%
\pgfpathlineto{\pgfqpoint{3.805165in}{0.385000in}}%
\pgfpathclose%
\pgfusepath{fill}%
\end{pgfscope}%
\begin{pgfscope}%
\pgfpathrectangle{\pgfqpoint{0.750000in}{0.385000in}}{\pgfqpoint{4.650000in}{2.695000in}}%
\pgfusepath{clip}%
\pgfsetbuttcap%
\pgfsetmiterjoin%
\definecolor{currentfill}{rgb}{0.000000,0.500000,0.000000}%
\pgfsetfillcolor{currentfill}%
\pgfsetfillopacity{0.600000}%
\pgfsetlinewidth{0.000000pt}%
\definecolor{currentstroke}{rgb}{0.000000,0.000000,0.000000}%
\pgfsetstrokecolor{currentstroke}%
\pgfsetstrokeopacity{0.600000}%
\pgfsetdash{}{0pt}%
\pgfpathmoveto{\pgfqpoint{3.816694in}{0.385000in}}%
\pgfpathlineto{\pgfqpoint{3.828223in}{0.385000in}}%
\pgfpathlineto{\pgfqpoint{3.828223in}{0.719264in}}%
\pgfpathlineto{\pgfqpoint{3.816694in}{0.719264in}}%
\pgfpathlineto{\pgfqpoint{3.816694in}{0.385000in}}%
\pgfpathclose%
\pgfusepath{fill}%
\end{pgfscope}%
\begin{pgfscope}%
\pgfpathrectangle{\pgfqpoint{0.750000in}{0.385000in}}{\pgfqpoint{4.650000in}{2.695000in}}%
\pgfusepath{clip}%
\pgfsetbuttcap%
\pgfsetmiterjoin%
\definecolor{currentfill}{rgb}{0.000000,0.500000,0.000000}%
\pgfsetfillcolor{currentfill}%
\pgfsetfillopacity{0.600000}%
\pgfsetlinewidth{0.000000pt}%
\definecolor{currentstroke}{rgb}{0.000000,0.000000,0.000000}%
\pgfsetstrokecolor{currentstroke}%
\pgfsetstrokeopacity{0.600000}%
\pgfsetdash{}{0pt}%
\pgfpathmoveto{\pgfqpoint{3.828223in}{0.385000in}}%
\pgfpathlineto{\pgfqpoint{3.839752in}{0.385000in}}%
\pgfpathlineto{\pgfqpoint{3.839752in}{0.755078in}}%
\pgfpathlineto{\pgfqpoint{3.828223in}{0.755078in}}%
\pgfpathlineto{\pgfqpoint{3.828223in}{0.385000in}}%
\pgfpathclose%
\pgfusepath{fill}%
\end{pgfscope}%
\begin{pgfscope}%
\pgfpathrectangle{\pgfqpoint{0.750000in}{0.385000in}}{\pgfqpoint{4.650000in}{2.695000in}}%
\pgfusepath{clip}%
\pgfsetbuttcap%
\pgfsetmiterjoin%
\definecolor{currentfill}{rgb}{0.000000,0.500000,0.000000}%
\pgfsetfillcolor{currentfill}%
\pgfsetfillopacity{0.600000}%
\pgfsetlinewidth{0.000000pt}%
\definecolor{currentstroke}{rgb}{0.000000,0.000000,0.000000}%
\pgfsetstrokecolor{currentstroke}%
\pgfsetstrokeopacity{0.600000}%
\pgfsetdash{}{0pt}%
\pgfpathmoveto{\pgfqpoint{3.839752in}{0.385000in}}%
\pgfpathlineto{\pgfqpoint{3.851281in}{0.385000in}}%
\pgfpathlineto{\pgfqpoint{3.851281in}{0.743140in}}%
\pgfpathlineto{\pgfqpoint{3.839752in}{0.743140in}}%
\pgfpathlineto{\pgfqpoint{3.839752in}{0.385000in}}%
\pgfpathclose%
\pgfusepath{fill}%
\end{pgfscope}%
\begin{pgfscope}%
\pgfpathrectangle{\pgfqpoint{0.750000in}{0.385000in}}{\pgfqpoint{4.650000in}{2.695000in}}%
\pgfusepath{clip}%
\pgfsetbuttcap%
\pgfsetmiterjoin%
\definecolor{currentfill}{rgb}{0.000000,0.500000,0.000000}%
\pgfsetfillcolor{currentfill}%
\pgfsetfillopacity{0.600000}%
\pgfsetlinewidth{0.000000pt}%
\definecolor{currentstroke}{rgb}{0.000000,0.000000,0.000000}%
\pgfsetstrokecolor{currentstroke}%
\pgfsetstrokeopacity{0.600000}%
\pgfsetdash{}{0pt}%
\pgfpathmoveto{\pgfqpoint{3.851281in}{0.385000in}}%
\pgfpathlineto{\pgfqpoint{3.862810in}{0.385000in}}%
\pgfpathlineto{\pgfqpoint{3.862810in}{0.790891in}}%
\pgfpathlineto{\pgfqpoint{3.851281in}{0.790891in}}%
\pgfpathlineto{\pgfqpoint{3.851281in}{0.385000in}}%
\pgfpathclose%
\pgfusepath{fill}%
\end{pgfscope}%
\begin{pgfscope}%
\pgfpathrectangle{\pgfqpoint{0.750000in}{0.385000in}}{\pgfqpoint{4.650000in}{2.695000in}}%
\pgfusepath{clip}%
\pgfsetbuttcap%
\pgfsetmiterjoin%
\definecolor{currentfill}{rgb}{0.000000,0.500000,0.000000}%
\pgfsetfillcolor{currentfill}%
\pgfsetfillopacity{0.600000}%
\pgfsetlinewidth{0.000000pt}%
\definecolor{currentstroke}{rgb}{0.000000,0.000000,0.000000}%
\pgfsetstrokecolor{currentstroke}%
\pgfsetstrokeopacity{0.600000}%
\pgfsetdash{}{0pt}%
\pgfpathmoveto{\pgfqpoint{3.862810in}{0.385000in}}%
\pgfpathlineto{\pgfqpoint{3.874339in}{0.385000in}}%
\pgfpathlineto{\pgfqpoint{3.874339in}{0.802829in}}%
\pgfpathlineto{\pgfqpoint{3.862810in}{0.802829in}}%
\pgfpathlineto{\pgfqpoint{3.862810in}{0.385000in}}%
\pgfpathclose%
\pgfusepath{fill}%
\end{pgfscope}%
\begin{pgfscope}%
\pgfpathrectangle{\pgfqpoint{0.750000in}{0.385000in}}{\pgfqpoint{4.650000in}{2.695000in}}%
\pgfusepath{clip}%
\pgfsetbuttcap%
\pgfsetmiterjoin%
\definecolor{currentfill}{rgb}{0.000000,0.500000,0.000000}%
\pgfsetfillcolor{currentfill}%
\pgfsetfillopacity{0.600000}%
\pgfsetlinewidth{0.000000pt}%
\definecolor{currentstroke}{rgb}{0.000000,0.000000,0.000000}%
\pgfsetstrokecolor{currentstroke}%
\pgfsetstrokeopacity{0.600000}%
\pgfsetdash{}{0pt}%
\pgfpathmoveto{\pgfqpoint{3.874339in}{0.385000in}}%
\pgfpathlineto{\pgfqpoint{3.885868in}{0.385000in}}%
\pgfpathlineto{\pgfqpoint{3.885868in}{0.707326in}}%
\pgfpathlineto{\pgfqpoint{3.874339in}{0.707326in}}%
\pgfpathlineto{\pgfqpoint{3.874339in}{0.385000in}}%
\pgfpathclose%
\pgfusepath{fill}%
\end{pgfscope}%
\begin{pgfscope}%
\pgfpathrectangle{\pgfqpoint{0.750000in}{0.385000in}}{\pgfqpoint{4.650000in}{2.695000in}}%
\pgfusepath{clip}%
\pgfsetbuttcap%
\pgfsetmiterjoin%
\definecolor{currentfill}{rgb}{0.000000,0.500000,0.000000}%
\pgfsetfillcolor{currentfill}%
\pgfsetfillopacity{0.600000}%
\pgfsetlinewidth{0.000000pt}%
\definecolor{currentstroke}{rgb}{0.000000,0.000000,0.000000}%
\pgfsetstrokecolor{currentstroke}%
\pgfsetstrokeopacity{0.600000}%
\pgfsetdash{}{0pt}%
\pgfpathmoveto{\pgfqpoint{3.885868in}{0.385000in}}%
\pgfpathlineto{\pgfqpoint{3.897397in}{0.385000in}}%
\pgfpathlineto{\pgfqpoint{3.897397in}{0.790891in}}%
\pgfpathlineto{\pgfqpoint{3.885868in}{0.790891in}}%
\pgfpathlineto{\pgfqpoint{3.885868in}{0.385000in}}%
\pgfpathclose%
\pgfusepath{fill}%
\end{pgfscope}%
\begin{pgfscope}%
\pgfpathrectangle{\pgfqpoint{0.750000in}{0.385000in}}{\pgfqpoint{4.650000in}{2.695000in}}%
\pgfusepath{clip}%
\pgfsetbuttcap%
\pgfsetmiterjoin%
\definecolor{currentfill}{rgb}{0.000000,0.500000,0.000000}%
\pgfsetfillcolor{currentfill}%
\pgfsetfillopacity{0.600000}%
\pgfsetlinewidth{0.000000pt}%
\definecolor{currentstroke}{rgb}{0.000000,0.000000,0.000000}%
\pgfsetstrokecolor{currentstroke}%
\pgfsetstrokeopacity{0.600000}%
\pgfsetdash{}{0pt}%
\pgfpathmoveto{\pgfqpoint{3.897397in}{0.385000in}}%
\pgfpathlineto{\pgfqpoint{3.908926in}{0.385000in}}%
\pgfpathlineto{\pgfqpoint{3.908926in}{0.659574in}}%
\pgfpathlineto{\pgfqpoint{3.897397in}{0.659574in}}%
\pgfpathlineto{\pgfqpoint{3.897397in}{0.385000in}}%
\pgfpathclose%
\pgfusepath{fill}%
\end{pgfscope}%
\begin{pgfscope}%
\pgfpathrectangle{\pgfqpoint{0.750000in}{0.385000in}}{\pgfqpoint{4.650000in}{2.695000in}}%
\pgfusepath{clip}%
\pgfsetbuttcap%
\pgfsetmiterjoin%
\definecolor{currentfill}{rgb}{0.000000,0.500000,0.000000}%
\pgfsetfillcolor{currentfill}%
\pgfsetfillopacity{0.600000}%
\pgfsetlinewidth{0.000000pt}%
\definecolor{currentstroke}{rgb}{0.000000,0.000000,0.000000}%
\pgfsetstrokecolor{currentstroke}%
\pgfsetstrokeopacity{0.600000}%
\pgfsetdash{}{0pt}%
\pgfpathmoveto{\pgfqpoint{3.908926in}{0.385000in}}%
\pgfpathlineto{\pgfqpoint{3.920455in}{0.385000in}}%
\pgfpathlineto{\pgfqpoint{3.920455in}{0.767016in}}%
\pgfpathlineto{\pgfqpoint{3.908926in}{0.767016in}}%
\pgfpathlineto{\pgfqpoint{3.908926in}{0.385000in}}%
\pgfpathclose%
\pgfusepath{fill}%
\end{pgfscope}%
\begin{pgfscope}%
\pgfpathrectangle{\pgfqpoint{0.750000in}{0.385000in}}{\pgfqpoint{4.650000in}{2.695000in}}%
\pgfusepath{clip}%
\pgfsetbuttcap%
\pgfsetmiterjoin%
\definecolor{currentfill}{rgb}{0.000000,0.500000,0.000000}%
\pgfsetfillcolor{currentfill}%
\pgfsetfillopacity{0.600000}%
\pgfsetlinewidth{0.000000pt}%
\definecolor{currentstroke}{rgb}{0.000000,0.000000,0.000000}%
\pgfsetstrokecolor{currentstroke}%
\pgfsetstrokeopacity{0.600000}%
\pgfsetdash{}{0pt}%
\pgfpathmoveto{\pgfqpoint{3.920455in}{0.385000in}}%
\pgfpathlineto{\pgfqpoint{3.931983in}{0.385000in}}%
\pgfpathlineto{\pgfqpoint{3.931983in}{0.671512in}}%
\pgfpathlineto{\pgfqpoint{3.920455in}{0.671512in}}%
\pgfpathlineto{\pgfqpoint{3.920455in}{0.385000in}}%
\pgfpathclose%
\pgfusepath{fill}%
\end{pgfscope}%
\begin{pgfscope}%
\pgfpathrectangle{\pgfqpoint{0.750000in}{0.385000in}}{\pgfqpoint{4.650000in}{2.695000in}}%
\pgfusepath{clip}%
\pgfsetbuttcap%
\pgfsetmiterjoin%
\definecolor{currentfill}{rgb}{0.000000,0.500000,0.000000}%
\pgfsetfillcolor{currentfill}%
\pgfsetfillopacity{0.600000}%
\pgfsetlinewidth{0.000000pt}%
\definecolor{currentstroke}{rgb}{0.000000,0.000000,0.000000}%
\pgfsetstrokecolor{currentstroke}%
\pgfsetstrokeopacity{0.600000}%
\pgfsetdash{}{0pt}%
\pgfpathmoveto{\pgfqpoint{3.931983in}{0.385000in}}%
\pgfpathlineto{\pgfqpoint{3.943512in}{0.385000in}}%
\pgfpathlineto{\pgfqpoint{3.943512in}{0.743140in}}%
\pgfpathlineto{\pgfqpoint{3.931983in}{0.743140in}}%
\pgfpathlineto{\pgfqpoint{3.931983in}{0.385000in}}%
\pgfpathclose%
\pgfusepath{fill}%
\end{pgfscope}%
\begin{pgfscope}%
\pgfpathrectangle{\pgfqpoint{0.750000in}{0.385000in}}{\pgfqpoint{4.650000in}{2.695000in}}%
\pgfusepath{clip}%
\pgfsetbuttcap%
\pgfsetmiterjoin%
\definecolor{currentfill}{rgb}{0.000000,0.500000,0.000000}%
\pgfsetfillcolor{currentfill}%
\pgfsetfillopacity{0.600000}%
\pgfsetlinewidth{0.000000pt}%
\definecolor{currentstroke}{rgb}{0.000000,0.000000,0.000000}%
\pgfsetstrokecolor{currentstroke}%
\pgfsetstrokeopacity{0.600000}%
\pgfsetdash{}{0pt}%
\pgfpathmoveto{\pgfqpoint{3.943512in}{0.385000in}}%
\pgfpathlineto{\pgfqpoint{3.955041in}{0.385000in}}%
\pgfpathlineto{\pgfqpoint{3.955041in}{0.695388in}}%
\pgfpathlineto{\pgfqpoint{3.943512in}{0.695388in}}%
\pgfpathlineto{\pgfqpoint{3.943512in}{0.385000in}}%
\pgfpathclose%
\pgfusepath{fill}%
\end{pgfscope}%
\begin{pgfscope}%
\pgfpathrectangle{\pgfqpoint{0.750000in}{0.385000in}}{\pgfqpoint{4.650000in}{2.695000in}}%
\pgfusepath{clip}%
\pgfsetbuttcap%
\pgfsetmiterjoin%
\definecolor{currentfill}{rgb}{0.000000,0.500000,0.000000}%
\pgfsetfillcolor{currentfill}%
\pgfsetfillopacity{0.600000}%
\pgfsetlinewidth{0.000000pt}%
\definecolor{currentstroke}{rgb}{0.000000,0.000000,0.000000}%
\pgfsetstrokecolor{currentstroke}%
\pgfsetstrokeopacity{0.600000}%
\pgfsetdash{}{0pt}%
\pgfpathmoveto{\pgfqpoint{3.955041in}{0.385000in}}%
\pgfpathlineto{\pgfqpoint{3.966570in}{0.385000in}}%
\pgfpathlineto{\pgfqpoint{3.966570in}{0.802829in}}%
\pgfpathlineto{\pgfqpoint{3.955041in}{0.802829in}}%
\pgfpathlineto{\pgfqpoint{3.955041in}{0.385000in}}%
\pgfpathclose%
\pgfusepath{fill}%
\end{pgfscope}%
\begin{pgfscope}%
\pgfpathrectangle{\pgfqpoint{0.750000in}{0.385000in}}{\pgfqpoint{4.650000in}{2.695000in}}%
\pgfusepath{clip}%
\pgfsetbuttcap%
\pgfsetmiterjoin%
\definecolor{currentfill}{rgb}{0.000000,0.500000,0.000000}%
\pgfsetfillcolor{currentfill}%
\pgfsetfillopacity{0.600000}%
\pgfsetlinewidth{0.000000pt}%
\definecolor{currentstroke}{rgb}{0.000000,0.000000,0.000000}%
\pgfsetstrokecolor{currentstroke}%
\pgfsetstrokeopacity{0.600000}%
\pgfsetdash{}{0pt}%
\pgfpathmoveto{\pgfqpoint{3.966570in}{0.385000in}}%
\pgfpathlineto{\pgfqpoint{3.978099in}{0.385000in}}%
\pgfpathlineto{\pgfqpoint{3.978099in}{0.671512in}}%
\pgfpathlineto{\pgfqpoint{3.966570in}{0.671512in}}%
\pgfpathlineto{\pgfqpoint{3.966570in}{0.385000in}}%
\pgfpathclose%
\pgfusepath{fill}%
\end{pgfscope}%
\begin{pgfscope}%
\pgfpathrectangle{\pgfqpoint{0.750000in}{0.385000in}}{\pgfqpoint{4.650000in}{2.695000in}}%
\pgfusepath{clip}%
\pgfsetbuttcap%
\pgfsetmiterjoin%
\definecolor{currentfill}{rgb}{0.000000,0.500000,0.000000}%
\pgfsetfillcolor{currentfill}%
\pgfsetfillopacity{0.600000}%
\pgfsetlinewidth{0.000000pt}%
\definecolor{currentstroke}{rgb}{0.000000,0.000000,0.000000}%
\pgfsetstrokecolor{currentstroke}%
\pgfsetstrokeopacity{0.600000}%
\pgfsetdash{}{0pt}%
\pgfpathmoveto{\pgfqpoint{3.978099in}{0.385000in}}%
\pgfpathlineto{\pgfqpoint{3.989628in}{0.385000in}}%
\pgfpathlineto{\pgfqpoint{3.989628in}{0.767016in}}%
\pgfpathlineto{\pgfqpoint{3.978099in}{0.767016in}}%
\pgfpathlineto{\pgfqpoint{3.978099in}{0.385000in}}%
\pgfpathclose%
\pgfusepath{fill}%
\end{pgfscope}%
\begin{pgfscope}%
\pgfpathrectangle{\pgfqpoint{0.750000in}{0.385000in}}{\pgfqpoint{4.650000in}{2.695000in}}%
\pgfusepath{clip}%
\pgfsetbuttcap%
\pgfsetmiterjoin%
\definecolor{currentfill}{rgb}{0.000000,0.500000,0.000000}%
\pgfsetfillcolor{currentfill}%
\pgfsetfillopacity{0.600000}%
\pgfsetlinewidth{0.000000pt}%
\definecolor{currentstroke}{rgb}{0.000000,0.000000,0.000000}%
\pgfsetstrokecolor{currentstroke}%
\pgfsetstrokeopacity{0.600000}%
\pgfsetdash{}{0pt}%
\pgfpathmoveto{\pgfqpoint{3.989628in}{0.385000in}}%
\pgfpathlineto{\pgfqpoint{4.001157in}{0.385000in}}%
\pgfpathlineto{\pgfqpoint{4.001157in}{0.671512in}}%
\pgfpathlineto{\pgfqpoint{3.989628in}{0.671512in}}%
\pgfpathlineto{\pgfqpoint{3.989628in}{0.385000in}}%
\pgfpathclose%
\pgfusepath{fill}%
\end{pgfscope}%
\begin{pgfscope}%
\pgfpathrectangle{\pgfqpoint{0.750000in}{0.385000in}}{\pgfqpoint{4.650000in}{2.695000in}}%
\pgfusepath{clip}%
\pgfsetbuttcap%
\pgfsetmiterjoin%
\definecolor{currentfill}{rgb}{0.000000,0.500000,0.000000}%
\pgfsetfillcolor{currentfill}%
\pgfsetfillopacity{0.600000}%
\pgfsetlinewidth{0.000000pt}%
\definecolor{currentstroke}{rgb}{0.000000,0.000000,0.000000}%
\pgfsetstrokecolor{currentstroke}%
\pgfsetstrokeopacity{0.600000}%
\pgfsetdash{}{0pt}%
\pgfpathmoveto{\pgfqpoint{4.001157in}{0.385000in}}%
\pgfpathlineto{\pgfqpoint{4.012686in}{0.385000in}}%
\pgfpathlineto{\pgfqpoint{4.012686in}{0.647636in}}%
\pgfpathlineto{\pgfqpoint{4.001157in}{0.647636in}}%
\pgfpathlineto{\pgfqpoint{4.001157in}{0.385000in}}%
\pgfpathclose%
\pgfusepath{fill}%
\end{pgfscope}%
\begin{pgfscope}%
\pgfpathrectangle{\pgfqpoint{0.750000in}{0.385000in}}{\pgfqpoint{4.650000in}{2.695000in}}%
\pgfusepath{clip}%
\pgfsetbuttcap%
\pgfsetmiterjoin%
\definecolor{currentfill}{rgb}{0.000000,0.500000,0.000000}%
\pgfsetfillcolor{currentfill}%
\pgfsetfillopacity{0.600000}%
\pgfsetlinewidth{0.000000pt}%
\definecolor{currentstroke}{rgb}{0.000000,0.000000,0.000000}%
\pgfsetstrokecolor{currentstroke}%
\pgfsetstrokeopacity{0.600000}%
\pgfsetdash{}{0pt}%
\pgfpathmoveto{\pgfqpoint{4.012686in}{0.385000in}}%
\pgfpathlineto{\pgfqpoint{4.024215in}{0.385000in}}%
\pgfpathlineto{\pgfqpoint{4.024215in}{0.564070in}}%
\pgfpathlineto{\pgfqpoint{4.012686in}{0.564070in}}%
\pgfpathlineto{\pgfqpoint{4.012686in}{0.385000in}}%
\pgfpathclose%
\pgfusepath{fill}%
\end{pgfscope}%
\begin{pgfscope}%
\pgfpathrectangle{\pgfqpoint{0.750000in}{0.385000in}}{\pgfqpoint{4.650000in}{2.695000in}}%
\pgfusepath{clip}%
\pgfsetbuttcap%
\pgfsetmiterjoin%
\definecolor{currentfill}{rgb}{0.000000,0.500000,0.000000}%
\pgfsetfillcolor{currentfill}%
\pgfsetfillopacity{0.600000}%
\pgfsetlinewidth{0.000000pt}%
\definecolor{currentstroke}{rgb}{0.000000,0.000000,0.000000}%
\pgfsetstrokecolor{currentstroke}%
\pgfsetstrokeopacity{0.600000}%
\pgfsetdash{}{0pt}%
\pgfpathmoveto{\pgfqpoint{4.024215in}{0.385000in}}%
\pgfpathlineto{\pgfqpoint{4.035744in}{0.385000in}}%
\pgfpathlineto{\pgfqpoint{4.035744in}{0.623760in}}%
\pgfpathlineto{\pgfqpoint{4.024215in}{0.623760in}}%
\pgfpathlineto{\pgfqpoint{4.024215in}{0.385000in}}%
\pgfpathclose%
\pgfusepath{fill}%
\end{pgfscope}%
\begin{pgfscope}%
\pgfpathrectangle{\pgfqpoint{0.750000in}{0.385000in}}{\pgfqpoint{4.650000in}{2.695000in}}%
\pgfusepath{clip}%
\pgfsetbuttcap%
\pgfsetmiterjoin%
\definecolor{currentfill}{rgb}{0.000000,0.500000,0.000000}%
\pgfsetfillcolor{currentfill}%
\pgfsetfillopacity{0.600000}%
\pgfsetlinewidth{0.000000pt}%
\definecolor{currentstroke}{rgb}{0.000000,0.000000,0.000000}%
\pgfsetstrokecolor{currentstroke}%
\pgfsetstrokeopacity{0.600000}%
\pgfsetdash{}{0pt}%
\pgfpathmoveto{\pgfqpoint{4.035744in}{0.385000in}}%
\pgfpathlineto{\pgfqpoint{4.047273in}{0.385000in}}%
\pgfpathlineto{\pgfqpoint{4.047273in}{0.671512in}}%
\pgfpathlineto{\pgfqpoint{4.035744in}{0.671512in}}%
\pgfpathlineto{\pgfqpoint{4.035744in}{0.385000in}}%
\pgfpathclose%
\pgfusepath{fill}%
\end{pgfscope}%
\begin{pgfscope}%
\pgfpathrectangle{\pgfqpoint{0.750000in}{0.385000in}}{\pgfqpoint{4.650000in}{2.695000in}}%
\pgfusepath{clip}%
\pgfsetbuttcap%
\pgfsetmiterjoin%
\definecolor{currentfill}{rgb}{0.000000,0.500000,0.000000}%
\pgfsetfillcolor{currentfill}%
\pgfsetfillopacity{0.600000}%
\pgfsetlinewidth{0.000000pt}%
\definecolor{currentstroke}{rgb}{0.000000,0.000000,0.000000}%
\pgfsetstrokecolor{currentstroke}%
\pgfsetstrokeopacity{0.600000}%
\pgfsetdash{}{0pt}%
\pgfpathmoveto{\pgfqpoint{4.047273in}{0.385000in}}%
\pgfpathlineto{\pgfqpoint{4.058802in}{0.385000in}}%
\pgfpathlineto{\pgfqpoint{4.058802in}{0.552132in}}%
\pgfpathlineto{\pgfqpoint{4.047273in}{0.552132in}}%
\pgfpathlineto{\pgfqpoint{4.047273in}{0.385000in}}%
\pgfpathclose%
\pgfusepath{fill}%
\end{pgfscope}%
\begin{pgfscope}%
\pgfpathrectangle{\pgfqpoint{0.750000in}{0.385000in}}{\pgfqpoint{4.650000in}{2.695000in}}%
\pgfusepath{clip}%
\pgfsetbuttcap%
\pgfsetmiterjoin%
\definecolor{currentfill}{rgb}{0.000000,0.500000,0.000000}%
\pgfsetfillcolor{currentfill}%
\pgfsetfillopacity{0.600000}%
\pgfsetlinewidth{0.000000pt}%
\definecolor{currentstroke}{rgb}{0.000000,0.000000,0.000000}%
\pgfsetstrokecolor{currentstroke}%
\pgfsetstrokeopacity{0.600000}%
\pgfsetdash{}{0pt}%
\pgfpathmoveto{\pgfqpoint{4.058802in}{0.385000in}}%
\pgfpathlineto{\pgfqpoint{4.070331in}{0.385000in}}%
\pgfpathlineto{\pgfqpoint{4.070331in}{0.743140in}}%
\pgfpathlineto{\pgfqpoint{4.058802in}{0.743140in}}%
\pgfpathlineto{\pgfqpoint{4.058802in}{0.385000in}}%
\pgfpathclose%
\pgfusepath{fill}%
\end{pgfscope}%
\begin{pgfscope}%
\pgfpathrectangle{\pgfqpoint{0.750000in}{0.385000in}}{\pgfqpoint{4.650000in}{2.695000in}}%
\pgfusepath{clip}%
\pgfsetbuttcap%
\pgfsetmiterjoin%
\definecolor{currentfill}{rgb}{0.000000,0.500000,0.000000}%
\pgfsetfillcolor{currentfill}%
\pgfsetfillopacity{0.600000}%
\pgfsetlinewidth{0.000000pt}%
\definecolor{currentstroke}{rgb}{0.000000,0.000000,0.000000}%
\pgfsetstrokecolor{currentstroke}%
\pgfsetstrokeopacity{0.600000}%
\pgfsetdash{}{0pt}%
\pgfpathmoveto{\pgfqpoint{4.070331in}{0.385000in}}%
\pgfpathlineto{\pgfqpoint{4.081860in}{0.385000in}}%
\pgfpathlineto{\pgfqpoint{4.081860in}{0.683450in}}%
\pgfpathlineto{\pgfqpoint{4.070331in}{0.683450in}}%
\pgfpathlineto{\pgfqpoint{4.070331in}{0.385000in}}%
\pgfpathclose%
\pgfusepath{fill}%
\end{pgfscope}%
\begin{pgfscope}%
\pgfpathrectangle{\pgfqpoint{0.750000in}{0.385000in}}{\pgfqpoint{4.650000in}{2.695000in}}%
\pgfusepath{clip}%
\pgfsetbuttcap%
\pgfsetmiterjoin%
\definecolor{currentfill}{rgb}{0.000000,0.500000,0.000000}%
\pgfsetfillcolor{currentfill}%
\pgfsetfillopacity{0.600000}%
\pgfsetlinewidth{0.000000pt}%
\definecolor{currentstroke}{rgb}{0.000000,0.000000,0.000000}%
\pgfsetstrokecolor{currentstroke}%
\pgfsetstrokeopacity{0.600000}%
\pgfsetdash{}{0pt}%
\pgfpathmoveto{\pgfqpoint{4.081860in}{0.385000in}}%
\pgfpathlineto{\pgfqpoint{4.093388in}{0.385000in}}%
\pgfpathlineto{\pgfqpoint{4.093388in}{0.576008in}}%
\pgfpathlineto{\pgfqpoint{4.081860in}{0.576008in}}%
\pgfpathlineto{\pgfqpoint{4.081860in}{0.385000in}}%
\pgfpathclose%
\pgfusepath{fill}%
\end{pgfscope}%
\begin{pgfscope}%
\pgfpathrectangle{\pgfqpoint{0.750000in}{0.385000in}}{\pgfqpoint{4.650000in}{2.695000in}}%
\pgfusepath{clip}%
\pgfsetbuttcap%
\pgfsetmiterjoin%
\definecolor{currentfill}{rgb}{0.000000,0.500000,0.000000}%
\pgfsetfillcolor{currentfill}%
\pgfsetfillopacity{0.600000}%
\pgfsetlinewidth{0.000000pt}%
\definecolor{currentstroke}{rgb}{0.000000,0.000000,0.000000}%
\pgfsetstrokecolor{currentstroke}%
\pgfsetstrokeopacity{0.600000}%
\pgfsetdash{}{0pt}%
\pgfpathmoveto{\pgfqpoint{4.093388in}{0.385000in}}%
\pgfpathlineto{\pgfqpoint{4.104917in}{0.385000in}}%
\pgfpathlineto{\pgfqpoint{4.104917in}{0.540194in}}%
\pgfpathlineto{\pgfqpoint{4.093388in}{0.540194in}}%
\pgfpathlineto{\pgfqpoint{4.093388in}{0.385000in}}%
\pgfpathclose%
\pgfusepath{fill}%
\end{pgfscope}%
\begin{pgfscope}%
\pgfpathrectangle{\pgfqpoint{0.750000in}{0.385000in}}{\pgfqpoint{4.650000in}{2.695000in}}%
\pgfusepath{clip}%
\pgfsetbuttcap%
\pgfsetmiterjoin%
\definecolor{currentfill}{rgb}{0.000000,0.500000,0.000000}%
\pgfsetfillcolor{currentfill}%
\pgfsetfillopacity{0.600000}%
\pgfsetlinewidth{0.000000pt}%
\definecolor{currentstroke}{rgb}{0.000000,0.000000,0.000000}%
\pgfsetstrokecolor{currentstroke}%
\pgfsetstrokeopacity{0.600000}%
\pgfsetdash{}{0pt}%
\pgfpathmoveto{\pgfqpoint{4.104917in}{0.385000in}}%
\pgfpathlineto{\pgfqpoint{4.116446in}{0.385000in}}%
\pgfpathlineto{\pgfqpoint{4.116446in}{0.707326in}}%
\pgfpathlineto{\pgfqpoint{4.104917in}{0.707326in}}%
\pgfpathlineto{\pgfqpoint{4.104917in}{0.385000in}}%
\pgfpathclose%
\pgfusepath{fill}%
\end{pgfscope}%
\begin{pgfscope}%
\pgfpathrectangle{\pgfqpoint{0.750000in}{0.385000in}}{\pgfqpoint{4.650000in}{2.695000in}}%
\pgfusepath{clip}%
\pgfsetbuttcap%
\pgfsetmiterjoin%
\definecolor{currentfill}{rgb}{0.000000,0.500000,0.000000}%
\pgfsetfillcolor{currentfill}%
\pgfsetfillopacity{0.600000}%
\pgfsetlinewidth{0.000000pt}%
\definecolor{currentstroke}{rgb}{0.000000,0.000000,0.000000}%
\pgfsetstrokecolor{currentstroke}%
\pgfsetstrokeopacity{0.600000}%
\pgfsetdash{}{0pt}%
\pgfpathmoveto{\pgfqpoint{4.116446in}{0.385000in}}%
\pgfpathlineto{\pgfqpoint{4.127975in}{0.385000in}}%
\pgfpathlineto{\pgfqpoint{4.127975in}{0.564070in}}%
\pgfpathlineto{\pgfqpoint{4.116446in}{0.564070in}}%
\pgfpathlineto{\pgfqpoint{4.116446in}{0.385000in}}%
\pgfpathclose%
\pgfusepath{fill}%
\end{pgfscope}%
\begin{pgfscope}%
\pgfpathrectangle{\pgfqpoint{0.750000in}{0.385000in}}{\pgfqpoint{4.650000in}{2.695000in}}%
\pgfusepath{clip}%
\pgfsetbuttcap%
\pgfsetmiterjoin%
\definecolor{currentfill}{rgb}{0.000000,0.500000,0.000000}%
\pgfsetfillcolor{currentfill}%
\pgfsetfillopacity{0.600000}%
\pgfsetlinewidth{0.000000pt}%
\definecolor{currentstroke}{rgb}{0.000000,0.000000,0.000000}%
\pgfsetstrokecolor{currentstroke}%
\pgfsetstrokeopacity{0.600000}%
\pgfsetdash{}{0pt}%
\pgfpathmoveto{\pgfqpoint{4.127975in}{0.385000in}}%
\pgfpathlineto{\pgfqpoint{4.139504in}{0.385000in}}%
\pgfpathlineto{\pgfqpoint{4.139504in}{0.587946in}}%
\pgfpathlineto{\pgfqpoint{4.127975in}{0.587946in}}%
\pgfpathlineto{\pgfqpoint{4.127975in}{0.385000in}}%
\pgfpathclose%
\pgfusepath{fill}%
\end{pgfscope}%
\begin{pgfscope}%
\pgfpathrectangle{\pgfqpoint{0.750000in}{0.385000in}}{\pgfqpoint{4.650000in}{2.695000in}}%
\pgfusepath{clip}%
\pgfsetbuttcap%
\pgfsetmiterjoin%
\definecolor{currentfill}{rgb}{0.000000,0.500000,0.000000}%
\pgfsetfillcolor{currentfill}%
\pgfsetfillopacity{0.600000}%
\pgfsetlinewidth{0.000000pt}%
\definecolor{currentstroke}{rgb}{0.000000,0.000000,0.000000}%
\pgfsetstrokecolor{currentstroke}%
\pgfsetstrokeopacity{0.600000}%
\pgfsetdash{}{0pt}%
\pgfpathmoveto{\pgfqpoint{4.139504in}{0.385000in}}%
\pgfpathlineto{\pgfqpoint{4.151033in}{0.385000in}}%
\pgfpathlineto{\pgfqpoint{4.151033in}{0.540194in}}%
\pgfpathlineto{\pgfqpoint{4.139504in}{0.540194in}}%
\pgfpathlineto{\pgfqpoint{4.139504in}{0.385000in}}%
\pgfpathclose%
\pgfusepath{fill}%
\end{pgfscope}%
\begin{pgfscope}%
\pgfpathrectangle{\pgfqpoint{0.750000in}{0.385000in}}{\pgfqpoint{4.650000in}{2.695000in}}%
\pgfusepath{clip}%
\pgfsetbuttcap%
\pgfsetmiterjoin%
\definecolor{currentfill}{rgb}{0.000000,0.500000,0.000000}%
\pgfsetfillcolor{currentfill}%
\pgfsetfillopacity{0.600000}%
\pgfsetlinewidth{0.000000pt}%
\definecolor{currentstroke}{rgb}{0.000000,0.000000,0.000000}%
\pgfsetstrokecolor{currentstroke}%
\pgfsetstrokeopacity{0.600000}%
\pgfsetdash{}{0pt}%
\pgfpathmoveto{\pgfqpoint{4.151033in}{0.385000in}}%
\pgfpathlineto{\pgfqpoint{4.162562in}{0.385000in}}%
\pgfpathlineto{\pgfqpoint{4.162562in}{0.528256in}}%
\pgfpathlineto{\pgfqpoint{4.151033in}{0.528256in}}%
\pgfpathlineto{\pgfqpoint{4.151033in}{0.385000in}}%
\pgfpathclose%
\pgfusepath{fill}%
\end{pgfscope}%
\begin{pgfscope}%
\pgfpathrectangle{\pgfqpoint{0.750000in}{0.385000in}}{\pgfqpoint{4.650000in}{2.695000in}}%
\pgfusepath{clip}%
\pgfsetbuttcap%
\pgfsetmiterjoin%
\definecolor{currentfill}{rgb}{0.000000,0.500000,0.000000}%
\pgfsetfillcolor{currentfill}%
\pgfsetfillopacity{0.600000}%
\pgfsetlinewidth{0.000000pt}%
\definecolor{currentstroke}{rgb}{0.000000,0.000000,0.000000}%
\pgfsetstrokecolor{currentstroke}%
\pgfsetstrokeopacity{0.600000}%
\pgfsetdash{}{0pt}%
\pgfpathmoveto{\pgfqpoint{4.162562in}{0.385000in}}%
\pgfpathlineto{\pgfqpoint{4.174091in}{0.385000in}}%
\pgfpathlineto{\pgfqpoint{4.174091in}{0.611822in}}%
\pgfpathlineto{\pgfqpoint{4.162562in}{0.611822in}}%
\pgfpathlineto{\pgfqpoint{4.162562in}{0.385000in}}%
\pgfpathclose%
\pgfusepath{fill}%
\end{pgfscope}%
\begin{pgfscope}%
\pgfpathrectangle{\pgfqpoint{0.750000in}{0.385000in}}{\pgfqpoint{4.650000in}{2.695000in}}%
\pgfusepath{clip}%
\pgfsetbuttcap%
\pgfsetmiterjoin%
\definecolor{currentfill}{rgb}{0.000000,0.500000,0.000000}%
\pgfsetfillcolor{currentfill}%
\pgfsetfillopacity{0.600000}%
\pgfsetlinewidth{0.000000pt}%
\definecolor{currentstroke}{rgb}{0.000000,0.000000,0.000000}%
\pgfsetstrokecolor{currentstroke}%
\pgfsetstrokeopacity{0.600000}%
\pgfsetdash{}{0pt}%
\pgfpathmoveto{\pgfqpoint{4.174091in}{0.385000in}}%
\pgfpathlineto{\pgfqpoint{4.185620in}{0.385000in}}%
\pgfpathlineto{\pgfqpoint{4.185620in}{0.576008in}}%
\pgfpathlineto{\pgfqpoint{4.174091in}{0.576008in}}%
\pgfpathlineto{\pgfqpoint{4.174091in}{0.385000in}}%
\pgfpathclose%
\pgfusepath{fill}%
\end{pgfscope}%
\begin{pgfscope}%
\pgfpathrectangle{\pgfqpoint{0.750000in}{0.385000in}}{\pgfqpoint{4.650000in}{2.695000in}}%
\pgfusepath{clip}%
\pgfsetbuttcap%
\pgfsetmiterjoin%
\definecolor{currentfill}{rgb}{0.000000,0.500000,0.000000}%
\pgfsetfillcolor{currentfill}%
\pgfsetfillopacity{0.600000}%
\pgfsetlinewidth{0.000000pt}%
\definecolor{currentstroke}{rgb}{0.000000,0.000000,0.000000}%
\pgfsetstrokecolor{currentstroke}%
\pgfsetstrokeopacity{0.600000}%
\pgfsetdash{}{0pt}%
\pgfpathmoveto{\pgfqpoint{4.185620in}{0.385000in}}%
\pgfpathlineto{\pgfqpoint{4.197149in}{0.385000in}}%
\pgfpathlineto{\pgfqpoint{4.197149in}{0.599884in}}%
\pgfpathlineto{\pgfqpoint{4.185620in}{0.599884in}}%
\pgfpathlineto{\pgfqpoint{4.185620in}{0.385000in}}%
\pgfpathclose%
\pgfusepath{fill}%
\end{pgfscope}%
\begin{pgfscope}%
\pgfpathrectangle{\pgfqpoint{0.750000in}{0.385000in}}{\pgfqpoint{4.650000in}{2.695000in}}%
\pgfusepath{clip}%
\pgfsetbuttcap%
\pgfsetmiterjoin%
\definecolor{currentfill}{rgb}{0.000000,0.500000,0.000000}%
\pgfsetfillcolor{currentfill}%
\pgfsetfillopacity{0.600000}%
\pgfsetlinewidth{0.000000pt}%
\definecolor{currentstroke}{rgb}{0.000000,0.000000,0.000000}%
\pgfsetstrokecolor{currentstroke}%
\pgfsetstrokeopacity{0.600000}%
\pgfsetdash{}{0pt}%
\pgfpathmoveto{\pgfqpoint{4.197149in}{0.385000in}}%
\pgfpathlineto{\pgfqpoint{4.208678in}{0.385000in}}%
\pgfpathlineto{\pgfqpoint{4.208678in}{0.587946in}}%
\pgfpathlineto{\pgfqpoint{4.197149in}{0.587946in}}%
\pgfpathlineto{\pgfqpoint{4.197149in}{0.385000in}}%
\pgfpathclose%
\pgfusepath{fill}%
\end{pgfscope}%
\begin{pgfscope}%
\pgfpathrectangle{\pgfqpoint{0.750000in}{0.385000in}}{\pgfqpoint{4.650000in}{2.695000in}}%
\pgfusepath{clip}%
\pgfsetbuttcap%
\pgfsetmiterjoin%
\definecolor{currentfill}{rgb}{0.000000,0.500000,0.000000}%
\pgfsetfillcolor{currentfill}%
\pgfsetfillopacity{0.600000}%
\pgfsetlinewidth{0.000000pt}%
\definecolor{currentstroke}{rgb}{0.000000,0.000000,0.000000}%
\pgfsetstrokecolor{currentstroke}%
\pgfsetstrokeopacity{0.600000}%
\pgfsetdash{}{0pt}%
\pgfpathmoveto{\pgfqpoint{4.208678in}{0.385000in}}%
\pgfpathlineto{\pgfqpoint{4.220207in}{0.385000in}}%
\pgfpathlineto{\pgfqpoint{4.220207in}{0.635698in}}%
\pgfpathlineto{\pgfqpoint{4.208678in}{0.635698in}}%
\pgfpathlineto{\pgfqpoint{4.208678in}{0.385000in}}%
\pgfpathclose%
\pgfusepath{fill}%
\end{pgfscope}%
\begin{pgfscope}%
\pgfpathrectangle{\pgfqpoint{0.750000in}{0.385000in}}{\pgfqpoint{4.650000in}{2.695000in}}%
\pgfusepath{clip}%
\pgfsetbuttcap%
\pgfsetmiterjoin%
\definecolor{currentfill}{rgb}{0.000000,0.500000,0.000000}%
\pgfsetfillcolor{currentfill}%
\pgfsetfillopacity{0.600000}%
\pgfsetlinewidth{0.000000pt}%
\definecolor{currentstroke}{rgb}{0.000000,0.000000,0.000000}%
\pgfsetstrokecolor{currentstroke}%
\pgfsetstrokeopacity{0.600000}%
\pgfsetdash{}{0pt}%
\pgfpathmoveto{\pgfqpoint{4.220207in}{0.385000in}}%
\pgfpathlineto{\pgfqpoint{4.231736in}{0.385000in}}%
\pgfpathlineto{\pgfqpoint{4.231736in}{0.599884in}}%
\pgfpathlineto{\pgfqpoint{4.220207in}{0.599884in}}%
\pgfpathlineto{\pgfqpoint{4.220207in}{0.385000in}}%
\pgfpathclose%
\pgfusepath{fill}%
\end{pgfscope}%
\begin{pgfscope}%
\pgfpathrectangle{\pgfqpoint{0.750000in}{0.385000in}}{\pgfqpoint{4.650000in}{2.695000in}}%
\pgfusepath{clip}%
\pgfsetbuttcap%
\pgfsetmiterjoin%
\definecolor{currentfill}{rgb}{0.000000,0.500000,0.000000}%
\pgfsetfillcolor{currentfill}%
\pgfsetfillopacity{0.600000}%
\pgfsetlinewidth{0.000000pt}%
\definecolor{currentstroke}{rgb}{0.000000,0.000000,0.000000}%
\pgfsetstrokecolor{currentstroke}%
\pgfsetstrokeopacity{0.600000}%
\pgfsetdash{}{0pt}%
\pgfpathmoveto{\pgfqpoint{4.231736in}{0.385000in}}%
\pgfpathlineto{\pgfqpoint{4.243264in}{0.385000in}}%
\pgfpathlineto{\pgfqpoint{4.243264in}{0.647636in}}%
\pgfpathlineto{\pgfqpoint{4.231736in}{0.647636in}}%
\pgfpathlineto{\pgfqpoint{4.231736in}{0.385000in}}%
\pgfpathclose%
\pgfusepath{fill}%
\end{pgfscope}%
\begin{pgfscope}%
\pgfpathrectangle{\pgfqpoint{0.750000in}{0.385000in}}{\pgfqpoint{4.650000in}{2.695000in}}%
\pgfusepath{clip}%
\pgfsetbuttcap%
\pgfsetmiterjoin%
\definecolor{currentfill}{rgb}{0.000000,0.500000,0.000000}%
\pgfsetfillcolor{currentfill}%
\pgfsetfillopacity{0.600000}%
\pgfsetlinewidth{0.000000pt}%
\definecolor{currentstroke}{rgb}{0.000000,0.000000,0.000000}%
\pgfsetstrokecolor{currentstroke}%
\pgfsetstrokeopacity{0.600000}%
\pgfsetdash{}{0pt}%
\pgfpathmoveto{\pgfqpoint{4.243264in}{0.385000in}}%
\pgfpathlineto{\pgfqpoint{4.254793in}{0.385000in}}%
\pgfpathlineto{\pgfqpoint{4.254793in}{0.564070in}}%
\pgfpathlineto{\pgfqpoint{4.243264in}{0.564070in}}%
\pgfpathlineto{\pgfqpoint{4.243264in}{0.385000in}}%
\pgfpathclose%
\pgfusepath{fill}%
\end{pgfscope}%
\begin{pgfscope}%
\pgfpathrectangle{\pgfqpoint{0.750000in}{0.385000in}}{\pgfqpoint{4.650000in}{2.695000in}}%
\pgfusepath{clip}%
\pgfsetbuttcap%
\pgfsetmiterjoin%
\definecolor{currentfill}{rgb}{0.000000,0.500000,0.000000}%
\pgfsetfillcolor{currentfill}%
\pgfsetfillopacity{0.600000}%
\pgfsetlinewidth{0.000000pt}%
\definecolor{currentstroke}{rgb}{0.000000,0.000000,0.000000}%
\pgfsetstrokecolor{currentstroke}%
\pgfsetstrokeopacity{0.600000}%
\pgfsetdash{}{0pt}%
\pgfpathmoveto{\pgfqpoint{4.254793in}{0.385000in}}%
\pgfpathlineto{\pgfqpoint{4.266322in}{0.385000in}}%
\pgfpathlineto{\pgfqpoint{4.266322in}{0.504380in}}%
\pgfpathlineto{\pgfqpoint{4.254793in}{0.504380in}}%
\pgfpathlineto{\pgfqpoint{4.254793in}{0.385000in}}%
\pgfpathclose%
\pgfusepath{fill}%
\end{pgfscope}%
\begin{pgfscope}%
\pgfpathrectangle{\pgfqpoint{0.750000in}{0.385000in}}{\pgfqpoint{4.650000in}{2.695000in}}%
\pgfusepath{clip}%
\pgfsetbuttcap%
\pgfsetmiterjoin%
\definecolor{currentfill}{rgb}{0.000000,0.500000,0.000000}%
\pgfsetfillcolor{currentfill}%
\pgfsetfillopacity{0.600000}%
\pgfsetlinewidth{0.000000pt}%
\definecolor{currentstroke}{rgb}{0.000000,0.000000,0.000000}%
\pgfsetstrokecolor{currentstroke}%
\pgfsetstrokeopacity{0.600000}%
\pgfsetdash{}{0pt}%
\pgfpathmoveto{\pgfqpoint{4.266322in}{0.385000in}}%
\pgfpathlineto{\pgfqpoint{4.277851in}{0.385000in}}%
\pgfpathlineto{\pgfqpoint{4.277851in}{0.516318in}}%
\pgfpathlineto{\pgfqpoint{4.266322in}{0.516318in}}%
\pgfpathlineto{\pgfqpoint{4.266322in}{0.385000in}}%
\pgfpathclose%
\pgfusepath{fill}%
\end{pgfscope}%
\begin{pgfscope}%
\pgfpathrectangle{\pgfqpoint{0.750000in}{0.385000in}}{\pgfqpoint{4.650000in}{2.695000in}}%
\pgfusepath{clip}%
\pgfsetbuttcap%
\pgfsetmiterjoin%
\definecolor{currentfill}{rgb}{0.000000,0.500000,0.000000}%
\pgfsetfillcolor{currentfill}%
\pgfsetfillopacity{0.600000}%
\pgfsetlinewidth{0.000000pt}%
\definecolor{currentstroke}{rgb}{0.000000,0.000000,0.000000}%
\pgfsetstrokecolor{currentstroke}%
\pgfsetstrokeopacity{0.600000}%
\pgfsetdash{}{0pt}%
\pgfpathmoveto{\pgfqpoint{4.277851in}{0.385000in}}%
\pgfpathlineto{\pgfqpoint{4.289380in}{0.385000in}}%
\pgfpathlineto{\pgfqpoint{4.289380in}{0.611822in}}%
\pgfpathlineto{\pgfqpoint{4.277851in}{0.611822in}}%
\pgfpathlineto{\pgfqpoint{4.277851in}{0.385000in}}%
\pgfpathclose%
\pgfusepath{fill}%
\end{pgfscope}%
\begin{pgfscope}%
\pgfpathrectangle{\pgfqpoint{0.750000in}{0.385000in}}{\pgfqpoint{4.650000in}{2.695000in}}%
\pgfusepath{clip}%
\pgfsetbuttcap%
\pgfsetmiterjoin%
\definecolor{currentfill}{rgb}{0.000000,0.500000,0.000000}%
\pgfsetfillcolor{currentfill}%
\pgfsetfillopacity{0.600000}%
\pgfsetlinewidth{0.000000pt}%
\definecolor{currentstroke}{rgb}{0.000000,0.000000,0.000000}%
\pgfsetstrokecolor{currentstroke}%
\pgfsetstrokeopacity{0.600000}%
\pgfsetdash{}{0pt}%
\pgfpathmoveto{\pgfqpoint{4.289380in}{0.385000in}}%
\pgfpathlineto{\pgfqpoint{4.300909in}{0.385000in}}%
\pgfpathlineto{\pgfqpoint{4.300909in}{0.552132in}}%
\pgfpathlineto{\pgfqpoint{4.289380in}{0.552132in}}%
\pgfpathlineto{\pgfqpoint{4.289380in}{0.385000in}}%
\pgfpathclose%
\pgfusepath{fill}%
\end{pgfscope}%
\begin{pgfscope}%
\pgfpathrectangle{\pgfqpoint{0.750000in}{0.385000in}}{\pgfqpoint{4.650000in}{2.695000in}}%
\pgfusepath{clip}%
\pgfsetbuttcap%
\pgfsetmiterjoin%
\definecolor{currentfill}{rgb}{0.000000,0.500000,0.000000}%
\pgfsetfillcolor{currentfill}%
\pgfsetfillopacity{0.600000}%
\pgfsetlinewidth{0.000000pt}%
\definecolor{currentstroke}{rgb}{0.000000,0.000000,0.000000}%
\pgfsetstrokecolor{currentstroke}%
\pgfsetstrokeopacity{0.600000}%
\pgfsetdash{}{0pt}%
\pgfpathmoveto{\pgfqpoint{4.300909in}{0.385000in}}%
\pgfpathlineto{\pgfqpoint{4.312438in}{0.385000in}}%
\pgfpathlineto{\pgfqpoint{4.312438in}{0.516318in}}%
\pgfpathlineto{\pgfqpoint{4.300909in}{0.516318in}}%
\pgfpathlineto{\pgfqpoint{4.300909in}{0.385000in}}%
\pgfpathclose%
\pgfusepath{fill}%
\end{pgfscope}%
\begin{pgfscope}%
\pgfpathrectangle{\pgfqpoint{0.750000in}{0.385000in}}{\pgfqpoint{4.650000in}{2.695000in}}%
\pgfusepath{clip}%
\pgfsetbuttcap%
\pgfsetmiterjoin%
\definecolor{currentfill}{rgb}{0.000000,0.500000,0.000000}%
\pgfsetfillcolor{currentfill}%
\pgfsetfillopacity{0.600000}%
\pgfsetlinewidth{0.000000pt}%
\definecolor{currentstroke}{rgb}{0.000000,0.000000,0.000000}%
\pgfsetstrokecolor{currentstroke}%
\pgfsetstrokeopacity{0.600000}%
\pgfsetdash{}{0pt}%
\pgfpathmoveto{\pgfqpoint{4.312438in}{0.385000in}}%
\pgfpathlineto{\pgfqpoint{4.323967in}{0.385000in}}%
\pgfpathlineto{\pgfqpoint{4.323967in}{0.516318in}}%
\pgfpathlineto{\pgfqpoint{4.312438in}{0.516318in}}%
\pgfpathlineto{\pgfqpoint{4.312438in}{0.385000in}}%
\pgfpathclose%
\pgfusepath{fill}%
\end{pgfscope}%
\begin{pgfscope}%
\pgfpathrectangle{\pgfqpoint{0.750000in}{0.385000in}}{\pgfqpoint{4.650000in}{2.695000in}}%
\pgfusepath{clip}%
\pgfsetbuttcap%
\pgfsetmiterjoin%
\definecolor{currentfill}{rgb}{0.000000,0.500000,0.000000}%
\pgfsetfillcolor{currentfill}%
\pgfsetfillopacity{0.600000}%
\pgfsetlinewidth{0.000000pt}%
\definecolor{currentstroke}{rgb}{0.000000,0.000000,0.000000}%
\pgfsetstrokecolor{currentstroke}%
\pgfsetstrokeopacity{0.600000}%
\pgfsetdash{}{0pt}%
\pgfpathmoveto{\pgfqpoint{4.323967in}{0.385000in}}%
\pgfpathlineto{\pgfqpoint{4.335496in}{0.385000in}}%
\pgfpathlineto{\pgfqpoint{4.335496in}{0.564070in}}%
\pgfpathlineto{\pgfqpoint{4.323967in}{0.564070in}}%
\pgfpathlineto{\pgfqpoint{4.323967in}{0.385000in}}%
\pgfpathclose%
\pgfusepath{fill}%
\end{pgfscope}%
\begin{pgfscope}%
\pgfpathrectangle{\pgfqpoint{0.750000in}{0.385000in}}{\pgfqpoint{4.650000in}{2.695000in}}%
\pgfusepath{clip}%
\pgfsetbuttcap%
\pgfsetmiterjoin%
\definecolor{currentfill}{rgb}{0.000000,0.500000,0.000000}%
\pgfsetfillcolor{currentfill}%
\pgfsetfillopacity{0.600000}%
\pgfsetlinewidth{0.000000pt}%
\definecolor{currentstroke}{rgb}{0.000000,0.000000,0.000000}%
\pgfsetstrokecolor{currentstroke}%
\pgfsetstrokeopacity{0.600000}%
\pgfsetdash{}{0pt}%
\pgfpathmoveto{\pgfqpoint{4.335496in}{0.385000in}}%
\pgfpathlineto{\pgfqpoint{4.347025in}{0.385000in}}%
\pgfpathlineto{\pgfqpoint{4.347025in}{0.564070in}}%
\pgfpathlineto{\pgfqpoint{4.335496in}{0.564070in}}%
\pgfpathlineto{\pgfqpoint{4.335496in}{0.385000in}}%
\pgfpathclose%
\pgfusepath{fill}%
\end{pgfscope}%
\begin{pgfscope}%
\pgfpathrectangle{\pgfqpoint{0.750000in}{0.385000in}}{\pgfqpoint{4.650000in}{2.695000in}}%
\pgfusepath{clip}%
\pgfsetbuttcap%
\pgfsetmiterjoin%
\definecolor{currentfill}{rgb}{0.000000,0.500000,0.000000}%
\pgfsetfillcolor{currentfill}%
\pgfsetfillopacity{0.600000}%
\pgfsetlinewidth{0.000000pt}%
\definecolor{currentstroke}{rgb}{0.000000,0.000000,0.000000}%
\pgfsetstrokecolor{currentstroke}%
\pgfsetstrokeopacity{0.600000}%
\pgfsetdash{}{0pt}%
\pgfpathmoveto{\pgfqpoint{4.347025in}{0.385000in}}%
\pgfpathlineto{\pgfqpoint{4.358554in}{0.385000in}}%
\pgfpathlineto{\pgfqpoint{4.358554in}{0.564070in}}%
\pgfpathlineto{\pgfqpoint{4.347025in}{0.564070in}}%
\pgfpathlineto{\pgfqpoint{4.347025in}{0.385000in}}%
\pgfpathclose%
\pgfusepath{fill}%
\end{pgfscope}%
\begin{pgfscope}%
\pgfpathrectangle{\pgfqpoint{0.750000in}{0.385000in}}{\pgfqpoint{4.650000in}{2.695000in}}%
\pgfusepath{clip}%
\pgfsetbuttcap%
\pgfsetmiterjoin%
\definecolor{currentfill}{rgb}{0.000000,0.500000,0.000000}%
\pgfsetfillcolor{currentfill}%
\pgfsetfillopacity{0.600000}%
\pgfsetlinewidth{0.000000pt}%
\definecolor{currentstroke}{rgb}{0.000000,0.000000,0.000000}%
\pgfsetstrokecolor{currentstroke}%
\pgfsetstrokeopacity{0.600000}%
\pgfsetdash{}{0pt}%
\pgfpathmoveto{\pgfqpoint{4.358554in}{0.385000in}}%
\pgfpathlineto{\pgfqpoint{4.370083in}{0.385000in}}%
\pgfpathlineto{\pgfqpoint{4.370083in}{0.468566in}}%
\pgfpathlineto{\pgfqpoint{4.358554in}{0.468566in}}%
\pgfpathlineto{\pgfqpoint{4.358554in}{0.385000in}}%
\pgfpathclose%
\pgfusepath{fill}%
\end{pgfscope}%
\begin{pgfscope}%
\pgfpathrectangle{\pgfqpoint{0.750000in}{0.385000in}}{\pgfqpoint{4.650000in}{2.695000in}}%
\pgfusepath{clip}%
\pgfsetbuttcap%
\pgfsetmiterjoin%
\definecolor{currentfill}{rgb}{0.000000,0.500000,0.000000}%
\pgfsetfillcolor{currentfill}%
\pgfsetfillopacity{0.600000}%
\pgfsetlinewidth{0.000000pt}%
\definecolor{currentstroke}{rgb}{0.000000,0.000000,0.000000}%
\pgfsetstrokecolor{currentstroke}%
\pgfsetstrokeopacity{0.600000}%
\pgfsetdash{}{0pt}%
\pgfpathmoveto{\pgfqpoint{4.370083in}{0.385000in}}%
\pgfpathlineto{\pgfqpoint{4.381612in}{0.385000in}}%
\pgfpathlineto{\pgfqpoint{4.381612in}{0.468566in}}%
\pgfpathlineto{\pgfqpoint{4.370083in}{0.468566in}}%
\pgfpathlineto{\pgfqpoint{4.370083in}{0.385000in}}%
\pgfpathclose%
\pgfusepath{fill}%
\end{pgfscope}%
\begin{pgfscope}%
\pgfpathrectangle{\pgfqpoint{0.750000in}{0.385000in}}{\pgfqpoint{4.650000in}{2.695000in}}%
\pgfusepath{clip}%
\pgfsetbuttcap%
\pgfsetmiterjoin%
\definecolor{currentfill}{rgb}{0.000000,0.500000,0.000000}%
\pgfsetfillcolor{currentfill}%
\pgfsetfillopacity{0.600000}%
\pgfsetlinewidth{0.000000pt}%
\definecolor{currentstroke}{rgb}{0.000000,0.000000,0.000000}%
\pgfsetstrokecolor{currentstroke}%
\pgfsetstrokeopacity{0.600000}%
\pgfsetdash{}{0pt}%
\pgfpathmoveto{\pgfqpoint{4.381612in}{0.385000in}}%
\pgfpathlineto{\pgfqpoint{4.393140in}{0.385000in}}%
\pgfpathlineto{\pgfqpoint{4.393140in}{0.552132in}}%
\pgfpathlineto{\pgfqpoint{4.381612in}{0.552132in}}%
\pgfpathlineto{\pgfqpoint{4.381612in}{0.385000in}}%
\pgfpathclose%
\pgfusepath{fill}%
\end{pgfscope}%
\begin{pgfscope}%
\pgfpathrectangle{\pgfqpoint{0.750000in}{0.385000in}}{\pgfqpoint{4.650000in}{2.695000in}}%
\pgfusepath{clip}%
\pgfsetbuttcap%
\pgfsetmiterjoin%
\definecolor{currentfill}{rgb}{0.000000,0.500000,0.000000}%
\pgfsetfillcolor{currentfill}%
\pgfsetfillopacity{0.600000}%
\pgfsetlinewidth{0.000000pt}%
\definecolor{currentstroke}{rgb}{0.000000,0.000000,0.000000}%
\pgfsetstrokecolor{currentstroke}%
\pgfsetstrokeopacity{0.600000}%
\pgfsetdash{}{0pt}%
\pgfpathmoveto{\pgfqpoint{4.393140in}{0.385000in}}%
\pgfpathlineto{\pgfqpoint{4.404669in}{0.385000in}}%
\pgfpathlineto{\pgfqpoint{4.404669in}{0.528256in}}%
\pgfpathlineto{\pgfqpoint{4.393140in}{0.528256in}}%
\pgfpathlineto{\pgfqpoint{4.393140in}{0.385000in}}%
\pgfpathclose%
\pgfusepath{fill}%
\end{pgfscope}%
\begin{pgfscope}%
\pgfpathrectangle{\pgfqpoint{0.750000in}{0.385000in}}{\pgfqpoint{4.650000in}{2.695000in}}%
\pgfusepath{clip}%
\pgfsetbuttcap%
\pgfsetmiterjoin%
\definecolor{currentfill}{rgb}{0.000000,0.500000,0.000000}%
\pgfsetfillcolor{currentfill}%
\pgfsetfillopacity{0.600000}%
\pgfsetlinewidth{0.000000pt}%
\definecolor{currentstroke}{rgb}{0.000000,0.000000,0.000000}%
\pgfsetstrokecolor{currentstroke}%
\pgfsetstrokeopacity{0.600000}%
\pgfsetdash{}{0pt}%
\pgfpathmoveto{\pgfqpoint{4.404669in}{0.385000in}}%
\pgfpathlineto{\pgfqpoint{4.416198in}{0.385000in}}%
\pgfpathlineto{\pgfqpoint{4.416198in}{0.540194in}}%
\pgfpathlineto{\pgfqpoint{4.404669in}{0.540194in}}%
\pgfpathlineto{\pgfqpoint{4.404669in}{0.385000in}}%
\pgfpathclose%
\pgfusepath{fill}%
\end{pgfscope}%
\begin{pgfscope}%
\pgfpathrectangle{\pgfqpoint{0.750000in}{0.385000in}}{\pgfqpoint{4.650000in}{2.695000in}}%
\pgfusepath{clip}%
\pgfsetbuttcap%
\pgfsetmiterjoin%
\definecolor{currentfill}{rgb}{0.000000,0.500000,0.000000}%
\pgfsetfillcolor{currentfill}%
\pgfsetfillopacity{0.600000}%
\pgfsetlinewidth{0.000000pt}%
\definecolor{currentstroke}{rgb}{0.000000,0.000000,0.000000}%
\pgfsetstrokecolor{currentstroke}%
\pgfsetstrokeopacity{0.600000}%
\pgfsetdash{}{0pt}%
\pgfpathmoveto{\pgfqpoint{4.416198in}{0.385000in}}%
\pgfpathlineto{\pgfqpoint{4.427727in}{0.385000in}}%
\pgfpathlineto{\pgfqpoint{4.427727in}{0.492442in}}%
\pgfpathlineto{\pgfqpoint{4.416198in}{0.492442in}}%
\pgfpathlineto{\pgfqpoint{4.416198in}{0.385000in}}%
\pgfpathclose%
\pgfusepath{fill}%
\end{pgfscope}%
\begin{pgfscope}%
\pgfpathrectangle{\pgfqpoint{0.750000in}{0.385000in}}{\pgfqpoint{4.650000in}{2.695000in}}%
\pgfusepath{clip}%
\pgfsetbuttcap%
\pgfsetmiterjoin%
\definecolor{currentfill}{rgb}{0.000000,0.500000,0.000000}%
\pgfsetfillcolor{currentfill}%
\pgfsetfillopacity{0.600000}%
\pgfsetlinewidth{0.000000pt}%
\definecolor{currentstroke}{rgb}{0.000000,0.000000,0.000000}%
\pgfsetstrokecolor{currentstroke}%
\pgfsetstrokeopacity{0.600000}%
\pgfsetdash{}{0pt}%
\pgfpathmoveto{\pgfqpoint{4.427727in}{0.385000in}}%
\pgfpathlineto{\pgfqpoint{4.439256in}{0.385000in}}%
\pgfpathlineto{\pgfqpoint{4.439256in}{0.528256in}}%
\pgfpathlineto{\pgfqpoint{4.427727in}{0.528256in}}%
\pgfpathlineto{\pgfqpoint{4.427727in}{0.385000in}}%
\pgfpathclose%
\pgfusepath{fill}%
\end{pgfscope}%
\begin{pgfscope}%
\pgfpathrectangle{\pgfqpoint{0.750000in}{0.385000in}}{\pgfqpoint{4.650000in}{2.695000in}}%
\pgfusepath{clip}%
\pgfsetbuttcap%
\pgfsetmiterjoin%
\definecolor{currentfill}{rgb}{0.000000,0.500000,0.000000}%
\pgfsetfillcolor{currentfill}%
\pgfsetfillopacity{0.600000}%
\pgfsetlinewidth{0.000000pt}%
\definecolor{currentstroke}{rgb}{0.000000,0.000000,0.000000}%
\pgfsetstrokecolor{currentstroke}%
\pgfsetstrokeopacity{0.600000}%
\pgfsetdash{}{0pt}%
\pgfpathmoveto{\pgfqpoint{4.439256in}{0.385000in}}%
\pgfpathlineto{\pgfqpoint{4.450785in}{0.385000in}}%
\pgfpathlineto{\pgfqpoint{4.450785in}{0.516318in}}%
\pgfpathlineto{\pgfqpoint{4.439256in}{0.516318in}}%
\pgfpathlineto{\pgfqpoint{4.439256in}{0.385000in}}%
\pgfpathclose%
\pgfusepath{fill}%
\end{pgfscope}%
\begin{pgfscope}%
\pgfpathrectangle{\pgfqpoint{0.750000in}{0.385000in}}{\pgfqpoint{4.650000in}{2.695000in}}%
\pgfusepath{clip}%
\pgfsetbuttcap%
\pgfsetmiterjoin%
\definecolor{currentfill}{rgb}{0.000000,0.500000,0.000000}%
\pgfsetfillcolor{currentfill}%
\pgfsetfillopacity{0.600000}%
\pgfsetlinewidth{0.000000pt}%
\definecolor{currentstroke}{rgb}{0.000000,0.000000,0.000000}%
\pgfsetstrokecolor{currentstroke}%
\pgfsetstrokeopacity{0.600000}%
\pgfsetdash{}{0pt}%
\pgfpathmoveto{\pgfqpoint{4.450785in}{0.385000in}}%
\pgfpathlineto{\pgfqpoint{4.462314in}{0.385000in}}%
\pgfpathlineto{\pgfqpoint{4.462314in}{0.528256in}}%
\pgfpathlineto{\pgfqpoint{4.450785in}{0.528256in}}%
\pgfpathlineto{\pgfqpoint{4.450785in}{0.385000in}}%
\pgfpathclose%
\pgfusepath{fill}%
\end{pgfscope}%
\begin{pgfscope}%
\pgfpathrectangle{\pgfqpoint{0.750000in}{0.385000in}}{\pgfqpoint{4.650000in}{2.695000in}}%
\pgfusepath{clip}%
\pgfsetbuttcap%
\pgfsetmiterjoin%
\definecolor{currentfill}{rgb}{0.000000,0.500000,0.000000}%
\pgfsetfillcolor{currentfill}%
\pgfsetfillopacity{0.600000}%
\pgfsetlinewidth{0.000000pt}%
\definecolor{currentstroke}{rgb}{0.000000,0.000000,0.000000}%
\pgfsetstrokecolor{currentstroke}%
\pgfsetstrokeopacity{0.600000}%
\pgfsetdash{}{0pt}%
\pgfpathmoveto{\pgfqpoint{4.462314in}{0.385000in}}%
\pgfpathlineto{\pgfqpoint{4.473843in}{0.385000in}}%
\pgfpathlineto{\pgfqpoint{4.473843in}{0.528256in}}%
\pgfpathlineto{\pgfqpoint{4.462314in}{0.528256in}}%
\pgfpathlineto{\pgfqpoint{4.462314in}{0.385000in}}%
\pgfpathclose%
\pgfusepath{fill}%
\end{pgfscope}%
\begin{pgfscope}%
\pgfpathrectangle{\pgfqpoint{0.750000in}{0.385000in}}{\pgfqpoint{4.650000in}{2.695000in}}%
\pgfusepath{clip}%
\pgfsetbuttcap%
\pgfsetmiterjoin%
\definecolor{currentfill}{rgb}{0.000000,0.500000,0.000000}%
\pgfsetfillcolor{currentfill}%
\pgfsetfillopacity{0.600000}%
\pgfsetlinewidth{0.000000pt}%
\definecolor{currentstroke}{rgb}{0.000000,0.000000,0.000000}%
\pgfsetstrokecolor{currentstroke}%
\pgfsetstrokeopacity{0.600000}%
\pgfsetdash{}{0pt}%
\pgfpathmoveto{\pgfqpoint{4.473843in}{0.385000in}}%
\pgfpathlineto{\pgfqpoint{4.485372in}{0.385000in}}%
\pgfpathlineto{\pgfqpoint{4.485372in}{0.504380in}}%
\pgfpathlineto{\pgfqpoint{4.473843in}{0.504380in}}%
\pgfpathlineto{\pgfqpoint{4.473843in}{0.385000in}}%
\pgfpathclose%
\pgfusepath{fill}%
\end{pgfscope}%
\begin{pgfscope}%
\pgfpathrectangle{\pgfqpoint{0.750000in}{0.385000in}}{\pgfqpoint{4.650000in}{2.695000in}}%
\pgfusepath{clip}%
\pgfsetbuttcap%
\pgfsetmiterjoin%
\definecolor{currentfill}{rgb}{0.000000,0.500000,0.000000}%
\pgfsetfillcolor{currentfill}%
\pgfsetfillopacity{0.600000}%
\pgfsetlinewidth{0.000000pt}%
\definecolor{currentstroke}{rgb}{0.000000,0.000000,0.000000}%
\pgfsetstrokecolor{currentstroke}%
\pgfsetstrokeopacity{0.600000}%
\pgfsetdash{}{0pt}%
\pgfpathmoveto{\pgfqpoint{4.485372in}{0.385000in}}%
\pgfpathlineto{\pgfqpoint{4.496901in}{0.385000in}}%
\pgfpathlineto{\pgfqpoint{4.496901in}{0.480504in}}%
\pgfpathlineto{\pgfqpoint{4.485372in}{0.480504in}}%
\pgfpathlineto{\pgfqpoint{4.485372in}{0.385000in}}%
\pgfpathclose%
\pgfusepath{fill}%
\end{pgfscope}%
\begin{pgfscope}%
\pgfpathrectangle{\pgfqpoint{0.750000in}{0.385000in}}{\pgfqpoint{4.650000in}{2.695000in}}%
\pgfusepath{clip}%
\pgfsetbuttcap%
\pgfsetmiterjoin%
\definecolor{currentfill}{rgb}{0.000000,0.500000,0.000000}%
\pgfsetfillcolor{currentfill}%
\pgfsetfillopacity{0.600000}%
\pgfsetlinewidth{0.000000pt}%
\definecolor{currentstroke}{rgb}{0.000000,0.000000,0.000000}%
\pgfsetstrokecolor{currentstroke}%
\pgfsetstrokeopacity{0.600000}%
\pgfsetdash{}{0pt}%
\pgfpathmoveto{\pgfqpoint{4.496901in}{0.385000in}}%
\pgfpathlineto{\pgfqpoint{4.508430in}{0.385000in}}%
\pgfpathlineto{\pgfqpoint{4.508430in}{0.587946in}}%
\pgfpathlineto{\pgfqpoint{4.496901in}{0.587946in}}%
\pgfpathlineto{\pgfqpoint{4.496901in}{0.385000in}}%
\pgfpathclose%
\pgfusepath{fill}%
\end{pgfscope}%
\begin{pgfscope}%
\pgfpathrectangle{\pgfqpoint{0.750000in}{0.385000in}}{\pgfqpoint{4.650000in}{2.695000in}}%
\pgfusepath{clip}%
\pgfsetbuttcap%
\pgfsetmiterjoin%
\definecolor{currentfill}{rgb}{0.000000,0.500000,0.000000}%
\pgfsetfillcolor{currentfill}%
\pgfsetfillopacity{0.600000}%
\pgfsetlinewidth{0.000000pt}%
\definecolor{currentstroke}{rgb}{0.000000,0.000000,0.000000}%
\pgfsetstrokecolor{currentstroke}%
\pgfsetstrokeopacity{0.600000}%
\pgfsetdash{}{0pt}%
\pgfpathmoveto{\pgfqpoint{4.508430in}{0.385000in}}%
\pgfpathlineto{\pgfqpoint{4.519959in}{0.385000in}}%
\pgfpathlineto{\pgfqpoint{4.519959in}{0.480504in}}%
\pgfpathlineto{\pgfqpoint{4.508430in}{0.480504in}}%
\pgfpathlineto{\pgfqpoint{4.508430in}{0.385000in}}%
\pgfpathclose%
\pgfusepath{fill}%
\end{pgfscope}%
\begin{pgfscope}%
\pgfpathrectangle{\pgfqpoint{0.750000in}{0.385000in}}{\pgfqpoint{4.650000in}{2.695000in}}%
\pgfusepath{clip}%
\pgfsetbuttcap%
\pgfsetmiterjoin%
\definecolor{currentfill}{rgb}{0.000000,0.500000,0.000000}%
\pgfsetfillcolor{currentfill}%
\pgfsetfillopacity{0.600000}%
\pgfsetlinewidth{0.000000pt}%
\definecolor{currentstroke}{rgb}{0.000000,0.000000,0.000000}%
\pgfsetstrokecolor{currentstroke}%
\pgfsetstrokeopacity{0.600000}%
\pgfsetdash{}{0pt}%
\pgfpathmoveto{\pgfqpoint{4.519959in}{0.385000in}}%
\pgfpathlineto{\pgfqpoint{4.531488in}{0.385000in}}%
\pgfpathlineto{\pgfqpoint{4.531488in}{0.564070in}}%
\pgfpathlineto{\pgfqpoint{4.519959in}{0.564070in}}%
\pgfpathlineto{\pgfqpoint{4.519959in}{0.385000in}}%
\pgfpathclose%
\pgfusepath{fill}%
\end{pgfscope}%
\begin{pgfscope}%
\pgfpathrectangle{\pgfqpoint{0.750000in}{0.385000in}}{\pgfqpoint{4.650000in}{2.695000in}}%
\pgfusepath{clip}%
\pgfsetbuttcap%
\pgfsetmiterjoin%
\definecolor{currentfill}{rgb}{0.000000,0.500000,0.000000}%
\pgfsetfillcolor{currentfill}%
\pgfsetfillopacity{0.600000}%
\pgfsetlinewidth{0.000000pt}%
\definecolor{currentstroke}{rgb}{0.000000,0.000000,0.000000}%
\pgfsetstrokecolor{currentstroke}%
\pgfsetstrokeopacity{0.600000}%
\pgfsetdash{}{0pt}%
\pgfpathmoveto{\pgfqpoint{4.531488in}{0.385000in}}%
\pgfpathlineto{\pgfqpoint{4.543017in}{0.385000in}}%
\pgfpathlineto{\pgfqpoint{4.543017in}{0.540194in}}%
\pgfpathlineto{\pgfqpoint{4.531488in}{0.540194in}}%
\pgfpathlineto{\pgfqpoint{4.531488in}{0.385000in}}%
\pgfpathclose%
\pgfusepath{fill}%
\end{pgfscope}%
\begin{pgfscope}%
\pgfpathrectangle{\pgfqpoint{0.750000in}{0.385000in}}{\pgfqpoint{4.650000in}{2.695000in}}%
\pgfusepath{clip}%
\pgfsetbuttcap%
\pgfsetmiterjoin%
\definecolor{currentfill}{rgb}{0.000000,0.500000,0.000000}%
\pgfsetfillcolor{currentfill}%
\pgfsetfillopacity{0.600000}%
\pgfsetlinewidth{0.000000pt}%
\definecolor{currentstroke}{rgb}{0.000000,0.000000,0.000000}%
\pgfsetstrokecolor{currentstroke}%
\pgfsetstrokeopacity{0.600000}%
\pgfsetdash{}{0pt}%
\pgfpathmoveto{\pgfqpoint{4.543017in}{0.385000in}}%
\pgfpathlineto{\pgfqpoint{4.554545in}{0.385000in}}%
\pgfpathlineto{\pgfqpoint{4.554545in}{0.516318in}}%
\pgfpathlineto{\pgfqpoint{4.543017in}{0.516318in}}%
\pgfpathlineto{\pgfqpoint{4.543017in}{0.385000in}}%
\pgfpathclose%
\pgfusepath{fill}%
\end{pgfscope}%
\begin{pgfscope}%
\pgfpathrectangle{\pgfqpoint{0.750000in}{0.385000in}}{\pgfqpoint{4.650000in}{2.695000in}}%
\pgfusepath{clip}%
\pgfsetbuttcap%
\pgfsetmiterjoin%
\definecolor{currentfill}{rgb}{0.000000,0.500000,0.000000}%
\pgfsetfillcolor{currentfill}%
\pgfsetfillopacity{0.600000}%
\pgfsetlinewidth{0.000000pt}%
\definecolor{currentstroke}{rgb}{0.000000,0.000000,0.000000}%
\pgfsetstrokecolor{currentstroke}%
\pgfsetstrokeopacity{0.600000}%
\pgfsetdash{}{0pt}%
\pgfpathmoveto{\pgfqpoint{4.554545in}{0.385000in}}%
\pgfpathlineto{\pgfqpoint{4.566074in}{0.385000in}}%
\pgfpathlineto{\pgfqpoint{4.566074in}{0.492442in}}%
\pgfpathlineto{\pgfqpoint{4.554545in}{0.492442in}}%
\pgfpathlineto{\pgfqpoint{4.554545in}{0.385000in}}%
\pgfpathclose%
\pgfusepath{fill}%
\end{pgfscope}%
\begin{pgfscope}%
\pgfpathrectangle{\pgfqpoint{0.750000in}{0.385000in}}{\pgfqpoint{4.650000in}{2.695000in}}%
\pgfusepath{clip}%
\pgfsetbuttcap%
\pgfsetmiterjoin%
\definecolor{currentfill}{rgb}{0.000000,0.500000,0.000000}%
\pgfsetfillcolor{currentfill}%
\pgfsetfillopacity{0.600000}%
\pgfsetlinewidth{0.000000pt}%
\definecolor{currentstroke}{rgb}{0.000000,0.000000,0.000000}%
\pgfsetstrokecolor{currentstroke}%
\pgfsetstrokeopacity{0.600000}%
\pgfsetdash{}{0pt}%
\pgfpathmoveto{\pgfqpoint{4.566074in}{0.385000in}}%
\pgfpathlineto{\pgfqpoint{4.577603in}{0.385000in}}%
\pgfpathlineto{\pgfqpoint{4.577603in}{0.444690in}}%
\pgfpathlineto{\pgfqpoint{4.566074in}{0.444690in}}%
\pgfpathlineto{\pgfqpoint{4.566074in}{0.385000in}}%
\pgfpathclose%
\pgfusepath{fill}%
\end{pgfscope}%
\begin{pgfscope}%
\pgfpathrectangle{\pgfqpoint{0.750000in}{0.385000in}}{\pgfqpoint{4.650000in}{2.695000in}}%
\pgfusepath{clip}%
\pgfsetbuttcap%
\pgfsetmiterjoin%
\definecolor{currentfill}{rgb}{0.000000,0.500000,0.000000}%
\pgfsetfillcolor{currentfill}%
\pgfsetfillopacity{0.600000}%
\pgfsetlinewidth{0.000000pt}%
\definecolor{currentstroke}{rgb}{0.000000,0.000000,0.000000}%
\pgfsetstrokecolor{currentstroke}%
\pgfsetstrokeopacity{0.600000}%
\pgfsetdash{}{0pt}%
\pgfpathmoveto{\pgfqpoint{4.577603in}{0.385000in}}%
\pgfpathlineto{\pgfqpoint{4.589132in}{0.385000in}}%
\pgfpathlineto{\pgfqpoint{4.589132in}{0.480504in}}%
\pgfpathlineto{\pgfqpoint{4.577603in}{0.480504in}}%
\pgfpathlineto{\pgfqpoint{4.577603in}{0.385000in}}%
\pgfpathclose%
\pgfusepath{fill}%
\end{pgfscope}%
\begin{pgfscope}%
\pgfpathrectangle{\pgfqpoint{0.750000in}{0.385000in}}{\pgfqpoint{4.650000in}{2.695000in}}%
\pgfusepath{clip}%
\pgfsetbuttcap%
\pgfsetmiterjoin%
\definecolor{currentfill}{rgb}{0.000000,0.500000,0.000000}%
\pgfsetfillcolor{currentfill}%
\pgfsetfillopacity{0.600000}%
\pgfsetlinewidth{0.000000pt}%
\definecolor{currentstroke}{rgb}{0.000000,0.000000,0.000000}%
\pgfsetstrokecolor{currentstroke}%
\pgfsetstrokeopacity{0.600000}%
\pgfsetdash{}{0pt}%
\pgfpathmoveto{\pgfqpoint{4.589132in}{0.385000in}}%
\pgfpathlineto{\pgfqpoint{4.600661in}{0.385000in}}%
\pgfpathlineto{\pgfqpoint{4.600661in}{0.432752in}}%
\pgfpathlineto{\pgfqpoint{4.589132in}{0.432752in}}%
\pgfpathlineto{\pgfqpoint{4.589132in}{0.385000in}}%
\pgfpathclose%
\pgfusepath{fill}%
\end{pgfscope}%
\begin{pgfscope}%
\pgfpathrectangle{\pgfqpoint{0.750000in}{0.385000in}}{\pgfqpoint{4.650000in}{2.695000in}}%
\pgfusepath{clip}%
\pgfsetbuttcap%
\pgfsetmiterjoin%
\definecolor{currentfill}{rgb}{0.000000,0.500000,0.000000}%
\pgfsetfillcolor{currentfill}%
\pgfsetfillopacity{0.600000}%
\pgfsetlinewidth{0.000000pt}%
\definecolor{currentstroke}{rgb}{0.000000,0.000000,0.000000}%
\pgfsetstrokecolor{currentstroke}%
\pgfsetstrokeopacity{0.600000}%
\pgfsetdash{}{0pt}%
\pgfpathmoveto{\pgfqpoint{4.600661in}{0.385000in}}%
\pgfpathlineto{\pgfqpoint{4.612190in}{0.385000in}}%
\pgfpathlineto{\pgfqpoint{4.612190in}{0.528256in}}%
\pgfpathlineto{\pgfqpoint{4.600661in}{0.528256in}}%
\pgfpathlineto{\pgfqpoint{4.600661in}{0.385000in}}%
\pgfpathclose%
\pgfusepath{fill}%
\end{pgfscope}%
\begin{pgfscope}%
\pgfpathrectangle{\pgfqpoint{0.750000in}{0.385000in}}{\pgfqpoint{4.650000in}{2.695000in}}%
\pgfusepath{clip}%
\pgfsetbuttcap%
\pgfsetmiterjoin%
\definecolor{currentfill}{rgb}{0.000000,0.500000,0.000000}%
\pgfsetfillcolor{currentfill}%
\pgfsetfillopacity{0.600000}%
\pgfsetlinewidth{0.000000pt}%
\definecolor{currentstroke}{rgb}{0.000000,0.000000,0.000000}%
\pgfsetstrokecolor{currentstroke}%
\pgfsetstrokeopacity{0.600000}%
\pgfsetdash{}{0pt}%
\pgfpathmoveto{\pgfqpoint{4.612190in}{0.385000in}}%
\pgfpathlineto{\pgfqpoint{4.623719in}{0.385000in}}%
\pgfpathlineto{\pgfqpoint{4.623719in}{0.516318in}}%
\pgfpathlineto{\pgfqpoint{4.612190in}{0.516318in}}%
\pgfpathlineto{\pgfqpoint{4.612190in}{0.385000in}}%
\pgfpathclose%
\pgfusepath{fill}%
\end{pgfscope}%
\begin{pgfscope}%
\pgfpathrectangle{\pgfqpoint{0.750000in}{0.385000in}}{\pgfqpoint{4.650000in}{2.695000in}}%
\pgfusepath{clip}%
\pgfsetbuttcap%
\pgfsetmiterjoin%
\definecolor{currentfill}{rgb}{0.000000,0.500000,0.000000}%
\pgfsetfillcolor{currentfill}%
\pgfsetfillopacity{0.600000}%
\pgfsetlinewidth{0.000000pt}%
\definecolor{currentstroke}{rgb}{0.000000,0.000000,0.000000}%
\pgfsetstrokecolor{currentstroke}%
\pgfsetstrokeopacity{0.600000}%
\pgfsetdash{}{0pt}%
\pgfpathmoveto{\pgfqpoint{4.623719in}{0.385000in}}%
\pgfpathlineto{\pgfqpoint{4.635248in}{0.385000in}}%
\pgfpathlineto{\pgfqpoint{4.635248in}{0.480504in}}%
\pgfpathlineto{\pgfqpoint{4.623719in}{0.480504in}}%
\pgfpathlineto{\pgfqpoint{4.623719in}{0.385000in}}%
\pgfpathclose%
\pgfusepath{fill}%
\end{pgfscope}%
\begin{pgfscope}%
\pgfpathrectangle{\pgfqpoint{0.750000in}{0.385000in}}{\pgfqpoint{4.650000in}{2.695000in}}%
\pgfusepath{clip}%
\pgfsetbuttcap%
\pgfsetmiterjoin%
\definecolor{currentfill}{rgb}{0.000000,0.500000,0.000000}%
\pgfsetfillcolor{currentfill}%
\pgfsetfillopacity{0.600000}%
\pgfsetlinewidth{0.000000pt}%
\definecolor{currentstroke}{rgb}{0.000000,0.000000,0.000000}%
\pgfsetstrokecolor{currentstroke}%
\pgfsetstrokeopacity{0.600000}%
\pgfsetdash{}{0pt}%
\pgfpathmoveto{\pgfqpoint{4.635248in}{0.385000in}}%
\pgfpathlineto{\pgfqpoint{4.646777in}{0.385000in}}%
\pgfpathlineto{\pgfqpoint{4.646777in}{0.408876in}}%
\pgfpathlineto{\pgfqpoint{4.635248in}{0.408876in}}%
\pgfpathlineto{\pgfqpoint{4.635248in}{0.385000in}}%
\pgfpathclose%
\pgfusepath{fill}%
\end{pgfscope}%
\begin{pgfscope}%
\pgfpathrectangle{\pgfqpoint{0.750000in}{0.385000in}}{\pgfqpoint{4.650000in}{2.695000in}}%
\pgfusepath{clip}%
\pgfsetbuttcap%
\pgfsetmiterjoin%
\definecolor{currentfill}{rgb}{0.000000,0.500000,0.000000}%
\pgfsetfillcolor{currentfill}%
\pgfsetfillopacity{0.600000}%
\pgfsetlinewidth{0.000000pt}%
\definecolor{currentstroke}{rgb}{0.000000,0.000000,0.000000}%
\pgfsetstrokecolor{currentstroke}%
\pgfsetstrokeopacity{0.600000}%
\pgfsetdash{}{0pt}%
\pgfpathmoveto{\pgfqpoint{4.646777in}{0.385000in}}%
\pgfpathlineto{\pgfqpoint{4.658306in}{0.385000in}}%
\pgfpathlineto{\pgfqpoint{4.658306in}{0.480504in}}%
\pgfpathlineto{\pgfqpoint{4.646777in}{0.480504in}}%
\pgfpathlineto{\pgfqpoint{4.646777in}{0.385000in}}%
\pgfpathclose%
\pgfusepath{fill}%
\end{pgfscope}%
\begin{pgfscope}%
\pgfpathrectangle{\pgfqpoint{0.750000in}{0.385000in}}{\pgfqpoint{4.650000in}{2.695000in}}%
\pgfusepath{clip}%
\pgfsetbuttcap%
\pgfsetmiterjoin%
\definecolor{currentfill}{rgb}{0.000000,0.500000,0.000000}%
\pgfsetfillcolor{currentfill}%
\pgfsetfillopacity{0.600000}%
\pgfsetlinewidth{0.000000pt}%
\definecolor{currentstroke}{rgb}{0.000000,0.000000,0.000000}%
\pgfsetstrokecolor{currentstroke}%
\pgfsetstrokeopacity{0.600000}%
\pgfsetdash{}{0pt}%
\pgfpathmoveto{\pgfqpoint{4.658306in}{0.385000in}}%
\pgfpathlineto{\pgfqpoint{4.669835in}{0.385000in}}%
\pgfpathlineto{\pgfqpoint{4.669835in}{0.456628in}}%
\pgfpathlineto{\pgfqpoint{4.658306in}{0.456628in}}%
\pgfpathlineto{\pgfqpoint{4.658306in}{0.385000in}}%
\pgfpathclose%
\pgfusepath{fill}%
\end{pgfscope}%
\begin{pgfscope}%
\pgfpathrectangle{\pgfqpoint{0.750000in}{0.385000in}}{\pgfqpoint{4.650000in}{2.695000in}}%
\pgfusepath{clip}%
\pgfsetbuttcap%
\pgfsetmiterjoin%
\definecolor{currentfill}{rgb}{0.000000,0.500000,0.000000}%
\pgfsetfillcolor{currentfill}%
\pgfsetfillopacity{0.600000}%
\pgfsetlinewidth{0.000000pt}%
\definecolor{currentstroke}{rgb}{0.000000,0.000000,0.000000}%
\pgfsetstrokecolor{currentstroke}%
\pgfsetstrokeopacity{0.600000}%
\pgfsetdash{}{0pt}%
\pgfpathmoveto{\pgfqpoint{4.669835in}{0.385000in}}%
\pgfpathlineto{\pgfqpoint{4.681364in}{0.385000in}}%
\pgfpathlineto{\pgfqpoint{4.681364in}{0.516318in}}%
\pgfpathlineto{\pgfqpoint{4.669835in}{0.516318in}}%
\pgfpathlineto{\pgfqpoint{4.669835in}{0.385000in}}%
\pgfpathclose%
\pgfusepath{fill}%
\end{pgfscope}%
\begin{pgfscope}%
\pgfpathrectangle{\pgfqpoint{0.750000in}{0.385000in}}{\pgfqpoint{4.650000in}{2.695000in}}%
\pgfusepath{clip}%
\pgfsetbuttcap%
\pgfsetmiterjoin%
\definecolor{currentfill}{rgb}{0.000000,0.500000,0.000000}%
\pgfsetfillcolor{currentfill}%
\pgfsetfillopacity{0.600000}%
\pgfsetlinewidth{0.000000pt}%
\definecolor{currentstroke}{rgb}{0.000000,0.000000,0.000000}%
\pgfsetstrokecolor{currentstroke}%
\pgfsetstrokeopacity{0.600000}%
\pgfsetdash{}{0pt}%
\pgfpathmoveto{\pgfqpoint{4.681364in}{0.385000in}}%
\pgfpathlineto{\pgfqpoint{4.692893in}{0.385000in}}%
\pgfpathlineto{\pgfqpoint{4.692893in}{0.504380in}}%
\pgfpathlineto{\pgfqpoint{4.681364in}{0.504380in}}%
\pgfpathlineto{\pgfqpoint{4.681364in}{0.385000in}}%
\pgfpathclose%
\pgfusepath{fill}%
\end{pgfscope}%
\begin{pgfscope}%
\pgfpathrectangle{\pgfqpoint{0.750000in}{0.385000in}}{\pgfqpoint{4.650000in}{2.695000in}}%
\pgfusepath{clip}%
\pgfsetbuttcap%
\pgfsetmiterjoin%
\definecolor{currentfill}{rgb}{0.000000,0.500000,0.000000}%
\pgfsetfillcolor{currentfill}%
\pgfsetfillopacity{0.600000}%
\pgfsetlinewidth{0.000000pt}%
\definecolor{currentstroke}{rgb}{0.000000,0.000000,0.000000}%
\pgfsetstrokecolor{currentstroke}%
\pgfsetstrokeopacity{0.600000}%
\pgfsetdash{}{0pt}%
\pgfpathmoveto{\pgfqpoint{4.692893in}{0.385000in}}%
\pgfpathlineto{\pgfqpoint{4.704421in}{0.385000in}}%
\pgfpathlineto{\pgfqpoint{4.704421in}{0.492442in}}%
\pgfpathlineto{\pgfqpoint{4.692893in}{0.492442in}}%
\pgfpathlineto{\pgfqpoint{4.692893in}{0.385000in}}%
\pgfpathclose%
\pgfusepath{fill}%
\end{pgfscope}%
\begin{pgfscope}%
\pgfpathrectangle{\pgfqpoint{0.750000in}{0.385000in}}{\pgfqpoint{4.650000in}{2.695000in}}%
\pgfusepath{clip}%
\pgfsetbuttcap%
\pgfsetmiterjoin%
\definecolor{currentfill}{rgb}{0.000000,0.500000,0.000000}%
\pgfsetfillcolor{currentfill}%
\pgfsetfillopacity{0.600000}%
\pgfsetlinewidth{0.000000pt}%
\definecolor{currentstroke}{rgb}{0.000000,0.000000,0.000000}%
\pgfsetstrokecolor{currentstroke}%
\pgfsetstrokeopacity{0.600000}%
\pgfsetdash{}{0pt}%
\pgfpathmoveto{\pgfqpoint{4.704421in}{0.385000in}}%
\pgfpathlineto{\pgfqpoint{4.715950in}{0.385000in}}%
\pgfpathlineto{\pgfqpoint{4.715950in}{0.468566in}}%
\pgfpathlineto{\pgfqpoint{4.704421in}{0.468566in}}%
\pgfpathlineto{\pgfqpoint{4.704421in}{0.385000in}}%
\pgfpathclose%
\pgfusepath{fill}%
\end{pgfscope}%
\begin{pgfscope}%
\pgfpathrectangle{\pgfqpoint{0.750000in}{0.385000in}}{\pgfqpoint{4.650000in}{2.695000in}}%
\pgfusepath{clip}%
\pgfsetbuttcap%
\pgfsetmiterjoin%
\definecolor{currentfill}{rgb}{0.000000,0.500000,0.000000}%
\pgfsetfillcolor{currentfill}%
\pgfsetfillopacity{0.600000}%
\pgfsetlinewidth{0.000000pt}%
\definecolor{currentstroke}{rgb}{0.000000,0.000000,0.000000}%
\pgfsetstrokecolor{currentstroke}%
\pgfsetstrokeopacity{0.600000}%
\pgfsetdash{}{0pt}%
\pgfpathmoveto{\pgfqpoint{4.715950in}{0.385000in}}%
\pgfpathlineto{\pgfqpoint{4.727479in}{0.385000in}}%
\pgfpathlineto{\pgfqpoint{4.727479in}{0.444690in}}%
\pgfpathlineto{\pgfqpoint{4.715950in}{0.444690in}}%
\pgfpathlineto{\pgfqpoint{4.715950in}{0.385000in}}%
\pgfpathclose%
\pgfusepath{fill}%
\end{pgfscope}%
\begin{pgfscope}%
\pgfpathrectangle{\pgfqpoint{0.750000in}{0.385000in}}{\pgfqpoint{4.650000in}{2.695000in}}%
\pgfusepath{clip}%
\pgfsetbuttcap%
\pgfsetmiterjoin%
\definecolor{currentfill}{rgb}{0.000000,0.500000,0.000000}%
\pgfsetfillcolor{currentfill}%
\pgfsetfillopacity{0.600000}%
\pgfsetlinewidth{0.000000pt}%
\definecolor{currentstroke}{rgb}{0.000000,0.000000,0.000000}%
\pgfsetstrokecolor{currentstroke}%
\pgfsetstrokeopacity{0.600000}%
\pgfsetdash{}{0pt}%
\pgfpathmoveto{\pgfqpoint{4.727479in}{0.385000in}}%
\pgfpathlineto{\pgfqpoint{4.739008in}{0.385000in}}%
\pgfpathlineto{\pgfqpoint{4.739008in}{0.480504in}}%
\pgfpathlineto{\pgfqpoint{4.727479in}{0.480504in}}%
\pgfpathlineto{\pgfqpoint{4.727479in}{0.385000in}}%
\pgfpathclose%
\pgfusepath{fill}%
\end{pgfscope}%
\begin{pgfscope}%
\pgfpathrectangle{\pgfqpoint{0.750000in}{0.385000in}}{\pgfqpoint{4.650000in}{2.695000in}}%
\pgfusepath{clip}%
\pgfsetbuttcap%
\pgfsetmiterjoin%
\definecolor{currentfill}{rgb}{0.000000,0.500000,0.000000}%
\pgfsetfillcolor{currentfill}%
\pgfsetfillopacity{0.600000}%
\pgfsetlinewidth{0.000000pt}%
\definecolor{currentstroke}{rgb}{0.000000,0.000000,0.000000}%
\pgfsetstrokecolor{currentstroke}%
\pgfsetstrokeopacity{0.600000}%
\pgfsetdash{}{0pt}%
\pgfpathmoveto{\pgfqpoint{4.739008in}{0.385000in}}%
\pgfpathlineto{\pgfqpoint{4.750537in}{0.385000in}}%
\pgfpathlineto{\pgfqpoint{4.750537in}{0.480504in}}%
\pgfpathlineto{\pgfqpoint{4.739008in}{0.480504in}}%
\pgfpathlineto{\pgfqpoint{4.739008in}{0.385000in}}%
\pgfpathclose%
\pgfusepath{fill}%
\end{pgfscope}%
\begin{pgfscope}%
\pgfpathrectangle{\pgfqpoint{0.750000in}{0.385000in}}{\pgfqpoint{4.650000in}{2.695000in}}%
\pgfusepath{clip}%
\pgfsetbuttcap%
\pgfsetmiterjoin%
\definecolor{currentfill}{rgb}{0.000000,0.500000,0.000000}%
\pgfsetfillcolor{currentfill}%
\pgfsetfillopacity{0.600000}%
\pgfsetlinewidth{0.000000pt}%
\definecolor{currentstroke}{rgb}{0.000000,0.000000,0.000000}%
\pgfsetstrokecolor{currentstroke}%
\pgfsetstrokeopacity{0.600000}%
\pgfsetdash{}{0pt}%
\pgfpathmoveto{\pgfqpoint{4.750537in}{0.385000in}}%
\pgfpathlineto{\pgfqpoint{4.762066in}{0.385000in}}%
\pgfpathlineto{\pgfqpoint{4.762066in}{0.468566in}}%
\pgfpathlineto{\pgfqpoint{4.750537in}{0.468566in}}%
\pgfpathlineto{\pgfqpoint{4.750537in}{0.385000in}}%
\pgfpathclose%
\pgfusepath{fill}%
\end{pgfscope}%
\begin{pgfscope}%
\pgfpathrectangle{\pgfqpoint{0.750000in}{0.385000in}}{\pgfqpoint{4.650000in}{2.695000in}}%
\pgfusepath{clip}%
\pgfsetbuttcap%
\pgfsetmiterjoin%
\definecolor{currentfill}{rgb}{0.000000,0.500000,0.000000}%
\pgfsetfillcolor{currentfill}%
\pgfsetfillopacity{0.600000}%
\pgfsetlinewidth{0.000000pt}%
\definecolor{currentstroke}{rgb}{0.000000,0.000000,0.000000}%
\pgfsetstrokecolor{currentstroke}%
\pgfsetstrokeopacity{0.600000}%
\pgfsetdash{}{0pt}%
\pgfpathmoveto{\pgfqpoint{4.762066in}{0.385000in}}%
\pgfpathlineto{\pgfqpoint{4.773595in}{0.385000in}}%
\pgfpathlineto{\pgfqpoint{4.773595in}{0.552132in}}%
\pgfpathlineto{\pgfqpoint{4.762066in}{0.552132in}}%
\pgfpathlineto{\pgfqpoint{4.762066in}{0.385000in}}%
\pgfpathclose%
\pgfusepath{fill}%
\end{pgfscope}%
\begin{pgfscope}%
\pgfpathrectangle{\pgfqpoint{0.750000in}{0.385000in}}{\pgfqpoint{4.650000in}{2.695000in}}%
\pgfusepath{clip}%
\pgfsetbuttcap%
\pgfsetmiterjoin%
\definecolor{currentfill}{rgb}{0.000000,0.500000,0.000000}%
\pgfsetfillcolor{currentfill}%
\pgfsetfillopacity{0.600000}%
\pgfsetlinewidth{0.000000pt}%
\definecolor{currentstroke}{rgb}{0.000000,0.000000,0.000000}%
\pgfsetstrokecolor{currentstroke}%
\pgfsetstrokeopacity{0.600000}%
\pgfsetdash{}{0pt}%
\pgfpathmoveto{\pgfqpoint{4.773595in}{0.385000in}}%
\pgfpathlineto{\pgfqpoint{4.785124in}{0.385000in}}%
\pgfpathlineto{\pgfqpoint{4.785124in}{0.456628in}}%
\pgfpathlineto{\pgfqpoint{4.773595in}{0.456628in}}%
\pgfpathlineto{\pgfqpoint{4.773595in}{0.385000in}}%
\pgfpathclose%
\pgfusepath{fill}%
\end{pgfscope}%
\begin{pgfscope}%
\pgfpathrectangle{\pgfqpoint{0.750000in}{0.385000in}}{\pgfqpoint{4.650000in}{2.695000in}}%
\pgfusepath{clip}%
\pgfsetbuttcap%
\pgfsetmiterjoin%
\definecolor{currentfill}{rgb}{0.000000,0.500000,0.000000}%
\pgfsetfillcolor{currentfill}%
\pgfsetfillopacity{0.600000}%
\pgfsetlinewidth{0.000000pt}%
\definecolor{currentstroke}{rgb}{0.000000,0.000000,0.000000}%
\pgfsetstrokecolor{currentstroke}%
\pgfsetstrokeopacity{0.600000}%
\pgfsetdash{}{0pt}%
\pgfpathmoveto{\pgfqpoint{4.785124in}{0.385000in}}%
\pgfpathlineto{\pgfqpoint{4.796653in}{0.385000in}}%
\pgfpathlineto{\pgfqpoint{4.796653in}{0.432752in}}%
\pgfpathlineto{\pgfqpoint{4.785124in}{0.432752in}}%
\pgfpathlineto{\pgfqpoint{4.785124in}{0.385000in}}%
\pgfpathclose%
\pgfusepath{fill}%
\end{pgfscope}%
\begin{pgfscope}%
\pgfpathrectangle{\pgfqpoint{0.750000in}{0.385000in}}{\pgfqpoint{4.650000in}{2.695000in}}%
\pgfusepath{clip}%
\pgfsetbuttcap%
\pgfsetmiterjoin%
\definecolor{currentfill}{rgb}{0.000000,0.500000,0.000000}%
\pgfsetfillcolor{currentfill}%
\pgfsetfillopacity{0.600000}%
\pgfsetlinewidth{0.000000pt}%
\definecolor{currentstroke}{rgb}{0.000000,0.000000,0.000000}%
\pgfsetstrokecolor{currentstroke}%
\pgfsetstrokeopacity{0.600000}%
\pgfsetdash{}{0pt}%
\pgfpathmoveto{\pgfqpoint{4.796653in}{0.385000in}}%
\pgfpathlineto{\pgfqpoint{4.808182in}{0.385000in}}%
\pgfpathlineto{\pgfqpoint{4.808182in}{0.432752in}}%
\pgfpathlineto{\pgfqpoint{4.796653in}{0.432752in}}%
\pgfpathlineto{\pgfqpoint{4.796653in}{0.385000in}}%
\pgfpathclose%
\pgfusepath{fill}%
\end{pgfscope}%
\begin{pgfscope}%
\pgfpathrectangle{\pgfqpoint{0.750000in}{0.385000in}}{\pgfqpoint{4.650000in}{2.695000in}}%
\pgfusepath{clip}%
\pgfsetbuttcap%
\pgfsetmiterjoin%
\definecolor{currentfill}{rgb}{0.000000,0.500000,0.000000}%
\pgfsetfillcolor{currentfill}%
\pgfsetfillopacity{0.600000}%
\pgfsetlinewidth{0.000000pt}%
\definecolor{currentstroke}{rgb}{0.000000,0.000000,0.000000}%
\pgfsetstrokecolor{currentstroke}%
\pgfsetstrokeopacity{0.600000}%
\pgfsetdash{}{0pt}%
\pgfpathmoveto{\pgfqpoint{4.808182in}{0.385000in}}%
\pgfpathlineto{\pgfqpoint{4.819711in}{0.385000in}}%
\pgfpathlineto{\pgfqpoint{4.819711in}{0.432752in}}%
\pgfpathlineto{\pgfqpoint{4.808182in}{0.432752in}}%
\pgfpathlineto{\pgfqpoint{4.808182in}{0.385000in}}%
\pgfpathclose%
\pgfusepath{fill}%
\end{pgfscope}%
\begin{pgfscope}%
\pgfpathrectangle{\pgfqpoint{0.750000in}{0.385000in}}{\pgfqpoint{4.650000in}{2.695000in}}%
\pgfusepath{clip}%
\pgfsetbuttcap%
\pgfsetmiterjoin%
\definecolor{currentfill}{rgb}{0.000000,0.500000,0.000000}%
\pgfsetfillcolor{currentfill}%
\pgfsetfillopacity{0.600000}%
\pgfsetlinewidth{0.000000pt}%
\definecolor{currentstroke}{rgb}{0.000000,0.000000,0.000000}%
\pgfsetstrokecolor{currentstroke}%
\pgfsetstrokeopacity{0.600000}%
\pgfsetdash{}{0pt}%
\pgfpathmoveto{\pgfqpoint{4.819711in}{0.385000in}}%
\pgfpathlineto{\pgfqpoint{4.831240in}{0.385000in}}%
\pgfpathlineto{\pgfqpoint{4.831240in}{0.456628in}}%
\pgfpathlineto{\pgfqpoint{4.819711in}{0.456628in}}%
\pgfpathlineto{\pgfqpoint{4.819711in}{0.385000in}}%
\pgfpathclose%
\pgfusepath{fill}%
\end{pgfscope}%
\begin{pgfscope}%
\pgfpathrectangle{\pgfqpoint{0.750000in}{0.385000in}}{\pgfqpoint{4.650000in}{2.695000in}}%
\pgfusepath{clip}%
\pgfsetbuttcap%
\pgfsetmiterjoin%
\definecolor{currentfill}{rgb}{0.000000,0.500000,0.000000}%
\pgfsetfillcolor{currentfill}%
\pgfsetfillopacity{0.600000}%
\pgfsetlinewidth{0.000000pt}%
\definecolor{currentstroke}{rgb}{0.000000,0.000000,0.000000}%
\pgfsetstrokecolor{currentstroke}%
\pgfsetstrokeopacity{0.600000}%
\pgfsetdash{}{0pt}%
\pgfpathmoveto{\pgfqpoint{4.831240in}{0.385000in}}%
\pgfpathlineto{\pgfqpoint{4.842769in}{0.385000in}}%
\pgfpathlineto{\pgfqpoint{4.842769in}{0.456628in}}%
\pgfpathlineto{\pgfqpoint{4.831240in}{0.456628in}}%
\pgfpathlineto{\pgfqpoint{4.831240in}{0.385000in}}%
\pgfpathclose%
\pgfusepath{fill}%
\end{pgfscope}%
\begin{pgfscope}%
\pgfpathrectangle{\pgfqpoint{0.750000in}{0.385000in}}{\pgfqpoint{4.650000in}{2.695000in}}%
\pgfusepath{clip}%
\pgfsetbuttcap%
\pgfsetmiterjoin%
\definecolor{currentfill}{rgb}{0.000000,0.500000,0.000000}%
\pgfsetfillcolor{currentfill}%
\pgfsetfillopacity{0.600000}%
\pgfsetlinewidth{0.000000pt}%
\definecolor{currentstroke}{rgb}{0.000000,0.000000,0.000000}%
\pgfsetstrokecolor{currentstroke}%
\pgfsetstrokeopacity{0.600000}%
\pgfsetdash{}{0pt}%
\pgfpathmoveto{\pgfqpoint{4.842769in}{0.385000in}}%
\pgfpathlineto{\pgfqpoint{4.854298in}{0.385000in}}%
\pgfpathlineto{\pgfqpoint{4.854298in}{0.528256in}}%
\pgfpathlineto{\pgfqpoint{4.842769in}{0.528256in}}%
\pgfpathlineto{\pgfqpoint{4.842769in}{0.385000in}}%
\pgfpathclose%
\pgfusepath{fill}%
\end{pgfscope}%
\begin{pgfscope}%
\pgfpathrectangle{\pgfqpoint{0.750000in}{0.385000in}}{\pgfqpoint{4.650000in}{2.695000in}}%
\pgfusepath{clip}%
\pgfsetbuttcap%
\pgfsetmiterjoin%
\definecolor{currentfill}{rgb}{0.000000,0.500000,0.000000}%
\pgfsetfillcolor{currentfill}%
\pgfsetfillopacity{0.600000}%
\pgfsetlinewidth{0.000000pt}%
\definecolor{currentstroke}{rgb}{0.000000,0.000000,0.000000}%
\pgfsetstrokecolor{currentstroke}%
\pgfsetstrokeopacity{0.600000}%
\pgfsetdash{}{0pt}%
\pgfpathmoveto{\pgfqpoint{4.854298in}{0.385000in}}%
\pgfpathlineto{\pgfqpoint{4.865826in}{0.385000in}}%
\pgfpathlineto{\pgfqpoint{4.865826in}{0.480504in}}%
\pgfpathlineto{\pgfqpoint{4.854298in}{0.480504in}}%
\pgfpathlineto{\pgfqpoint{4.854298in}{0.385000in}}%
\pgfpathclose%
\pgfusepath{fill}%
\end{pgfscope}%
\begin{pgfscope}%
\pgfpathrectangle{\pgfqpoint{0.750000in}{0.385000in}}{\pgfqpoint{4.650000in}{2.695000in}}%
\pgfusepath{clip}%
\pgfsetbuttcap%
\pgfsetmiterjoin%
\definecolor{currentfill}{rgb}{0.000000,0.500000,0.000000}%
\pgfsetfillcolor{currentfill}%
\pgfsetfillopacity{0.600000}%
\pgfsetlinewidth{0.000000pt}%
\definecolor{currentstroke}{rgb}{0.000000,0.000000,0.000000}%
\pgfsetstrokecolor{currentstroke}%
\pgfsetstrokeopacity{0.600000}%
\pgfsetdash{}{0pt}%
\pgfpathmoveto{\pgfqpoint{4.865826in}{0.385000in}}%
\pgfpathlineto{\pgfqpoint{4.877355in}{0.385000in}}%
\pgfpathlineto{\pgfqpoint{4.877355in}{0.444690in}}%
\pgfpathlineto{\pgfqpoint{4.865826in}{0.444690in}}%
\pgfpathlineto{\pgfqpoint{4.865826in}{0.385000in}}%
\pgfpathclose%
\pgfusepath{fill}%
\end{pgfscope}%
\begin{pgfscope}%
\pgfpathrectangle{\pgfqpoint{0.750000in}{0.385000in}}{\pgfqpoint{4.650000in}{2.695000in}}%
\pgfusepath{clip}%
\pgfsetbuttcap%
\pgfsetmiterjoin%
\definecolor{currentfill}{rgb}{0.000000,0.500000,0.000000}%
\pgfsetfillcolor{currentfill}%
\pgfsetfillopacity{0.600000}%
\pgfsetlinewidth{0.000000pt}%
\definecolor{currentstroke}{rgb}{0.000000,0.000000,0.000000}%
\pgfsetstrokecolor{currentstroke}%
\pgfsetstrokeopacity{0.600000}%
\pgfsetdash{}{0pt}%
\pgfpathmoveto{\pgfqpoint{4.877355in}{0.385000in}}%
\pgfpathlineto{\pgfqpoint{4.888884in}{0.385000in}}%
\pgfpathlineto{\pgfqpoint{4.888884in}{0.408876in}}%
\pgfpathlineto{\pgfqpoint{4.877355in}{0.408876in}}%
\pgfpathlineto{\pgfqpoint{4.877355in}{0.385000in}}%
\pgfpathclose%
\pgfusepath{fill}%
\end{pgfscope}%
\begin{pgfscope}%
\pgfpathrectangle{\pgfqpoint{0.750000in}{0.385000in}}{\pgfqpoint{4.650000in}{2.695000in}}%
\pgfusepath{clip}%
\pgfsetbuttcap%
\pgfsetmiterjoin%
\definecolor{currentfill}{rgb}{0.000000,0.500000,0.000000}%
\pgfsetfillcolor{currentfill}%
\pgfsetfillopacity{0.600000}%
\pgfsetlinewidth{0.000000pt}%
\definecolor{currentstroke}{rgb}{0.000000,0.000000,0.000000}%
\pgfsetstrokecolor{currentstroke}%
\pgfsetstrokeopacity{0.600000}%
\pgfsetdash{}{0pt}%
\pgfpathmoveto{\pgfqpoint{4.888884in}{0.385000in}}%
\pgfpathlineto{\pgfqpoint{4.900413in}{0.385000in}}%
\pgfpathlineto{\pgfqpoint{4.900413in}{0.420814in}}%
\pgfpathlineto{\pgfqpoint{4.888884in}{0.420814in}}%
\pgfpathlineto{\pgfqpoint{4.888884in}{0.385000in}}%
\pgfpathclose%
\pgfusepath{fill}%
\end{pgfscope}%
\begin{pgfscope}%
\pgfpathrectangle{\pgfqpoint{0.750000in}{0.385000in}}{\pgfqpoint{4.650000in}{2.695000in}}%
\pgfusepath{clip}%
\pgfsetbuttcap%
\pgfsetmiterjoin%
\definecolor{currentfill}{rgb}{0.000000,0.500000,0.000000}%
\pgfsetfillcolor{currentfill}%
\pgfsetfillopacity{0.600000}%
\pgfsetlinewidth{0.000000pt}%
\definecolor{currentstroke}{rgb}{0.000000,0.000000,0.000000}%
\pgfsetstrokecolor{currentstroke}%
\pgfsetstrokeopacity{0.600000}%
\pgfsetdash{}{0pt}%
\pgfpathmoveto{\pgfqpoint{4.900413in}{0.385000in}}%
\pgfpathlineto{\pgfqpoint{4.911942in}{0.385000in}}%
\pgfpathlineto{\pgfqpoint{4.911942in}{0.444690in}}%
\pgfpathlineto{\pgfqpoint{4.900413in}{0.444690in}}%
\pgfpathlineto{\pgfqpoint{4.900413in}{0.385000in}}%
\pgfpathclose%
\pgfusepath{fill}%
\end{pgfscope}%
\begin{pgfscope}%
\pgfpathrectangle{\pgfqpoint{0.750000in}{0.385000in}}{\pgfqpoint{4.650000in}{2.695000in}}%
\pgfusepath{clip}%
\pgfsetbuttcap%
\pgfsetmiterjoin%
\definecolor{currentfill}{rgb}{0.000000,0.500000,0.000000}%
\pgfsetfillcolor{currentfill}%
\pgfsetfillopacity{0.600000}%
\pgfsetlinewidth{0.000000pt}%
\definecolor{currentstroke}{rgb}{0.000000,0.000000,0.000000}%
\pgfsetstrokecolor{currentstroke}%
\pgfsetstrokeopacity{0.600000}%
\pgfsetdash{}{0pt}%
\pgfpathmoveto{\pgfqpoint{4.911942in}{0.385000in}}%
\pgfpathlineto{\pgfqpoint{4.923471in}{0.385000in}}%
\pgfpathlineto{\pgfqpoint{4.923471in}{0.456628in}}%
\pgfpathlineto{\pgfqpoint{4.911942in}{0.456628in}}%
\pgfpathlineto{\pgfqpoint{4.911942in}{0.385000in}}%
\pgfpathclose%
\pgfusepath{fill}%
\end{pgfscope}%
\begin{pgfscope}%
\pgfpathrectangle{\pgfqpoint{0.750000in}{0.385000in}}{\pgfqpoint{4.650000in}{2.695000in}}%
\pgfusepath{clip}%
\pgfsetbuttcap%
\pgfsetmiterjoin%
\definecolor{currentfill}{rgb}{0.000000,0.500000,0.000000}%
\pgfsetfillcolor{currentfill}%
\pgfsetfillopacity{0.600000}%
\pgfsetlinewidth{0.000000pt}%
\definecolor{currentstroke}{rgb}{0.000000,0.000000,0.000000}%
\pgfsetstrokecolor{currentstroke}%
\pgfsetstrokeopacity{0.600000}%
\pgfsetdash{}{0pt}%
\pgfpathmoveto{\pgfqpoint{4.923471in}{0.385000in}}%
\pgfpathlineto{\pgfqpoint{4.935000in}{0.385000in}}%
\pgfpathlineto{\pgfqpoint{4.935000in}{0.444690in}}%
\pgfpathlineto{\pgfqpoint{4.923471in}{0.444690in}}%
\pgfpathlineto{\pgfqpoint{4.923471in}{0.385000in}}%
\pgfpathclose%
\pgfusepath{fill}%
\end{pgfscope}%
\begin{pgfscope}%
\pgfpathrectangle{\pgfqpoint{0.750000in}{0.385000in}}{\pgfqpoint{4.650000in}{2.695000in}}%
\pgfusepath{clip}%
\pgfsetbuttcap%
\pgfsetmiterjoin%
\definecolor{currentfill}{rgb}{0.000000,0.500000,0.000000}%
\pgfsetfillcolor{currentfill}%
\pgfsetfillopacity{0.600000}%
\pgfsetlinewidth{0.000000pt}%
\definecolor{currentstroke}{rgb}{0.000000,0.000000,0.000000}%
\pgfsetstrokecolor{currentstroke}%
\pgfsetstrokeopacity{0.600000}%
\pgfsetdash{}{0pt}%
\pgfpathmoveto{\pgfqpoint{4.935000in}{0.385000in}}%
\pgfpathlineto{\pgfqpoint{4.946529in}{0.385000in}}%
\pgfpathlineto{\pgfqpoint{4.946529in}{0.444690in}}%
\pgfpathlineto{\pgfqpoint{4.935000in}{0.444690in}}%
\pgfpathlineto{\pgfqpoint{4.935000in}{0.385000in}}%
\pgfpathclose%
\pgfusepath{fill}%
\end{pgfscope}%
\begin{pgfscope}%
\pgfpathrectangle{\pgfqpoint{0.750000in}{0.385000in}}{\pgfqpoint{4.650000in}{2.695000in}}%
\pgfusepath{clip}%
\pgfsetbuttcap%
\pgfsetmiterjoin%
\definecolor{currentfill}{rgb}{0.000000,0.500000,0.000000}%
\pgfsetfillcolor{currentfill}%
\pgfsetfillopacity{0.600000}%
\pgfsetlinewidth{0.000000pt}%
\definecolor{currentstroke}{rgb}{0.000000,0.000000,0.000000}%
\pgfsetstrokecolor{currentstroke}%
\pgfsetstrokeopacity{0.600000}%
\pgfsetdash{}{0pt}%
\pgfpathmoveto{\pgfqpoint{4.946529in}{0.385000in}}%
\pgfpathlineto{\pgfqpoint{4.958058in}{0.385000in}}%
\pgfpathlineto{\pgfqpoint{4.958058in}{0.480504in}}%
\pgfpathlineto{\pgfqpoint{4.946529in}{0.480504in}}%
\pgfpathlineto{\pgfqpoint{4.946529in}{0.385000in}}%
\pgfpathclose%
\pgfusepath{fill}%
\end{pgfscope}%
\begin{pgfscope}%
\pgfpathrectangle{\pgfqpoint{0.750000in}{0.385000in}}{\pgfqpoint{4.650000in}{2.695000in}}%
\pgfusepath{clip}%
\pgfsetbuttcap%
\pgfsetmiterjoin%
\definecolor{currentfill}{rgb}{0.000000,0.500000,0.000000}%
\pgfsetfillcolor{currentfill}%
\pgfsetfillopacity{0.600000}%
\pgfsetlinewidth{0.000000pt}%
\definecolor{currentstroke}{rgb}{0.000000,0.000000,0.000000}%
\pgfsetstrokecolor{currentstroke}%
\pgfsetstrokeopacity{0.600000}%
\pgfsetdash{}{0pt}%
\pgfpathmoveto{\pgfqpoint{4.958058in}{0.385000in}}%
\pgfpathlineto{\pgfqpoint{4.969587in}{0.385000in}}%
\pgfpathlineto{\pgfqpoint{4.969587in}{0.420814in}}%
\pgfpathlineto{\pgfqpoint{4.958058in}{0.420814in}}%
\pgfpathlineto{\pgfqpoint{4.958058in}{0.385000in}}%
\pgfpathclose%
\pgfusepath{fill}%
\end{pgfscope}%
\begin{pgfscope}%
\pgfpathrectangle{\pgfqpoint{0.750000in}{0.385000in}}{\pgfqpoint{4.650000in}{2.695000in}}%
\pgfusepath{clip}%
\pgfsetbuttcap%
\pgfsetmiterjoin%
\definecolor{currentfill}{rgb}{0.000000,0.500000,0.000000}%
\pgfsetfillcolor{currentfill}%
\pgfsetfillopacity{0.600000}%
\pgfsetlinewidth{0.000000pt}%
\definecolor{currentstroke}{rgb}{0.000000,0.000000,0.000000}%
\pgfsetstrokecolor{currentstroke}%
\pgfsetstrokeopacity{0.600000}%
\pgfsetdash{}{0pt}%
\pgfpathmoveto{\pgfqpoint{4.969587in}{0.385000in}}%
\pgfpathlineto{\pgfqpoint{4.981116in}{0.385000in}}%
\pgfpathlineto{\pgfqpoint{4.981116in}{0.420814in}}%
\pgfpathlineto{\pgfqpoint{4.969587in}{0.420814in}}%
\pgfpathlineto{\pgfqpoint{4.969587in}{0.385000in}}%
\pgfpathclose%
\pgfusepath{fill}%
\end{pgfscope}%
\begin{pgfscope}%
\pgfpathrectangle{\pgfqpoint{0.750000in}{0.385000in}}{\pgfqpoint{4.650000in}{2.695000in}}%
\pgfusepath{clip}%
\pgfsetbuttcap%
\pgfsetmiterjoin%
\definecolor{currentfill}{rgb}{0.000000,0.500000,0.000000}%
\pgfsetfillcolor{currentfill}%
\pgfsetfillopacity{0.600000}%
\pgfsetlinewidth{0.000000pt}%
\definecolor{currentstroke}{rgb}{0.000000,0.000000,0.000000}%
\pgfsetstrokecolor{currentstroke}%
\pgfsetstrokeopacity{0.600000}%
\pgfsetdash{}{0pt}%
\pgfpathmoveto{\pgfqpoint{4.981116in}{0.385000in}}%
\pgfpathlineto{\pgfqpoint{4.992645in}{0.385000in}}%
\pgfpathlineto{\pgfqpoint{4.992645in}{0.432752in}}%
\pgfpathlineto{\pgfqpoint{4.981116in}{0.432752in}}%
\pgfpathlineto{\pgfqpoint{4.981116in}{0.385000in}}%
\pgfpathclose%
\pgfusepath{fill}%
\end{pgfscope}%
\begin{pgfscope}%
\pgfpathrectangle{\pgfqpoint{0.750000in}{0.385000in}}{\pgfqpoint{4.650000in}{2.695000in}}%
\pgfusepath{clip}%
\pgfsetbuttcap%
\pgfsetmiterjoin%
\definecolor{currentfill}{rgb}{0.000000,0.500000,0.000000}%
\pgfsetfillcolor{currentfill}%
\pgfsetfillopacity{0.600000}%
\pgfsetlinewidth{0.000000pt}%
\definecolor{currentstroke}{rgb}{0.000000,0.000000,0.000000}%
\pgfsetstrokecolor{currentstroke}%
\pgfsetstrokeopacity{0.600000}%
\pgfsetdash{}{0pt}%
\pgfpathmoveto{\pgfqpoint{4.992645in}{0.385000in}}%
\pgfpathlineto{\pgfqpoint{5.004174in}{0.385000in}}%
\pgfpathlineto{\pgfqpoint{5.004174in}{0.432752in}}%
\pgfpathlineto{\pgfqpoint{4.992645in}{0.432752in}}%
\pgfpathlineto{\pgfqpoint{4.992645in}{0.385000in}}%
\pgfpathclose%
\pgfusepath{fill}%
\end{pgfscope}%
\begin{pgfscope}%
\pgfpathrectangle{\pgfqpoint{0.750000in}{0.385000in}}{\pgfqpoint{4.650000in}{2.695000in}}%
\pgfusepath{clip}%
\pgfsetbuttcap%
\pgfsetmiterjoin%
\definecolor{currentfill}{rgb}{0.000000,0.500000,0.000000}%
\pgfsetfillcolor{currentfill}%
\pgfsetfillopacity{0.600000}%
\pgfsetlinewidth{0.000000pt}%
\definecolor{currentstroke}{rgb}{0.000000,0.000000,0.000000}%
\pgfsetstrokecolor{currentstroke}%
\pgfsetstrokeopacity{0.600000}%
\pgfsetdash{}{0pt}%
\pgfpathmoveto{\pgfqpoint{5.004174in}{0.385000in}}%
\pgfpathlineto{\pgfqpoint{5.015702in}{0.385000in}}%
\pgfpathlineto{\pgfqpoint{5.015702in}{0.444690in}}%
\pgfpathlineto{\pgfqpoint{5.004174in}{0.444690in}}%
\pgfpathlineto{\pgfqpoint{5.004174in}{0.385000in}}%
\pgfpathclose%
\pgfusepath{fill}%
\end{pgfscope}%
\begin{pgfscope}%
\pgfpathrectangle{\pgfqpoint{0.750000in}{0.385000in}}{\pgfqpoint{4.650000in}{2.695000in}}%
\pgfusepath{clip}%
\pgfsetbuttcap%
\pgfsetmiterjoin%
\definecolor{currentfill}{rgb}{0.000000,0.500000,0.000000}%
\pgfsetfillcolor{currentfill}%
\pgfsetfillopacity{0.600000}%
\pgfsetlinewidth{0.000000pt}%
\definecolor{currentstroke}{rgb}{0.000000,0.000000,0.000000}%
\pgfsetstrokecolor{currentstroke}%
\pgfsetstrokeopacity{0.600000}%
\pgfsetdash{}{0pt}%
\pgfpathmoveto{\pgfqpoint{5.015702in}{0.385000in}}%
\pgfpathlineto{\pgfqpoint{5.027231in}{0.385000in}}%
\pgfpathlineto{\pgfqpoint{5.027231in}{0.480504in}}%
\pgfpathlineto{\pgfqpoint{5.015702in}{0.480504in}}%
\pgfpathlineto{\pgfqpoint{5.015702in}{0.385000in}}%
\pgfpathclose%
\pgfusepath{fill}%
\end{pgfscope}%
\begin{pgfscope}%
\pgfpathrectangle{\pgfqpoint{0.750000in}{0.385000in}}{\pgfqpoint{4.650000in}{2.695000in}}%
\pgfusepath{clip}%
\pgfsetbuttcap%
\pgfsetmiterjoin%
\definecolor{currentfill}{rgb}{0.000000,0.500000,0.000000}%
\pgfsetfillcolor{currentfill}%
\pgfsetfillopacity{0.600000}%
\pgfsetlinewidth{0.000000pt}%
\definecolor{currentstroke}{rgb}{0.000000,0.000000,0.000000}%
\pgfsetstrokecolor{currentstroke}%
\pgfsetstrokeopacity{0.600000}%
\pgfsetdash{}{0pt}%
\pgfpathmoveto{\pgfqpoint{5.027231in}{0.385000in}}%
\pgfpathlineto{\pgfqpoint{5.038760in}{0.385000in}}%
\pgfpathlineto{\pgfqpoint{5.038760in}{0.480504in}}%
\pgfpathlineto{\pgfqpoint{5.027231in}{0.480504in}}%
\pgfpathlineto{\pgfqpoint{5.027231in}{0.385000in}}%
\pgfpathclose%
\pgfusepath{fill}%
\end{pgfscope}%
\begin{pgfscope}%
\pgfpathrectangle{\pgfqpoint{0.750000in}{0.385000in}}{\pgfqpoint{4.650000in}{2.695000in}}%
\pgfusepath{clip}%
\pgfsetbuttcap%
\pgfsetmiterjoin%
\definecolor{currentfill}{rgb}{0.000000,0.500000,0.000000}%
\pgfsetfillcolor{currentfill}%
\pgfsetfillopacity{0.600000}%
\pgfsetlinewidth{0.000000pt}%
\definecolor{currentstroke}{rgb}{0.000000,0.000000,0.000000}%
\pgfsetstrokecolor{currentstroke}%
\pgfsetstrokeopacity{0.600000}%
\pgfsetdash{}{0pt}%
\pgfpathmoveto{\pgfqpoint{5.038760in}{0.385000in}}%
\pgfpathlineto{\pgfqpoint{5.050289in}{0.385000in}}%
\pgfpathlineto{\pgfqpoint{5.050289in}{0.444690in}}%
\pgfpathlineto{\pgfqpoint{5.038760in}{0.444690in}}%
\pgfpathlineto{\pgfqpoint{5.038760in}{0.385000in}}%
\pgfpathclose%
\pgfusepath{fill}%
\end{pgfscope}%
\begin{pgfscope}%
\pgfpathrectangle{\pgfqpoint{0.750000in}{0.385000in}}{\pgfqpoint{4.650000in}{2.695000in}}%
\pgfusepath{clip}%
\pgfsetbuttcap%
\pgfsetmiterjoin%
\definecolor{currentfill}{rgb}{0.000000,0.500000,0.000000}%
\pgfsetfillcolor{currentfill}%
\pgfsetfillopacity{0.600000}%
\pgfsetlinewidth{0.000000pt}%
\definecolor{currentstroke}{rgb}{0.000000,0.000000,0.000000}%
\pgfsetstrokecolor{currentstroke}%
\pgfsetstrokeopacity{0.600000}%
\pgfsetdash{}{0pt}%
\pgfpathmoveto{\pgfqpoint{5.050289in}{0.385000in}}%
\pgfpathlineto{\pgfqpoint{5.061818in}{0.385000in}}%
\pgfpathlineto{\pgfqpoint{5.061818in}{0.444690in}}%
\pgfpathlineto{\pgfqpoint{5.050289in}{0.444690in}}%
\pgfpathlineto{\pgfqpoint{5.050289in}{0.385000in}}%
\pgfpathclose%
\pgfusepath{fill}%
\end{pgfscope}%
\begin{pgfscope}%
\pgfpathrectangle{\pgfqpoint{0.750000in}{0.385000in}}{\pgfqpoint{4.650000in}{2.695000in}}%
\pgfusepath{clip}%
\pgfsetbuttcap%
\pgfsetmiterjoin%
\definecolor{currentfill}{rgb}{0.000000,0.500000,0.000000}%
\pgfsetfillcolor{currentfill}%
\pgfsetfillopacity{0.600000}%
\pgfsetlinewidth{0.000000pt}%
\definecolor{currentstroke}{rgb}{0.000000,0.000000,0.000000}%
\pgfsetstrokecolor{currentstroke}%
\pgfsetstrokeopacity{0.600000}%
\pgfsetdash{}{0pt}%
\pgfpathmoveto{\pgfqpoint{5.061818in}{0.385000in}}%
\pgfpathlineto{\pgfqpoint{5.073347in}{0.385000in}}%
\pgfpathlineto{\pgfqpoint{5.073347in}{0.516318in}}%
\pgfpathlineto{\pgfqpoint{5.061818in}{0.516318in}}%
\pgfpathlineto{\pgfqpoint{5.061818in}{0.385000in}}%
\pgfpathclose%
\pgfusepath{fill}%
\end{pgfscope}%
\begin{pgfscope}%
\pgfpathrectangle{\pgfqpoint{0.750000in}{0.385000in}}{\pgfqpoint{4.650000in}{2.695000in}}%
\pgfusepath{clip}%
\pgfsetbuttcap%
\pgfsetmiterjoin%
\definecolor{currentfill}{rgb}{0.000000,0.500000,0.000000}%
\pgfsetfillcolor{currentfill}%
\pgfsetfillopacity{0.600000}%
\pgfsetlinewidth{0.000000pt}%
\definecolor{currentstroke}{rgb}{0.000000,0.000000,0.000000}%
\pgfsetstrokecolor{currentstroke}%
\pgfsetstrokeopacity{0.600000}%
\pgfsetdash{}{0pt}%
\pgfpathmoveto{\pgfqpoint{5.073347in}{0.385000in}}%
\pgfpathlineto{\pgfqpoint{5.084876in}{0.385000in}}%
\pgfpathlineto{\pgfqpoint{5.084876in}{0.456628in}}%
\pgfpathlineto{\pgfqpoint{5.073347in}{0.456628in}}%
\pgfpathlineto{\pgfqpoint{5.073347in}{0.385000in}}%
\pgfpathclose%
\pgfusepath{fill}%
\end{pgfscope}%
\begin{pgfscope}%
\pgfpathrectangle{\pgfqpoint{0.750000in}{0.385000in}}{\pgfqpoint{4.650000in}{2.695000in}}%
\pgfusepath{clip}%
\pgfsetbuttcap%
\pgfsetmiterjoin%
\definecolor{currentfill}{rgb}{0.000000,0.500000,0.000000}%
\pgfsetfillcolor{currentfill}%
\pgfsetfillopacity{0.600000}%
\pgfsetlinewidth{0.000000pt}%
\definecolor{currentstroke}{rgb}{0.000000,0.000000,0.000000}%
\pgfsetstrokecolor{currentstroke}%
\pgfsetstrokeopacity{0.600000}%
\pgfsetdash{}{0pt}%
\pgfpathmoveto{\pgfqpoint{5.084876in}{0.385000in}}%
\pgfpathlineto{\pgfqpoint{5.096405in}{0.385000in}}%
\pgfpathlineto{\pgfqpoint{5.096405in}{0.444690in}}%
\pgfpathlineto{\pgfqpoint{5.084876in}{0.444690in}}%
\pgfpathlineto{\pgfqpoint{5.084876in}{0.385000in}}%
\pgfpathclose%
\pgfusepath{fill}%
\end{pgfscope}%
\begin{pgfscope}%
\pgfpathrectangle{\pgfqpoint{0.750000in}{0.385000in}}{\pgfqpoint{4.650000in}{2.695000in}}%
\pgfusepath{clip}%
\pgfsetbuttcap%
\pgfsetmiterjoin%
\definecolor{currentfill}{rgb}{0.000000,0.500000,0.000000}%
\pgfsetfillcolor{currentfill}%
\pgfsetfillopacity{0.600000}%
\pgfsetlinewidth{0.000000pt}%
\definecolor{currentstroke}{rgb}{0.000000,0.000000,0.000000}%
\pgfsetstrokecolor{currentstroke}%
\pgfsetstrokeopacity{0.600000}%
\pgfsetdash{}{0pt}%
\pgfpathmoveto{\pgfqpoint{5.096405in}{0.385000in}}%
\pgfpathlineto{\pgfqpoint{5.107934in}{0.385000in}}%
\pgfpathlineto{\pgfqpoint{5.107934in}{0.444690in}}%
\pgfpathlineto{\pgfqpoint{5.096405in}{0.444690in}}%
\pgfpathlineto{\pgfqpoint{5.096405in}{0.385000in}}%
\pgfpathclose%
\pgfusepath{fill}%
\end{pgfscope}%
\begin{pgfscope}%
\pgfpathrectangle{\pgfqpoint{0.750000in}{0.385000in}}{\pgfqpoint{4.650000in}{2.695000in}}%
\pgfusepath{clip}%
\pgfsetbuttcap%
\pgfsetmiterjoin%
\definecolor{currentfill}{rgb}{0.000000,0.500000,0.000000}%
\pgfsetfillcolor{currentfill}%
\pgfsetfillopacity{0.600000}%
\pgfsetlinewidth{0.000000pt}%
\definecolor{currentstroke}{rgb}{0.000000,0.000000,0.000000}%
\pgfsetstrokecolor{currentstroke}%
\pgfsetstrokeopacity{0.600000}%
\pgfsetdash{}{0pt}%
\pgfpathmoveto{\pgfqpoint{5.107934in}{0.385000in}}%
\pgfpathlineto{\pgfqpoint{5.119463in}{0.385000in}}%
\pgfpathlineto{\pgfqpoint{5.119463in}{0.456628in}}%
\pgfpathlineto{\pgfqpoint{5.107934in}{0.456628in}}%
\pgfpathlineto{\pgfqpoint{5.107934in}{0.385000in}}%
\pgfpathclose%
\pgfusepath{fill}%
\end{pgfscope}%
\begin{pgfscope}%
\pgfpathrectangle{\pgfqpoint{0.750000in}{0.385000in}}{\pgfqpoint{4.650000in}{2.695000in}}%
\pgfusepath{clip}%
\pgfsetbuttcap%
\pgfsetmiterjoin%
\definecolor{currentfill}{rgb}{0.000000,0.500000,0.000000}%
\pgfsetfillcolor{currentfill}%
\pgfsetfillopacity{0.600000}%
\pgfsetlinewidth{0.000000pt}%
\definecolor{currentstroke}{rgb}{0.000000,0.000000,0.000000}%
\pgfsetstrokecolor{currentstroke}%
\pgfsetstrokeopacity{0.600000}%
\pgfsetdash{}{0pt}%
\pgfpathmoveto{\pgfqpoint{5.119463in}{0.385000in}}%
\pgfpathlineto{\pgfqpoint{5.130992in}{0.385000in}}%
\pgfpathlineto{\pgfqpoint{5.130992in}{0.480504in}}%
\pgfpathlineto{\pgfqpoint{5.119463in}{0.480504in}}%
\pgfpathlineto{\pgfqpoint{5.119463in}{0.385000in}}%
\pgfpathclose%
\pgfusepath{fill}%
\end{pgfscope}%
\begin{pgfscope}%
\pgfpathrectangle{\pgfqpoint{0.750000in}{0.385000in}}{\pgfqpoint{4.650000in}{2.695000in}}%
\pgfusepath{clip}%
\pgfsetbuttcap%
\pgfsetmiterjoin%
\definecolor{currentfill}{rgb}{0.000000,0.500000,0.000000}%
\pgfsetfillcolor{currentfill}%
\pgfsetfillopacity{0.600000}%
\pgfsetlinewidth{0.000000pt}%
\definecolor{currentstroke}{rgb}{0.000000,0.000000,0.000000}%
\pgfsetstrokecolor{currentstroke}%
\pgfsetstrokeopacity{0.600000}%
\pgfsetdash{}{0pt}%
\pgfpathmoveto{\pgfqpoint{5.130992in}{0.385000in}}%
\pgfpathlineto{\pgfqpoint{5.142521in}{0.385000in}}%
\pgfpathlineto{\pgfqpoint{5.142521in}{0.456628in}}%
\pgfpathlineto{\pgfqpoint{5.130992in}{0.456628in}}%
\pgfpathlineto{\pgfqpoint{5.130992in}{0.385000in}}%
\pgfpathclose%
\pgfusepath{fill}%
\end{pgfscope}%
\begin{pgfscope}%
\pgfpathrectangle{\pgfqpoint{0.750000in}{0.385000in}}{\pgfqpoint{4.650000in}{2.695000in}}%
\pgfusepath{clip}%
\pgfsetbuttcap%
\pgfsetmiterjoin%
\definecolor{currentfill}{rgb}{0.000000,0.500000,0.000000}%
\pgfsetfillcolor{currentfill}%
\pgfsetfillopacity{0.600000}%
\pgfsetlinewidth{0.000000pt}%
\definecolor{currentstroke}{rgb}{0.000000,0.000000,0.000000}%
\pgfsetstrokecolor{currentstroke}%
\pgfsetstrokeopacity{0.600000}%
\pgfsetdash{}{0pt}%
\pgfpathmoveto{\pgfqpoint{5.142521in}{0.385000in}}%
\pgfpathlineto{\pgfqpoint{5.154050in}{0.385000in}}%
\pgfpathlineto{\pgfqpoint{5.154050in}{0.432752in}}%
\pgfpathlineto{\pgfqpoint{5.142521in}{0.432752in}}%
\pgfpathlineto{\pgfqpoint{5.142521in}{0.385000in}}%
\pgfpathclose%
\pgfusepath{fill}%
\end{pgfscope}%
\begin{pgfscope}%
\pgfpathrectangle{\pgfqpoint{0.750000in}{0.385000in}}{\pgfqpoint{4.650000in}{2.695000in}}%
\pgfusepath{clip}%
\pgfsetbuttcap%
\pgfsetmiterjoin%
\definecolor{currentfill}{rgb}{0.000000,0.500000,0.000000}%
\pgfsetfillcolor{currentfill}%
\pgfsetfillopacity{0.600000}%
\pgfsetlinewidth{0.000000pt}%
\definecolor{currentstroke}{rgb}{0.000000,0.000000,0.000000}%
\pgfsetstrokecolor{currentstroke}%
\pgfsetstrokeopacity{0.600000}%
\pgfsetdash{}{0pt}%
\pgfpathmoveto{\pgfqpoint{5.154050in}{0.385000in}}%
\pgfpathlineto{\pgfqpoint{5.165579in}{0.385000in}}%
\pgfpathlineto{\pgfqpoint{5.165579in}{0.396938in}}%
\pgfpathlineto{\pgfqpoint{5.154050in}{0.396938in}}%
\pgfpathlineto{\pgfqpoint{5.154050in}{0.385000in}}%
\pgfpathclose%
\pgfusepath{fill}%
\end{pgfscope}%
\begin{pgfscope}%
\pgfpathrectangle{\pgfqpoint{0.750000in}{0.385000in}}{\pgfqpoint{4.650000in}{2.695000in}}%
\pgfusepath{clip}%
\pgfsetbuttcap%
\pgfsetmiterjoin%
\definecolor{currentfill}{rgb}{0.000000,0.500000,0.000000}%
\pgfsetfillcolor{currentfill}%
\pgfsetfillopacity{0.600000}%
\pgfsetlinewidth{0.000000pt}%
\definecolor{currentstroke}{rgb}{0.000000,0.000000,0.000000}%
\pgfsetstrokecolor{currentstroke}%
\pgfsetstrokeopacity{0.600000}%
\pgfsetdash{}{0pt}%
\pgfpathmoveto{\pgfqpoint{5.165579in}{0.385000in}}%
\pgfpathlineto{\pgfqpoint{5.177107in}{0.385000in}}%
\pgfpathlineto{\pgfqpoint{5.177107in}{0.432752in}}%
\pgfpathlineto{\pgfqpoint{5.165579in}{0.432752in}}%
\pgfpathlineto{\pgfqpoint{5.165579in}{0.385000in}}%
\pgfpathclose%
\pgfusepath{fill}%
\end{pgfscope}%
\begin{pgfscope}%
\pgfpathrectangle{\pgfqpoint{0.750000in}{0.385000in}}{\pgfqpoint{4.650000in}{2.695000in}}%
\pgfusepath{clip}%
\pgfsetbuttcap%
\pgfsetmiterjoin%
\definecolor{currentfill}{rgb}{0.000000,0.500000,0.000000}%
\pgfsetfillcolor{currentfill}%
\pgfsetfillopacity{0.600000}%
\pgfsetlinewidth{0.000000pt}%
\definecolor{currentstroke}{rgb}{0.000000,0.000000,0.000000}%
\pgfsetstrokecolor{currentstroke}%
\pgfsetstrokeopacity{0.600000}%
\pgfsetdash{}{0pt}%
\pgfpathmoveto{\pgfqpoint{5.177107in}{0.385000in}}%
\pgfpathlineto{\pgfqpoint{5.188636in}{0.385000in}}%
\pgfpathlineto{\pgfqpoint{5.188636in}{0.420814in}}%
\pgfpathlineto{\pgfqpoint{5.177107in}{0.420814in}}%
\pgfpathlineto{\pgfqpoint{5.177107in}{0.385000in}}%
\pgfpathclose%
\pgfusepath{fill}%
\end{pgfscope}%
\begin{pgfscope}%
\pgfsetbuttcap%
\pgfsetroundjoin%
\definecolor{currentfill}{rgb}{0.000000,0.000000,0.000000}%
\pgfsetfillcolor{currentfill}%
\pgfsetlinewidth{0.803000pt}%
\definecolor{currentstroke}{rgb}{0.000000,0.000000,0.000000}%
\pgfsetstrokecolor{currentstroke}%
\pgfsetdash{}{0pt}%
\pgfsys@defobject{currentmarker}{\pgfqpoint{0.000000in}{-0.048611in}}{\pgfqpoint{0.000000in}{0.000000in}}{%
\pgfpathmoveto{\pgfqpoint{0.000000in}{0.000000in}}%
\pgfpathlineto{\pgfqpoint{0.000000in}{-0.048611in}}%
\pgfusepath{stroke,fill}%
}%
\begin{pgfscope}%
\pgfsys@transformshift{1.345661in}{0.385000in}%
\pgfsys@useobject{currentmarker}{}%
\end{pgfscope}%
\end{pgfscope}%
\begin{pgfscope}%
\definecolor{textcolor}{rgb}{0.000000,0.000000,0.000000}%
\pgfsetstrokecolor{textcolor}%
\pgfsetfillcolor{textcolor}%
\pgftext[x=1.345661in,y=0.287778in,,top]{\color{textcolor}{\sffamily\fontsize{10.000000}{12.000000}\selectfont\catcode`\^=\active\def^{\ifmmode\sp\else\^{}\fi}\catcode`\%=\active\def%{\%}\ensuremath{-}4}}%
\end{pgfscope}%
\begin{pgfscope}%
\pgfsetbuttcap%
\pgfsetroundjoin%
\definecolor{currentfill}{rgb}{0.000000,0.000000,0.000000}%
\pgfsetfillcolor{currentfill}%
\pgfsetlinewidth{0.803000pt}%
\definecolor{currentstroke}{rgb}{0.000000,0.000000,0.000000}%
\pgfsetstrokecolor{currentstroke}%
\pgfsetdash{}{0pt}%
\pgfsys@defobject{currentmarker}{\pgfqpoint{0.000000in}{-0.048611in}}{\pgfqpoint{0.000000in}{0.000000in}}{%
\pgfpathmoveto{\pgfqpoint{0.000000in}{0.000000in}}%
\pgfpathlineto{\pgfqpoint{0.000000in}{-0.048611in}}%
\pgfusepath{stroke,fill}%
}%
\begin{pgfscope}%
\pgfsys@transformshift{2.114256in}{0.385000in}%
\pgfsys@useobject{currentmarker}{}%
\end{pgfscope}%
\end{pgfscope}%
\begin{pgfscope}%
\definecolor{textcolor}{rgb}{0.000000,0.000000,0.000000}%
\pgfsetstrokecolor{textcolor}%
\pgfsetfillcolor{textcolor}%
\pgftext[x=2.114256in,y=0.287778in,,top]{\color{textcolor}{\sffamily\fontsize{10.000000}{12.000000}\selectfont\catcode`\^=\active\def^{\ifmmode\sp\else\^{}\fi}\catcode`\%=\active\def%{\%}\ensuremath{-}2}}%
\end{pgfscope}%
\begin{pgfscope}%
\pgfsetbuttcap%
\pgfsetroundjoin%
\definecolor{currentfill}{rgb}{0.000000,0.000000,0.000000}%
\pgfsetfillcolor{currentfill}%
\pgfsetlinewidth{0.803000pt}%
\definecolor{currentstroke}{rgb}{0.000000,0.000000,0.000000}%
\pgfsetstrokecolor{currentstroke}%
\pgfsetdash{}{0pt}%
\pgfsys@defobject{currentmarker}{\pgfqpoint{0.000000in}{-0.048611in}}{\pgfqpoint{0.000000in}{0.000000in}}{%
\pgfpathmoveto{\pgfqpoint{0.000000in}{0.000000in}}%
\pgfpathlineto{\pgfqpoint{0.000000in}{-0.048611in}}%
\pgfusepath{stroke,fill}%
}%
\begin{pgfscope}%
\pgfsys@transformshift{2.882851in}{0.385000in}%
\pgfsys@useobject{currentmarker}{}%
\end{pgfscope}%
\end{pgfscope}%
\begin{pgfscope}%
\definecolor{textcolor}{rgb}{0.000000,0.000000,0.000000}%
\pgfsetstrokecolor{textcolor}%
\pgfsetfillcolor{textcolor}%
\pgftext[x=2.882851in,y=0.287778in,,top]{\color{textcolor}{\sffamily\fontsize{10.000000}{12.000000}\selectfont\catcode`\^=\active\def^{\ifmmode\sp\else\^{}\fi}\catcode`\%=\active\def%{\%}0}}%
\end{pgfscope}%
\begin{pgfscope}%
\pgfsetbuttcap%
\pgfsetroundjoin%
\definecolor{currentfill}{rgb}{0.000000,0.000000,0.000000}%
\pgfsetfillcolor{currentfill}%
\pgfsetlinewidth{0.803000pt}%
\definecolor{currentstroke}{rgb}{0.000000,0.000000,0.000000}%
\pgfsetstrokecolor{currentstroke}%
\pgfsetdash{}{0pt}%
\pgfsys@defobject{currentmarker}{\pgfqpoint{0.000000in}{-0.048611in}}{\pgfqpoint{0.000000in}{0.000000in}}{%
\pgfpathmoveto{\pgfqpoint{0.000000in}{0.000000in}}%
\pgfpathlineto{\pgfqpoint{0.000000in}{-0.048611in}}%
\pgfusepath{stroke,fill}%
}%
\begin{pgfscope}%
\pgfsys@transformshift{3.651446in}{0.385000in}%
\pgfsys@useobject{currentmarker}{}%
\end{pgfscope}%
\end{pgfscope}%
\begin{pgfscope}%
\definecolor{textcolor}{rgb}{0.000000,0.000000,0.000000}%
\pgfsetstrokecolor{textcolor}%
\pgfsetfillcolor{textcolor}%
\pgftext[x=3.651446in,y=0.287778in,,top]{\color{textcolor}{\sffamily\fontsize{10.000000}{12.000000}\selectfont\catcode`\^=\active\def^{\ifmmode\sp\else\^{}\fi}\catcode`\%=\active\def%{\%}2}}%
\end{pgfscope}%
\begin{pgfscope}%
\pgfsetbuttcap%
\pgfsetroundjoin%
\definecolor{currentfill}{rgb}{0.000000,0.000000,0.000000}%
\pgfsetfillcolor{currentfill}%
\pgfsetlinewidth{0.803000pt}%
\definecolor{currentstroke}{rgb}{0.000000,0.000000,0.000000}%
\pgfsetstrokecolor{currentstroke}%
\pgfsetdash{}{0pt}%
\pgfsys@defobject{currentmarker}{\pgfqpoint{0.000000in}{-0.048611in}}{\pgfqpoint{0.000000in}{0.000000in}}{%
\pgfpathmoveto{\pgfqpoint{0.000000in}{0.000000in}}%
\pgfpathlineto{\pgfqpoint{0.000000in}{-0.048611in}}%
\pgfusepath{stroke,fill}%
}%
\begin{pgfscope}%
\pgfsys@transformshift{4.420041in}{0.385000in}%
\pgfsys@useobject{currentmarker}{}%
\end{pgfscope}%
\end{pgfscope}%
\begin{pgfscope}%
\definecolor{textcolor}{rgb}{0.000000,0.000000,0.000000}%
\pgfsetstrokecolor{textcolor}%
\pgfsetfillcolor{textcolor}%
\pgftext[x=4.420041in,y=0.287778in,,top]{\color{textcolor}{\sffamily\fontsize{10.000000}{12.000000}\selectfont\catcode`\^=\active\def^{\ifmmode\sp\else\^{}\fi}\catcode`\%=\active\def%{\%}4}}%
\end{pgfscope}%
\begin{pgfscope}%
\pgfsetbuttcap%
\pgfsetroundjoin%
\definecolor{currentfill}{rgb}{0.000000,0.000000,0.000000}%
\pgfsetfillcolor{currentfill}%
\pgfsetlinewidth{0.803000pt}%
\definecolor{currentstroke}{rgb}{0.000000,0.000000,0.000000}%
\pgfsetstrokecolor{currentstroke}%
\pgfsetdash{}{0pt}%
\pgfsys@defobject{currentmarker}{\pgfqpoint{0.000000in}{-0.048611in}}{\pgfqpoint{0.000000in}{0.000000in}}{%
\pgfpathmoveto{\pgfqpoint{0.000000in}{0.000000in}}%
\pgfpathlineto{\pgfqpoint{0.000000in}{-0.048611in}}%
\pgfusepath{stroke,fill}%
}%
\begin{pgfscope}%
\pgfsys@transformshift{5.188636in}{0.385000in}%
\pgfsys@useobject{currentmarker}{}%
\end{pgfscope}%
\end{pgfscope}%
\begin{pgfscope}%
\definecolor{textcolor}{rgb}{0.000000,0.000000,0.000000}%
\pgfsetstrokecolor{textcolor}%
\pgfsetfillcolor{textcolor}%
\pgftext[x=5.188636in,y=0.287778in,,top]{\color{textcolor}{\sffamily\fontsize{10.000000}{12.000000}\selectfont\catcode`\^=\active\def^{\ifmmode\sp\else\^{}\fi}\catcode`\%=\active\def%{\%}6}}%
\end{pgfscope}%
\begin{pgfscope}%
\definecolor{textcolor}{rgb}{0.000000,0.000000,0.000000}%
\pgfsetstrokecolor{textcolor}%
\pgfsetfillcolor{textcolor}%
\pgftext[x=3.075000in,y=0.097809in,,top]{\color{textcolor}{\sffamily\fontsize{10.000000}{12.000000}\selectfont\catcode`\^=\active\def^{\ifmmode\sp\else\^{}\fi}\catcode`\%=\active\def%{\%}r.v}}%
\end{pgfscope}%
\begin{pgfscope}%
\pgfsetbuttcap%
\pgfsetroundjoin%
\definecolor{currentfill}{rgb}{0.000000,0.000000,0.000000}%
\pgfsetfillcolor{currentfill}%
\pgfsetlinewidth{0.803000pt}%
\definecolor{currentstroke}{rgb}{0.000000,0.000000,0.000000}%
\pgfsetstrokecolor{currentstroke}%
\pgfsetdash{}{0pt}%
\pgfsys@defobject{currentmarker}{\pgfqpoint{-0.048611in}{0.000000in}}{\pgfqpoint{-0.000000in}{0.000000in}}{%
\pgfpathmoveto{\pgfqpoint{-0.000000in}{0.000000in}}%
\pgfpathlineto{\pgfqpoint{-0.048611in}{0.000000in}}%
\pgfusepath{stroke,fill}%
}%
\begin{pgfscope}%
\pgfsys@transformshift{0.750000in}{0.385000in}%
\pgfsys@useobject{currentmarker}{}%
\end{pgfscope}%
\end{pgfscope}%
\begin{pgfscope}%
\definecolor{textcolor}{rgb}{0.000000,0.000000,0.000000}%
\pgfsetstrokecolor{textcolor}%
\pgfsetfillcolor{textcolor}%
\pgftext[x=0.431898in, y=0.332238in, left, base]{\color{textcolor}{\sffamily\fontsize{10.000000}{12.000000}\selectfont\catcode`\^=\active\def^{\ifmmode\sp\else\^{}\fi}\catcode`\%=\active\def%{\%}0.0}}%
\end{pgfscope}%
\begin{pgfscope}%
\pgfsetbuttcap%
\pgfsetroundjoin%
\definecolor{currentfill}{rgb}{0.000000,0.000000,0.000000}%
\pgfsetfillcolor{currentfill}%
\pgfsetlinewidth{0.803000pt}%
\definecolor{currentstroke}{rgb}{0.000000,0.000000,0.000000}%
\pgfsetstrokecolor{currentstroke}%
\pgfsetdash{}{0pt}%
\pgfsys@defobject{currentmarker}{\pgfqpoint{-0.048611in}{0.000000in}}{\pgfqpoint{-0.000000in}{0.000000in}}{%
\pgfpathmoveto{\pgfqpoint{-0.000000in}{0.000000in}}%
\pgfpathlineto{\pgfqpoint{-0.048611in}{0.000000in}}%
\pgfusepath{stroke,fill}%
}%
\begin{pgfscope}%
\pgfsys@transformshift{0.750000in}{0.729351in}%
\pgfsys@useobject{currentmarker}{}%
\end{pgfscope}%
\end{pgfscope}%
\begin{pgfscope}%
\definecolor{textcolor}{rgb}{0.000000,0.000000,0.000000}%
\pgfsetstrokecolor{textcolor}%
\pgfsetfillcolor{textcolor}%
\pgftext[x=0.431898in, y=0.676590in, left, base]{\color{textcolor}{\sffamily\fontsize{10.000000}{12.000000}\selectfont\catcode`\^=\active\def^{\ifmmode\sp\else\^{}\fi}\catcode`\%=\active\def%{\%}0.1}}%
\end{pgfscope}%
\begin{pgfscope}%
\pgfsetbuttcap%
\pgfsetroundjoin%
\definecolor{currentfill}{rgb}{0.000000,0.000000,0.000000}%
\pgfsetfillcolor{currentfill}%
\pgfsetlinewidth{0.803000pt}%
\definecolor{currentstroke}{rgb}{0.000000,0.000000,0.000000}%
\pgfsetstrokecolor{currentstroke}%
\pgfsetdash{}{0pt}%
\pgfsys@defobject{currentmarker}{\pgfqpoint{-0.048611in}{0.000000in}}{\pgfqpoint{-0.000000in}{0.000000in}}{%
\pgfpathmoveto{\pgfqpoint{-0.000000in}{0.000000in}}%
\pgfpathlineto{\pgfqpoint{-0.048611in}{0.000000in}}%
\pgfusepath{stroke,fill}%
}%
\begin{pgfscope}%
\pgfsys@transformshift{0.750000in}{1.073702in}%
\pgfsys@useobject{currentmarker}{}%
\end{pgfscope}%
\end{pgfscope}%
\begin{pgfscope}%
\definecolor{textcolor}{rgb}{0.000000,0.000000,0.000000}%
\pgfsetstrokecolor{textcolor}%
\pgfsetfillcolor{textcolor}%
\pgftext[x=0.431898in, y=1.020941in, left, base]{\color{textcolor}{\sffamily\fontsize{10.000000}{12.000000}\selectfont\catcode`\^=\active\def^{\ifmmode\sp\else\^{}\fi}\catcode`\%=\active\def%{\%}0.2}}%
\end{pgfscope}%
\begin{pgfscope}%
\pgfsetbuttcap%
\pgfsetroundjoin%
\definecolor{currentfill}{rgb}{0.000000,0.000000,0.000000}%
\pgfsetfillcolor{currentfill}%
\pgfsetlinewidth{0.803000pt}%
\definecolor{currentstroke}{rgb}{0.000000,0.000000,0.000000}%
\pgfsetstrokecolor{currentstroke}%
\pgfsetdash{}{0pt}%
\pgfsys@defobject{currentmarker}{\pgfqpoint{-0.048611in}{0.000000in}}{\pgfqpoint{-0.000000in}{0.000000in}}{%
\pgfpathmoveto{\pgfqpoint{-0.000000in}{0.000000in}}%
\pgfpathlineto{\pgfqpoint{-0.048611in}{0.000000in}}%
\pgfusepath{stroke,fill}%
}%
\begin{pgfscope}%
\pgfsys@transformshift{0.750000in}{1.418053in}%
\pgfsys@useobject{currentmarker}{}%
\end{pgfscope}%
\end{pgfscope}%
\begin{pgfscope}%
\definecolor{textcolor}{rgb}{0.000000,0.000000,0.000000}%
\pgfsetstrokecolor{textcolor}%
\pgfsetfillcolor{textcolor}%
\pgftext[x=0.431898in, y=1.365292in, left, base]{\color{textcolor}{\sffamily\fontsize{10.000000}{12.000000}\selectfont\catcode`\^=\active\def^{\ifmmode\sp\else\^{}\fi}\catcode`\%=\active\def%{\%}0.3}}%
\end{pgfscope}%
\begin{pgfscope}%
\pgfsetbuttcap%
\pgfsetroundjoin%
\definecolor{currentfill}{rgb}{0.000000,0.000000,0.000000}%
\pgfsetfillcolor{currentfill}%
\pgfsetlinewidth{0.803000pt}%
\definecolor{currentstroke}{rgb}{0.000000,0.000000,0.000000}%
\pgfsetstrokecolor{currentstroke}%
\pgfsetdash{}{0pt}%
\pgfsys@defobject{currentmarker}{\pgfqpoint{-0.048611in}{0.000000in}}{\pgfqpoint{-0.000000in}{0.000000in}}{%
\pgfpathmoveto{\pgfqpoint{-0.000000in}{0.000000in}}%
\pgfpathlineto{\pgfqpoint{-0.048611in}{0.000000in}}%
\pgfusepath{stroke,fill}%
}%
\begin{pgfscope}%
\pgfsys@transformshift{0.750000in}{1.762405in}%
\pgfsys@useobject{currentmarker}{}%
\end{pgfscope}%
\end{pgfscope}%
\begin{pgfscope}%
\definecolor{textcolor}{rgb}{0.000000,0.000000,0.000000}%
\pgfsetstrokecolor{textcolor}%
\pgfsetfillcolor{textcolor}%
\pgftext[x=0.431898in, y=1.709643in, left, base]{\color{textcolor}{\sffamily\fontsize{10.000000}{12.000000}\selectfont\catcode`\^=\active\def^{\ifmmode\sp\else\^{}\fi}\catcode`\%=\active\def%{\%}0.4}}%
\end{pgfscope}%
\begin{pgfscope}%
\pgfsetbuttcap%
\pgfsetroundjoin%
\definecolor{currentfill}{rgb}{0.000000,0.000000,0.000000}%
\pgfsetfillcolor{currentfill}%
\pgfsetlinewidth{0.803000pt}%
\definecolor{currentstroke}{rgb}{0.000000,0.000000,0.000000}%
\pgfsetstrokecolor{currentstroke}%
\pgfsetdash{}{0pt}%
\pgfsys@defobject{currentmarker}{\pgfqpoint{-0.048611in}{0.000000in}}{\pgfqpoint{-0.000000in}{0.000000in}}{%
\pgfpathmoveto{\pgfqpoint{-0.000000in}{0.000000in}}%
\pgfpathlineto{\pgfqpoint{-0.048611in}{0.000000in}}%
\pgfusepath{stroke,fill}%
}%
\begin{pgfscope}%
\pgfsys@transformshift{0.750000in}{2.106756in}%
\pgfsys@useobject{currentmarker}{}%
\end{pgfscope}%
\end{pgfscope}%
\begin{pgfscope}%
\definecolor{textcolor}{rgb}{0.000000,0.000000,0.000000}%
\pgfsetstrokecolor{textcolor}%
\pgfsetfillcolor{textcolor}%
\pgftext[x=0.431898in, y=2.053994in, left, base]{\color{textcolor}{\sffamily\fontsize{10.000000}{12.000000}\selectfont\catcode`\^=\active\def^{\ifmmode\sp\else\^{}\fi}\catcode`\%=\active\def%{\%}0.5}}%
\end{pgfscope}%
\begin{pgfscope}%
\pgfsetbuttcap%
\pgfsetroundjoin%
\definecolor{currentfill}{rgb}{0.000000,0.000000,0.000000}%
\pgfsetfillcolor{currentfill}%
\pgfsetlinewidth{0.803000pt}%
\definecolor{currentstroke}{rgb}{0.000000,0.000000,0.000000}%
\pgfsetstrokecolor{currentstroke}%
\pgfsetdash{}{0pt}%
\pgfsys@defobject{currentmarker}{\pgfqpoint{-0.048611in}{0.000000in}}{\pgfqpoint{-0.000000in}{0.000000in}}{%
\pgfpathmoveto{\pgfqpoint{-0.000000in}{0.000000in}}%
\pgfpathlineto{\pgfqpoint{-0.048611in}{0.000000in}}%
\pgfusepath{stroke,fill}%
}%
\begin{pgfscope}%
\pgfsys@transformshift{0.750000in}{2.451107in}%
\pgfsys@useobject{currentmarker}{}%
\end{pgfscope}%
\end{pgfscope}%
\begin{pgfscope}%
\definecolor{textcolor}{rgb}{0.000000,0.000000,0.000000}%
\pgfsetstrokecolor{textcolor}%
\pgfsetfillcolor{textcolor}%
\pgftext[x=0.431898in, y=2.398345in, left, base]{\color{textcolor}{\sffamily\fontsize{10.000000}{12.000000}\selectfont\catcode`\^=\active\def^{\ifmmode\sp\else\^{}\fi}\catcode`\%=\active\def%{\%}0.6}}%
\end{pgfscope}%
\begin{pgfscope}%
\pgfsetbuttcap%
\pgfsetroundjoin%
\definecolor{currentfill}{rgb}{0.000000,0.000000,0.000000}%
\pgfsetfillcolor{currentfill}%
\pgfsetlinewidth{0.803000pt}%
\definecolor{currentstroke}{rgb}{0.000000,0.000000,0.000000}%
\pgfsetstrokecolor{currentstroke}%
\pgfsetdash{}{0pt}%
\pgfsys@defobject{currentmarker}{\pgfqpoint{-0.048611in}{0.000000in}}{\pgfqpoint{-0.000000in}{0.000000in}}{%
\pgfpathmoveto{\pgfqpoint{-0.000000in}{0.000000in}}%
\pgfpathlineto{\pgfqpoint{-0.048611in}{0.000000in}}%
\pgfusepath{stroke,fill}%
}%
\begin{pgfscope}%
\pgfsys@transformshift{0.750000in}{2.795458in}%
\pgfsys@useobject{currentmarker}{}%
\end{pgfscope}%
\end{pgfscope}%
\begin{pgfscope}%
\definecolor{textcolor}{rgb}{0.000000,0.000000,0.000000}%
\pgfsetstrokecolor{textcolor}%
\pgfsetfillcolor{textcolor}%
\pgftext[x=0.431898in, y=2.742697in, left, base]{\color{textcolor}{\sffamily\fontsize{10.000000}{12.000000}\selectfont\catcode`\^=\active\def^{\ifmmode\sp\else\^{}\fi}\catcode`\%=\active\def%{\%}0.7}}%
\end{pgfscope}%
\begin{pgfscope}%
\definecolor{textcolor}{rgb}{0.000000,0.000000,0.000000}%
\pgfsetstrokecolor{textcolor}%
\pgfsetfillcolor{textcolor}%
\pgftext[x=0.376343in,y=1.732500in,,bottom,rotate=90.000000]{\color{textcolor}{\sffamily\fontsize{10.000000}{12.000000}\selectfont\catcode`\^=\active\def^{\ifmmode\sp\else\^{}\fi}\catcode`\%=\active\def%{\%}p.d.f}}%
\end{pgfscope}%
\begin{pgfscope}%
\pgfpathrectangle{\pgfqpoint{0.750000in}{0.385000in}}{\pgfqpoint{4.650000in}{2.695000in}}%
\pgfusepath{clip}%
\pgfsetrectcap%
\pgfsetroundjoin%
\pgfsetlinewidth{1.505625pt}%
\definecolor{currentstroke}{rgb}{0.121569,0.466667,0.705882}%
\pgfsetstrokecolor{currentstroke}%
\pgfsetdash{}{0pt}%
\pgfpathmoveto{\pgfqpoint{0.961364in}{0.385005in}}%
\pgfpathlineto{\pgfqpoint{1.434523in}{0.386131in}}%
\pgfpathlineto{\pgfqpoint{1.557621in}{0.388595in}}%
\pgfpathlineto{\pgfqpoint{1.638404in}{0.392259in}}%
\pgfpathlineto{\pgfqpoint{1.703800in}{0.397414in}}%
\pgfpathlineto{\pgfqpoint{1.757656in}{0.403896in}}%
\pgfpathlineto{\pgfqpoint{1.803818in}{0.411667in}}%
\pgfpathlineto{\pgfqpoint{1.846133in}{0.421108in}}%
\pgfpathlineto{\pgfqpoint{1.884601in}{0.432066in}}%
\pgfpathlineto{\pgfqpoint{1.919222in}{0.444235in}}%
\pgfpathlineto{\pgfqpoint{1.953844in}{0.458947in}}%
\pgfpathlineto{\pgfqpoint{1.984618in}{0.474455in}}%
\pgfpathlineto{\pgfqpoint{2.015393in}{0.492523in}}%
\pgfpathlineto{\pgfqpoint{2.046167in}{0.513414in}}%
\pgfpathlineto{\pgfqpoint{2.073095in}{0.534211in}}%
\pgfpathlineto{\pgfqpoint{2.100023in}{0.557526in}}%
\pgfpathlineto{\pgfqpoint{2.126951in}{0.583508in}}%
\pgfpathlineto{\pgfqpoint{2.153878in}{0.612284in}}%
\pgfpathlineto{\pgfqpoint{2.184653in}{0.648722in}}%
\pgfpathlineto{\pgfqpoint{2.215428in}{0.689047in}}%
\pgfpathlineto{\pgfqpoint{2.246202in}{0.733296in}}%
\pgfpathlineto{\pgfqpoint{2.276977in}{0.781435in}}%
\pgfpathlineto{\pgfqpoint{2.311598in}{0.840086in}}%
\pgfpathlineto{\pgfqpoint{2.346220in}{0.903193in}}%
\pgfpathlineto{\pgfqpoint{2.384688in}{0.977954in}}%
\pgfpathlineto{\pgfqpoint{2.430850in}{1.072879in}}%
\pgfpathlineto{\pgfqpoint{2.496246in}{1.213225in}}%
\pgfpathlineto{\pgfqpoint{2.577029in}{1.385908in}}%
\pgfpathlineto{\pgfqpoint{2.619344in}{1.470967in}}%
\pgfpathlineto{\pgfqpoint{2.653965in}{1.535488in}}%
\pgfpathlineto{\pgfqpoint{2.684740in}{1.587828in}}%
\pgfpathlineto{\pgfqpoint{2.711668in}{1.629013in}}%
\pgfpathlineto{\pgfqpoint{2.734749in}{1.660441in}}%
\pgfpathlineto{\pgfqpoint{2.757830in}{1.687955in}}%
\pgfpathlineto{\pgfqpoint{2.777064in}{1.707687in}}%
\pgfpathlineto{\pgfqpoint{2.796298in}{1.724358in}}%
\pgfpathlineto{\pgfqpoint{2.815532in}{1.737845in}}%
\pgfpathlineto{\pgfqpoint{2.830919in}{1.746276in}}%
\pgfpathlineto{\pgfqpoint{2.846306in}{1.752565in}}%
\pgfpathlineto{\pgfqpoint{2.861694in}{1.756682in}}%
\pgfpathlineto{\pgfqpoint{2.877081in}{1.758608in}}%
\pgfpathlineto{\pgfqpoint{2.892468in}{1.758332in}}%
\pgfpathlineto{\pgfqpoint{2.907856in}{1.755858in}}%
\pgfpathlineto{\pgfqpoint{2.923243in}{1.751195in}}%
\pgfpathlineto{\pgfqpoint{2.938630in}{1.744368in}}%
\pgfpathlineto{\pgfqpoint{2.954017in}{1.735408in}}%
\pgfpathlineto{\pgfqpoint{2.973252in}{1.721274in}}%
\pgfpathlineto{\pgfqpoint{2.992486in}{1.703981in}}%
\pgfpathlineto{\pgfqpoint{3.011720in}{1.683654in}}%
\pgfpathlineto{\pgfqpoint{3.034801in}{1.655467in}}%
\pgfpathlineto{\pgfqpoint{3.057882in}{1.623416in}}%
\pgfpathlineto{\pgfqpoint{3.084809in}{1.581577in}}%
\pgfpathlineto{\pgfqpoint{3.111737in}{1.535488in}}%
\pgfpathlineto{\pgfqpoint{3.142512in}{1.478392in}}%
\pgfpathlineto{\pgfqpoint{3.180980in}{1.401778in}}%
\pgfpathlineto{\pgfqpoint{3.230989in}{1.296394in}}%
\pgfpathlineto{\pgfqpoint{3.369474in}{1.001213in}}%
\pgfpathlineto{\pgfqpoint{3.411789in}{0.917777in}}%
\pgfpathlineto{\pgfqpoint{3.450257in}{0.846886in}}%
\pgfpathlineto{\pgfqpoint{3.484879in}{0.787721in}}%
\pgfpathlineto{\pgfqpoint{3.515653in}{0.739102in}}%
\pgfpathlineto{\pgfqpoint{3.546428in}{0.694363in}}%
\pgfpathlineto{\pgfqpoint{3.577203in}{0.653549in}}%
\pgfpathlineto{\pgfqpoint{3.607977in}{0.616629in}}%
\pgfpathlineto{\pgfqpoint{3.638752in}{0.583508in}}%
\pgfpathlineto{\pgfqpoint{3.669526in}{0.554035in}}%
\pgfpathlineto{\pgfqpoint{3.700301in}{0.528018in}}%
\pgfpathlineto{\pgfqpoint{3.731075in}{0.505232in}}%
\pgfpathlineto{\pgfqpoint{3.761850in}{0.485431in}}%
\pgfpathlineto{\pgfqpoint{3.792625in}{0.468354in}}%
\pgfpathlineto{\pgfqpoint{3.823399in}{0.453738in}}%
\pgfpathlineto{\pgfqpoint{3.858021in}{0.439914in}}%
\pgfpathlineto{\pgfqpoint{3.892642in}{0.428515in}}%
\pgfpathlineto{\pgfqpoint{3.931110in}{0.418283in}}%
\pgfpathlineto{\pgfqpoint{3.973425in}{0.409499in}}%
\pgfpathlineto{\pgfqpoint{4.019587in}{0.402297in}}%
\pgfpathlineto{\pgfqpoint{4.073443in}{0.396316in}}%
\pgfpathlineto{\pgfqpoint{4.134992in}{0.391802in}}%
\pgfpathlineto{\pgfqpoint{4.211928in}{0.388472in}}%
\pgfpathlineto{\pgfqpoint{4.319639in}{0.386266in}}%
\pgfpathlineto{\pgfqpoint{4.496593in}{0.385204in}}%
\pgfpathlineto{\pgfqpoint{4.804339in}{0.385005in}}%
\pgfpathlineto{\pgfqpoint{4.804339in}{0.385005in}}%
\pgfusepath{stroke}%
\end{pgfscope}%
\begin{pgfscope}%
\pgfpathrectangle{\pgfqpoint{0.750000in}{0.385000in}}{\pgfqpoint{4.650000in}{2.695000in}}%
\pgfusepath{clip}%
\pgfsetrectcap%
\pgfsetroundjoin%
\pgfsetlinewidth{1.505625pt}%
\definecolor{currentstroke}{rgb}{1.000000,0.647059,0.000000}%
\pgfsetstrokecolor{currentstroke}%
\pgfsetdash{}{0pt}%
\pgfpathmoveto{\pgfqpoint{2.884775in}{0.385221in}}%
\pgfpathlineto{\pgfqpoint{2.888621in}{0.398592in}}%
\pgfpathlineto{\pgfqpoint{2.892468in}{0.446128in}}%
\pgfpathlineto{\pgfqpoint{2.896315in}{0.527701in}}%
\pgfpathlineto{\pgfqpoint{2.904009in}{0.757936in}}%
\pgfpathlineto{\pgfqpoint{2.930937in}{1.651220in}}%
\pgfpathlineto{\pgfqpoint{2.942477in}{1.945309in}}%
\pgfpathlineto{\pgfqpoint{2.954017in}{2.174581in}}%
\pgfpathlineto{\pgfqpoint{2.965558in}{2.346761in}}%
\pgfpathlineto{\pgfqpoint{2.973252in}{2.434453in}}%
\pgfpathlineto{\pgfqpoint{2.980945in}{2.503612in}}%
\pgfpathlineto{\pgfqpoint{2.988639in}{2.556695in}}%
\pgfpathlineto{\pgfqpoint{2.996332in}{2.595905in}}%
\pgfpathlineto{\pgfqpoint{3.004026in}{2.623189in}}%
\pgfpathlineto{\pgfqpoint{3.011720in}{2.640256in}}%
\pgfpathlineto{\pgfqpoint{3.015567in}{2.645431in}}%
\pgfpathlineto{\pgfqpoint{3.019413in}{2.648593in}}%
\pgfpathlineto{\pgfqpoint{3.023260in}{2.649898in}}%
\pgfpathlineto{\pgfqpoint{3.027107in}{2.649491in}}%
\pgfpathlineto{\pgfqpoint{3.030954in}{2.647505in}}%
\pgfpathlineto{\pgfqpoint{3.038648in}{2.639292in}}%
\pgfpathlineto{\pgfqpoint{3.046341in}{2.626157in}}%
\pgfpathlineto{\pgfqpoint{3.054035in}{2.608876in}}%
\pgfpathlineto{\pgfqpoint{3.065575in}{2.576622in}}%
\pgfpathlineto{\pgfqpoint{3.080963in}{2.524623in}}%
\pgfpathlineto{\pgfqpoint{3.100197in}{2.449881in}}%
\pgfpathlineto{\pgfqpoint{3.130971in}{2.318540in}}%
\pgfpathlineto{\pgfqpoint{3.204061in}{2.002364in}}%
\pgfpathlineto{\pgfqpoint{3.238682in}{1.864276in}}%
\pgfpathlineto{\pgfqpoint{3.269457in}{1.750536in}}%
\pgfpathlineto{\pgfqpoint{3.300231in}{1.645569in}}%
\pgfpathlineto{\pgfqpoint{3.331006in}{1.549179in}}%
\pgfpathlineto{\pgfqpoint{3.361781in}{1.460932in}}%
\pgfpathlineto{\pgfqpoint{3.392555in}{1.380273in}}%
\pgfpathlineto{\pgfqpoint{3.423330in}{1.306606in}}%
\pgfpathlineto{\pgfqpoint{3.454104in}{1.239333in}}%
\pgfpathlineto{\pgfqpoint{3.484879in}{1.177881in}}%
\pgfpathlineto{\pgfqpoint{3.515653in}{1.121712in}}%
\pgfpathlineto{\pgfqpoint{3.546428in}{1.070329in}}%
\pgfpathlineto{\pgfqpoint{3.577203in}{1.023279in}}%
\pgfpathlineto{\pgfqpoint{3.607977in}{0.980150in}}%
\pgfpathlineto{\pgfqpoint{3.638752in}{0.940570in}}%
\pgfpathlineto{\pgfqpoint{3.673373in}{0.899869in}}%
\pgfpathlineto{\pgfqpoint{3.707995in}{0.862810in}}%
\pgfpathlineto{\pgfqpoint{3.742616in}{0.829016in}}%
\pgfpathlineto{\pgfqpoint{3.777237in}{0.798150in}}%
\pgfpathlineto{\pgfqpoint{3.815706in}{0.766931in}}%
\pgfpathlineto{\pgfqpoint{3.854174in}{0.738604in}}%
\pgfpathlineto{\pgfqpoint{3.896489in}{0.710411in}}%
\pgfpathlineto{\pgfqpoint{3.938804in}{0.684971in}}%
\pgfpathlineto{\pgfqpoint{3.984966in}{0.659990in}}%
\pgfpathlineto{\pgfqpoint{4.031128in}{0.637556in}}%
\pgfpathlineto{\pgfqpoint{4.081136in}{0.615773in}}%
\pgfpathlineto{\pgfqpoint{4.134992in}{0.594874in}}%
\pgfpathlineto{\pgfqpoint{4.192694in}{0.575038in}}%
\pgfpathlineto{\pgfqpoint{4.254243in}{0.556390in}}%
\pgfpathlineto{\pgfqpoint{4.319639in}{0.539009in}}%
\pgfpathlineto{\pgfqpoint{4.388882in}{0.522933in}}%
\pgfpathlineto{\pgfqpoint{4.465819in}{0.507444in}}%
\pgfpathlineto{\pgfqpoint{4.546602in}{0.493438in}}%
\pgfpathlineto{\pgfqpoint{4.635079in}{0.480306in}}%
\pgfpathlineto{\pgfqpoint{4.735096in}{0.467750in}}%
\pgfpathlineto{\pgfqpoint{4.804339in}{0.460243in}}%
\pgfpathlineto{\pgfqpoint{4.804339in}{0.460243in}}%
\pgfusepath{stroke}%
\end{pgfscope}%
\begin{pgfscope}%
\pgfsetrectcap%
\pgfsetmiterjoin%
\pgfsetlinewidth{0.803000pt}%
\definecolor{currentstroke}{rgb}{0.000000,0.000000,0.000000}%
\pgfsetstrokecolor{currentstroke}%
\pgfsetdash{}{0pt}%
\pgfpathmoveto{\pgfqpoint{0.750000in}{0.385000in}}%
\pgfpathlineto{\pgfqpoint{0.750000in}{3.080000in}}%
\pgfusepath{stroke}%
\end{pgfscope}%
\begin{pgfscope}%
\pgfsetrectcap%
\pgfsetmiterjoin%
\pgfsetlinewidth{0.803000pt}%
\definecolor{currentstroke}{rgb}{0.000000,0.000000,0.000000}%
\pgfsetstrokecolor{currentstroke}%
\pgfsetdash{}{0pt}%
\pgfpathmoveto{\pgfqpoint{5.400000in}{0.385000in}}%
\pgfpathlineto{\pgfqpoint{5.400000in}{3.080000in}}%
\pgfusepath{stroke}%
\end{pgfscope}%
\begin{pgfscope}%
\pgfsetrectcap%
\pgfsetmiterjoin%
\pgfsetlinewidth{0.803000pt}%
\definecolor{currentstroke}{rgb}{0.000000,0.000000,0.000000}%
\pgfsetstrokecolor{currentstroke}%
\pgfsetdash{}{0pt}%
\pgfpathmoveto{\pgfqpoint{0.750000in}{0.385000in}}%
\pgfpathlineto{\pgfqpoint{5.400000in}{0.385000in}}%
\pgfusepath{stroke}%
\end{pgfscope}%
\begin{pgfscope}%
\pgfsetrectcap%
\pgfsetmiterjoin%
\pgfsetlinewidth{0.803000pt}%
\definecolor{currentstroke}{rgb}{0.000000,0.000000,0.000000}%
\pgfsetstrokecolor{currentstroke}%
\pgfsetdash{}{0pt}%
\pgfpathmoveto{\pgfqpoint{0.750000in}{3.080000in}}%
\pgfpathlineto{\pgfqpoint{5.400000in}{3.080000in}}%
\pgfusepath{stroke}%
\end{pgfscope}%
\begin{pgfscope}%
\definecolor{textcolor}{rgb}{0.000000,0.000000,0.000000}%
\pgfsetstrokecolor{textcolor}%
\pgfsetfillcolor{textcolor}%
\pgftext[x=3.075000in,y=3.163333in,,base]{\color{textcolor}{\sffamily\fontsize{12.000000}{14.400000}\selectfont\catcode`\^=\active\def^{\ifmmode\sp\else\^{}\fi}\catcode`\%=\active\def%{\%}Graph 1.}}%
\end{pgfscope}%
\begin{pgfscope}%
\pgfsetbuttcap%
\pgfsetmiterjoin%
\definecolor{currentfill}{rgb}{1.000000,1.000000,1.000000}%
\pgfsetfillcolor{currentfill}%
\pgfsetfillopacity{0.800000}%
\pgfsetlinewidth{1.003750pt}%
\definecolor{currentstroke}{rgb}{0.800000,0.800000,0.800000}%
\pgfsetstrokecolor{currentstroke}%
\pgfsetstrokeopacity{0.800000}%
\pgfsetdash{}{0pt}%
\pgfpathmoveto{\pgfqpoint{4.497347in}{2.549509in}}%
\pgfpathlineto{\pgfqpoint{5.302778in}{2.549509in}}%
\pgfpathquadraticcurveto{\pgfqpoint{5.330556in}{2.549509in}}{\pgfqpoint{5.330556in}{2.577287in}}%
\pgfpathlineto{\pgfqpoint{5.330556in}{2.982778in}}%
\pgfpathquadraticcurveto{\pgfqpoint{5.330556in}{3.010556in}}{\pgfqpoint{5.302778in}{3.010556in}}%
\pgfpathlineto{\pgfqpoint{4.497347in}{3.010556in}}%
\pgfpathquadraticcurveto{\pgfqpoint{4.469569in}{3.010556in}}{\pgfqpoint{4.469569in}{2.982778in}}%
\pgfpathlineto{\pgfqpoint{4.469569in}{2.577287in}}%
\pgfpathquadraticcurveto{\pgfqpoint{4.469569in}{2.549509in}}{\pgfqpoint{4.497347in}{2.549509in}}%
\pgfpathlineto{\pgfqpoint{4.497347in}{2.549509in}}%
\pgfpathclose%
\pgfusepath{stroke,fill}%
\end{pgfscope}%
\begin{pgfscope}%
\pgfsetrectcap%
\pgfsetroundjoin%
\pgfsetlinewidth{1.505625pt}%
\definecolor{currentstroke}{rgb}{0.121569,0.466667,0.705882}%
\pgfsetstrokecolor{currentstroke}%
\pgfsetdash{}{0pt}%
\pgfpathmoveto{\pgfqpoint{4.525124in}{2.898088in}}%
\pgfpathlineto{\pgfqpoint{4.664013in}{2.898088in}}%
\pgfpathlineto{\pgfqpoint{4.802902in}{2.898088in}}%
\pgfusepath{stroke}%
\end{pgfscope}%
\begin{pgfscope}%
\definecolor{textcolor}{rgb}{0.000000,0.000000,0.000000}%
\pgfsetstrokecolor{textcolor}%
\pgfsetfillcolor{textcolor}%
\pgftext[x=4.914013in,y=2.849477in,left,base]{\color{textcolor}{\sffamily\fontsize{10.000000}{12.000000}\selectfont\catcode`\^=\active\def^{\ifmmode\sp\else\^{}\fi}\catcode`\%=\active\def%{\%}$f_X(x)$}}%
\end{pgfscope}%
\begin{pgfscope}%
\pgfsetrectcap%
\pgfsetroundjoin%
\pgfsetlinewidth{1.505625pt}%
\definecolor{currentstroke}{rgb}{1.000000,0.647059,0.000000}%
\pgfsetstrokecolor{currentstroke}%
\pgfsetdash{}{0pt}%
\pgfpathmoveto{\pgfqpoint{4.525124in}{2.688398in}}%
\pgfpathlineto{\pgfqpoint{4.664013in}{2.688398in}}%
\pgfpathlineto{\pgfqpoint{4.802902in}{2.688398in}}%
\pgfusepath{stroke}%
\end{pgfscope}%
\begin{pgfscope}%
\definecolor{textcolor}{rgb}{0.000000,0.000000,0.000000}%
\pgfsetstrokecolor{textcolor}%
\pgfsetfillcolor{textcolor}%
\pgftext[x=4.914013in,y=2.639787in,left,base]{\color{textcolor}{\sffamily\fontsize{10.000000}{12.000000}\selectfont\catcode`\^=\active\def^{\ifmmode\sp\else\^{}\fi}\catcode`\%=\active\def%{\%}$f_Y(y)$}}%
\end{pgfscope}%
\end{pgfpicture}%
\makeatother%
\endgroup%
  \end{center}

    
    It is kind of wrong to name y and x labels and "p.d.f" and "r.v", but I really can't think of better labels.\\
    As expected, the green histogram nicely overlaps with our yellow plot.
    