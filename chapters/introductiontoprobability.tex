\chapter{Introduction to Probability}
The concept ``probability'' is used very often in everyday language to describe the chance of something happening. Mathematically, Probability is a language to quantify uncertainty. 
This chapter will introduce necessary and basic concepts and namely, \textbf{Probability Theory}. We will start the chapter with the elementary \textit{Set Theory}.
%-------------------------------------------------
\section{Set Theory}
Set Theory is a branch of mathematics that studies \textit{sets}, which we will define shortly. This branch is, like other parts of mathematics, very deep and complex. We will learn only the most important concepts , which is in high-school level, needed to understand later sections and chapters.
\par
We will quickly introduce the concepts and briefly explain them. The reader may skip this section if they already know about sets and their basic properties.

\subsection*{Sets}
A \textbf{Set} is a collection of different objects, which are called \textit{elements} of the set. The sets are notated as capital letters such as $S$.
If $x$ is an element of a set $S$, we write $x \in S$. Otherwise we write $ x \not\in S$. A set with no elements is called \textbf{empty set} and is notated as $\emptyset$. \par
If $x_1,x_2,...,x_n$ are the elements of the set $S$, we write:
$$ S \in \{x_1,x_2,...,x_n\} $$
We can visualize the sets in 2D with a rectangle, circles and their intersections. The rectangle represent $\textbf{Sample Space} \ \Omega$, or $\textbf{Universal Set}$, which we will explain next section.    
\par
If S is set of all even numbers smaller than 12, we can draw the diagram as:



We can specify our set  as a selection from a larger set. If we want to write the set of all even integers, we can write (Here the set of integers is the universal set):
$$ S = \{n \in \mathbb{Z}: \frac{n}{2} \ \text{is an integer} \}$$
\par 
If a set $A$'s elements are also the elements of $B$, we say that $A$ is a \textbf{subset} of $B$. We can notate it as:
$$ A \subseteq B $$
\par
If a set $A$ is subset of $B$, but is not equal to $B$, we say that $A$ is \textbf{proper subset} of $B$. We can notate it as:
$$A \subsetneq B$$

\subsection*{Set operations}
\textbf{Union} of sets $A$,$B$ is a set that contains the elements of $A$ and $B$:
$$A \cup B = \{n:n \in A  \lor  n  \in B\}$$
\textbf{Intersection} of sets $A$,$B$ is a set that contains both the elements of $A$ and $B$:
$$A \cap B = \{n:n \in A  \land  n  \in B\}$$

\par

%--------------------------------------------------




\section{Sample Space and Events}
The Sample Space, usually denoted as $S$ or $\Omega$, is the $\textit{set}$ of all possible outcomes of an experiment. It is also called  \textbf{universal set}. Subsets of $\Omega$ are called $\textbf{events}$. A sample element of $\Omega$ is denoted as $\omega$.

\begin{example}
    If we toss a six dice once, then $\Omega = \{1,2,3,4,5,6\}$, the even that the side is even is $A= \{2,4,6\}$ while $\omega \in \{1,2,3,4,5,6\}$
\end{example}

With the new definition, we can make another set operation: $\textbf{complement}$ of the set $A$ is a set of elements $\Omega$ that do not belong to $A$. 
$$A^{c} = \{n: n \in \Omega \land n \not\in A \}$$