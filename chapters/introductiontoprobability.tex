\chapter{Introduction to Probability}
The concept ``probability'' is used very often in everyday language to describe the chance of something happening. Mathematically, Probability is a language to quantify uncertainty. 
This chapter will introduce necessary and basic concepts and namely, \textbf{Probability Theory}. We will start the chapter with the elementary \textit{Set Theory}.
%-------------------------------------------------
\section{Set Theory}
Set Theory is a branch of mathematics that studies \textit{sets}, which we will define shortly. This branch is, like other parts of mathematics, very deep and complex. We will learn only the most important concepts , which is in high-school level, needed to understand later sections and chapters.
\par
We will quickly introduce the concepts and briefly explain them. The reader may skip this section if they already know about sets and their basic properties.

\subsection*{Sets}
A \textbf{Set} is a collection of different objects, which are called \textit{elements} of the set. The sets are notated as capital letters such as $S$.
If $x$ is an element of a set $S$, we write $x \in S$. Otherwise we write $ x \not\in S$. A set with no elements is called \textbf{empty set} and is notated as $\emptyset$. \par
If $x_1,x_2,...,x_n$ are the elements of the set $S$, we write:
$$ S \in \{x_1,x_2,...,x_n\} $$
\par
If S is set of all even numbers smaller than 12, we can draw the diagram as:



We can specify our set  as a selection from a larger set. If we want to write the set of all even integers, we can write (Here the set of integers is the universal set):
$$ S = \{n \in \mathbb{Z}: \frac{n}{2} \ \text{is an integer} \}$$
\par 
If a set $A$'s elements are also the elements of $B$, we say that $A$ is a \textbf{subset} of $B$. We can notate it as:
$$ A \subseteq B $$
\par
If a set $A$ is subset of $B$, but is not equal to $B$, we say that $A$ is \textbf{proper subset} of $B$. We can notate it as:
$$A \subsetneq B$$

\subsection*{Set operations}
\textbf{Union} of sets $A$,$B$ is a set that contains the elements of $A$ and $B$:
$$A \cup B = \{n:n \in A  \lor  n  \in B\}$$
We can visualize the sets in 2D with circles and their intersections.  
\begin{figure}[!h]
    \centering
    \usetikzlibrary{arrows}
\definecolor{ccffff}{rgb}{0.8,1.,1.}
\begin{tikzpicture}[line cap=round,line join=round,>=triangle 45,x=1.0cm,y=1.0cm, scale=0.6,every node/.style={scale=0.6}]
\clip(-0.1870638729100964,-0.46671950332886286) rectangle (10.301322760748397,5.120090328019539);
\draw [line width=1.pt,color=ccffff,fill=ccffff,fill opacity=1.0] (3.7100909661527863,2.5) circle (2.1cm);
\draw [line width=1.pt,color=ccffff,fill=ccffff,fill opacity=1.0] (6.289909033847214,2.5) circle (2.1cm);
\draw [line width=1.pt] (0.,0.)-- (0.,5.);
\draw [line width=1.pt] (0.,5.)-- (10.,5.);
\draw [line width=1.pt] (10.,5.)-- (10.,0.);
\draw [line width=1.pt] (10.,0.)-- (0.,0.);
\draw (9.3,4.910199129935732) node[anchor=north west] {$\Omega$};
\draw (1.7,4.35460478206683) node[anchor=north west] {A};
\draw (7.949306688103378,4.29904534727994) node[anchor=north west] {B};
\draw [line width=1.pt] (6.289909033847214,2.5) circle (2.097024222769593cm);
\draw [line width=1.pt] (3.7100909661527863,2.5) circle (2.097024222769593cm);
\end{tikzpicture}

    \caption{$A \cup B$}
\end{figure}











\textbf{Intersection} of sets $A$,$B$ is a set that contains both the elements of $A$ and $B$:
$$A \cap B = \{n:n \in A  \land  n  \in B\}$$

\par






\subsection*{Sample Space and Events}
The Sample Space, usually denoted as $S$ or $\Omega$, is the $\textit{set}$ of all possible outcomes of an experiment. It is also called  \textbf{universal set}. Subsets of $\Omega$ are called $\textbf{events}$. A sample element of $\Omega$ is denoted as $\omega$.

\begin{example}
    If we toss a six sided dice once, then $\Omega = \{1,2,3,4,5,6\}$, the even that the side is even is $A= \{2,4,6\}$ while $\omega \in \{1,2,3,4,5,6\}$
\end{example}

\begin{example}
    If we toss a two sided coin twice, then $$\Omega = \{(HH),(TT),(HT),(TH)\} \ \land \ \omega \in \{(HH),(TT),(HT),(TH)\}$$
\end{example}
\begin{example}
    If we toss a 2 sided coin forever, then $$\Omega = \{\omega= (\omega_1,\omega_2,...): \ \omega_i \in \{H,T\} \}$$
\end{example}

\begin{example}
    Let $E$ be the event that only even numbers appear in the six sided dice toss. Then,
    $$E = \{2,4,6 \}$$
\end{example}
With the new definition, we can make more set operation: 
$\textbf{complement}$ of the event $A$ is a set of elements $\Omega$ that do not belong to $A$. 
$$A^{c} = \{n: n \in \Omega \land n \not\in A \}$$
\par
$\textbf{difference}$ of the set $A$ from B is a set of elements of $A$ that do not also belong to $B$
$$A \setminus B = A \cap B^c$$
\par we say that $E_1,E_2,...,E_N$ are \textbf{disjoint} if 
$$A_i \cap A_j =  \emptyset $$
\par
A partition of $\Omega$ is a sequence of disjoint events  such that $$\bigcup^{\infty} E_i = \Omega$$

\par 

Similiar to \textbf{monotone functions}, we define \textbf{monotone increasing} sequence of sets $A_1,A_2,...$ as the sequence of sets such that $A_1 \subset A_2 \subset...$ and $\lim_{n \rightarrow \infty} A_n = \bigcup A_i$

\par

Moreover, we can define certain rules similar to the rules of algebra:
$$\begin{aligned}
    &\text{Commutative laws} \qquad  &&A \cup B = B \cup A \\
    &\text{Associative laws} \qquad  &&(A \cup B) \cup C = A \cup (B \cup C) \\
    &\text{Distributive laws} \qquad &&A \cap (B \cup C) = (A \cap B) \cup (A \cap C)
\end{aligned}$$

\par

And lastly, \textbf{DeMorgan's laws} states that
$$ \left( \bigcup_{i=1}^n A_i \right) ^c = \bigcap_{i=1}^n A_i^c $$
$$ \left( \bigcap_{i=1}^n A_i \right) ^c = \bigcup_{i=1}^n A_i^c $$
Which is, in my opinion, very intuitive and can be easily understood with sketching venn diagrams.
These are all of the terminology and notations we will be using for learning the probability.




%--------------------------------------------------





\section{Probability Law}
To show the probability of a event $A$, we assign a real number $P(A)$ or $\mathbb{P}(A)$ in some textbooks, called $\textbf{probability of $A$}$. In other words, $P()$ is a unique function with unique properties that inputs an event $A$, and outputs its probability.
\par 
To qualify as probability, $P$ must satisfy $3$ axioms:
\begin{itemize}
    \item[\textbf{Axiom 1}] $P(A) \ge 0$ \ for every $A$
    \item[\textbf{Axiom 2}] $P(\Omega)=1$
    \item[\textbf{Axiom 3}] If $A_1,A_2,...$ are disjoing: 
    $$P \left( \bigcup^{\infty}_{i=1} A_i \right)= \sum^{\infty}_{i=1}P(A_i) $$
\end{itemize}

\par
Let's explain the axioms. The first axiom is very simple, a probability can't be negative, since the meaning of the word probability.
Second axiom is also very simple, the probability of any possible outcomes happening is $1$, since there must be a outcome at the end of the experiment.
Third axiom, assume we have $2$ disjoint sets. Then
$$P(A \cup B)= P(A)+P(B)$$
This is true simply because sets are disjoint. Similiarly, we can use induction to prove the above property for $n$ sets. Proving for infinite sets are out of scope of this section, therefore we will skip it.

\par
We can derive many properties from these axioms. These are the most simple and intuitive ones:
$$ \begin{aligned}
    P(\emptyset) \qquad &= \qquad 0 \\
    A \subset B \qquad &\Longrightarrow  \qquad P(A) \le P(B) \\
    0 \le       \qquad &P(A) \qquad \le 1 \\ 
    P(A^c)  \qquad  & = \qquad 1- P(A)
\end{aligned}$$ 

And a less obvious property: 
\begin{lemma} For  events $A$ and $B$,
    $$ P \left(A \cup B\right)= P(A)+P(B)-P(A \cap B)$$
\end{lemma}

\begin{proof}
    We can rewrite $A \cup B$ as union of $A \setminus B$, $B \setminus A$, and $A \cap B$, since these are the slices of the thing we want to begin with. Moreover, these slices are disjoint, therefore we can apply our third axiom ($P$ is additive):
$$ 
\begin{aligned} 
    P \left(A \cup B\right) &= P \bigl( (A \setminus B) \cup (B \setminus A) \cup (A \cap B) \bigr) \\
                            &= P(A \setminus B) + P( B \setminus A) + P(A \cap B)  \\
                            &= P(A \setminus B) + P( A \cap B)+ P( B \setminus A) + P(A \cap B) - P(A \cap B)  \\
                            &= P(A)+P(B)-P(A \cap B) 
\end{aligned}
$$ 
\end{proof}


%--------------------------------------------------

\section{Uniform Probability Distribution}