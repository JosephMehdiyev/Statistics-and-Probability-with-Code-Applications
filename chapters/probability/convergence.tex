\chapter{Convergence, CLT and LLN}
\section{Introduction}
We already know the caclulus definition of convergence. We say that $x_n$ \textbf{converges} to $x$ if, for every $\epsilon > 0 $.
\[|x_n - x| < \epsilon \]
as $n$ goes to infinity.  \\
However, there are multiple definitions of convergence in Probability and Statistics. The core idea is simple, in layman terms, as we repeat a process for a long time, something approaches to a thing. Mathematics requires regirousity, and hence there are multiple convergence definitions for the use case. We will learn convergence in the next section \\
\par
\textbf{CLT}, in other words, \textbf{Central Limit theorem}, states that, under correct conditions, the distribution of \textbf{normalized verison} of the sample mean \textbf{converges in distribution} to a standard normal distribution. There are multiple types of CLT. \\

\textbf{LLN}, or \textbf{Law of Large Numbers} states that under right conditions,  the sample average $\overline{X}$ \textbf{convergec in probability}  to the expectation $\mu = \E(X)$. There are also types of LLN. 
\section{Types of Convergence}
There are two main, most used types of convergence in the Statistics and Probability. The weakest one is, the \textbf{convergence in distribution}
\begin{definition}
    Let $X_1,X_2, ...$ be a sequence of r.v  with c.d.f $F_1, \ldots $. $F_n$ is sait to be \textbf{converging in distribution} or \textbf{converge weakly} to c.d.f $F$ of a r.v $X$ if
    \[ \lim_{n \rightarrow \infty} F_n(x) = F(x)\]
    For every $x$ that  $F$ is \textbf{continous}. Convergence in distribution may be denoted as 
    \[ X_n \stackrel{d}{\rightarrow} X \]
    Concept of convergence in distribution \textbf{does not require} $X_n$ to be close to $X$. \\
\end{definition}
More stronger covnergence, \textbf{convergence in probability} is defined as,
\begin{definition}
    Let $X_1,X_2, \ldots $ be a sequence of r.v. $X_n$ is said to be \textbf{converging in probability} to r.v $X$ if for all $\epsilon > 0$,
    \[ \lim_{n \rightarrow \infty} P(|X_n - X| > \epsilon) = 0 \]
    Convergence in probabiltiy is denoted as, 
    \[ X_n \stackrel{p}{\rightarrow} X\]
    Visually, the converge in probability states that, as $n$ grows large $X_n$ tends to be inside the $\epsilon$ brackets of $X$. However, \textbf{it is still possible that $X_n$ goes out of the bound for some time}, but very unlikely. \\
    Almost surely converge is just stronger version of this converge, eliminating the ``out of the bound´´ chance.
\end{definition}
And other two less commonly used convergences,
\begin{definition}
    Sequence $\{ X_n \}$  converges \textbf{Almost surely} or \textbf{strongly} towards $X$ if,
    \[ P \biggl( \lim_{n \rightarrow \infty}|X_n - X | = 0 \biggr) = 1\]
    Almost surely convergence is denoted as, 
    \[ X_n \stackrel{a.s}{\rightarrow} X\]
    We can also rewrite the definition as, (Ferguson 1996, p. 5)
    \[ \lim_{n\to\infty} P\left({\sup_{m\geq n}|X_m-X|>\epsilon }\right) = 0\]
Notice how this equation similar is to probability convergence.\\
This converge states that, there exists a large number $n$ that $X_n$ \textbf{will always be} in the bounds of $\epsilon$.
There is also stronger type of this version, called \textbf{sure convergence}. However, it is rarely used and there is no practical difference between this and the weaker version almost sure convergence. Therefore we don't talk about it.\\
\end{definition}
and lastly, \textbf{convergence in mean},
\begin{definition}
    For  a real number $r \ge 1$, $X_n$ said to be \textbf{converges in r-th mean} to a r.v $X$ if, 
    \[\lim_{x \rightarrow \infty} \E(|X_n - X|^r) = 0 \]
    This convergence is also called $L_r$ convergence or $L^r$ convergence and is denoted as,
    \[ X_n  \stackrel{L^{r}}{\rightarrow} X \]
\end{definition}

\section{Properties of Convergences}
Here is a basic diagram showing the chain of implications, from Wikipedia.
\begin{align*}
    \stackrel{L^{s}}{\longrightarrow} \quad \underset{s>r\ge 1}{\Rightarrow} \quad &\stackrel{L^{r}}{\longrightarrow} \\
    & \ \  \Downarrow \\
    \stackrel{a.s}{\longrightarrow} \quad  \Rightarrow \quad &\stackrel{p}{\longrightarrow} \quad \Rightarrow \quad \stackrel{d}{\longrightarrow}
\end{align*}

\begin{itemize}
    \item Almost surely convergence implies convergence in probability
        \[ X_n \stackrel{a.s}{\longrightarrow} X\quad \Rightarrow \quad X_n \stackrel{p}{\longrightarrow} X\]
    \item Convergence in probability implies convergence in distribution
        \[ X_n \stackrel{p}{\longrightarrow} X \quad \Rightarrow  \quad X_n  \stackrel{d}{\longrightarrow} X\]
    \item Convergence in $L^s$ implies convergence in $L^r$ such that $s > r \ge 1$.
        \[ X_n \stackrel{L^s}{\longrightarrow} X \quad \Rightarrow \quad X_n \stackrel{L^r}{\longrightarrow} X\]
    \item Convergence in $L^r$ implies convergence in probability.
        \[ X_n \stackrel{L^r}{\longrightarrow} X \quad \Rightarrow \quad  X_n \stackrel{p}{\longrightarrow} X\]
\end{itemize}
All of these are intuitively, from the definition, makes sense. No proofs shall be provided.
\begin{theorem} \textbf{Continous mapping theorem} Let $\{ X_n \}$ be r.v sequence, and let $X$ be a r.v. Let $g$ be a continous function. then, the below is true,
    \begin{align*}
        X_n \stackrel{p}{\longrightarrow} X \qquad \Rightarrow \qquad g(X_n)\stackrel{p}{\longrightarrow} g(X) \\
        X_n \stackrel{d}{\longrightarrow} X \qquad \Rightarrow \qquad g(X_n) \stackrel{d}{\longrightarrow} g(X) \\
        X_n \stackrel{a.s}{\longrightarrow} X \qquad \Rightarrow \qquad g(X_n) \stackrel{a.s}{\longrightarrow} g(X)
    \end{align*}
\end{theorem}
Note to myself: Add addivity and multiplicative property with its proof, Slutzky's theorem.
\section{LLN, Law of Large Numbers}
There are \textbf{strong} and \textbf{weak} type of LLN, Law of large Numbers. However, both of them state the same idea, $\overline{X_n}$ converges to $\mu$ as  $n$ goes to infinity.
\begin{theorem}
    \textbf{The weak Law of Large Numbers} or shortly \textbf{WLLN} states that if $X_1, \ldots X_n$ are i.i.d r.vs, then,
    \[ \overline{X}_n \stackrel{p}{\longrightarrow} \mu\]
    \begin{proof}
        For the sake of simplicity, assume variance is finite.
        The theorem is the direct consequence of  Chebysev's inequality.
        \[ P( |\overline{X}_n - \mu| \ge \epsilon) \le \frac{\V(\overline{X}_n)}{\epsilon^2} = \frac{\V(X_1)}{n\epsilon^2}\]
    Which right side obviously converges to $0$.
    \end{proof}
\end{theorem}
\begin{theorem}
    \textbf{The strong Law of Large Number} or shortly SLLN states that if $X_1, \ldots,X_n$ are i.i.d r.vs and $\mu < \infty$, then,
    \[ \overline{X}_n \stackrel{a.s}{\longrightarrow} \mu \]
    Proof is complex, so I will avoid giving it here.
\end{theorem}
Practically there is not much difference between WLLN and SLLN, therefore SLLN is preferred.
\section{CLT, Central Limit Theorem}
\begin{theorem}
    \textbf{Central Limit Theorem}, or shortly \textbf{CLT} states that for i.i.d r.vs $X_1, \ldots X_n$ with mean $\mu$ and variance $\sigma^2$,
    \[ \frac{\overline{X}_n - \mu}{\sqrt{\V ( \overline{X}_n)}} \stackrel{d}{\longrightarrow} N(0,1)\]
\end{theorem}

\section{References}
\begin{enumerate}
    \item \url{https://www.stat.cmu.edu/~larry/=stat325.01/chapter5.pdf} 
    \item \url{https://en.wikipedia.org/wiki/Convergence_of_random_variables}
    \item \url{https://imai.fas.harvard.edu/teaching/files/Convergence.pdf}
    \item \url{https://stats.stackexchange.com/questions/2230/convergence-in-probability-vs-almost-sure-convergence}
    \item \url{Lafaye de Micheaux, P.,  Liquet, B. (2009). Understanding Convergence Concepts: A Visual-Minded and Graphical Simulation-Based Approach. The American Statistician, 63(2), 173–178. doi:10.1198/tas.2009.0032}
    \item \url{https://stats.stackexchange.com/questions/3734/what-intuitive-explanation-is-there-for-the-central-limit-theorem}
\end{enumerate}

\section{Exercises}
