\chapter*{Preface}

\section*{About the Book}
The reason I wrote this is because I wanted to learn how the academic writings are made, also to learn the statistics field in a unique way. The main purpose of the book is to be a handbook or a notebook for my personal uses. You can find source code below.\\
The book consists of $3$ parts. \textbf{Probability, Statistical Inference and Data mining}. Each section have multiple chapters, references and exercises. Each exercise subsection have \textbf{R simulation practices with visualizations}
\par
The chapters are structured in a very abstract and short way. When I wrote this, I imagined that I was talking with myself. I really don't know how to explain it, I recommend skimming through the book to understand the way the book is written.
\par
To learn the field and write this book, I used various books from known authors, countless mathematics forums about statistics and probability, wikipedia (duh-duh) articles, and MIT OpenCourseWare. These are all the major sources I used while writing the book.
\begin{itemize}
    \item Larry Wasserman - All of Statistics - A Concise Course in Statistical Inference.
    \item Probability and Statistics by Morris H. DeGroot and Mark J. Schervish.
    \item R for Data Science.
    \item An Introduction To Statistical Learning with R/Python

    \item MIT OpenCourseWare Introduction to Probability and Statistics, spring 2022.
\end{itemize}

The OpenCourseWare has its own license, you can find it here: \url{https://creativecommons.org/licenses/by-nc-sa/4.0/}


\section*{Book's source}
This book is fully open source with its source code shared in author's \href{https://github.com/JosephMehdiyev}{github}. You may use the source code for whatever purposes you want to use it for. If you have questions about the book or its source code, you may contact me with my email: yusifmehdiyev55@gmail.com 
\section*{How to use the Book}
As the book is precise and short, you may use the book as a revisit or a secondary material. The book shortly and simply explains the concepts and ideas. Important concepts' proofs are provided. However, other proofs explaned in sentences rather than other classic rigorous proofs.

\section*{R source code}
In this book, I mainly used \textbf{tidyverse} packages to plot, visualize, model, clean data and more. My IDE for R is \textbf{RStudio}. You can find R source code in my github, in src directory.
